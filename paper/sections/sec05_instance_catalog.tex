%\begin{itemize}
%  \item Coin-SMI (\url{https://github.com/coin-or/Smi}): compile with CPLEX; can read SMPS and solve in extensive form.
%  \item DSP: Benders/dual decomposition
%  \item PySP: Progressive Hedging
%\end{itemize}
%
%- problem characteristics
%
%- preliminary results using CPLEX

%\kk{Please describe what you present in this section. Subsectioning would also help.} \yoc{This section is newely written. Need more justification.} 
\siplibtwo\ provides some pregenerated \smps\ instances. Throughout this section, we report detailed information for each instance. This includes size, sparsity, and computational results. In particular, the purpose of reporting the computational results is largely divided into two folds: computational benchmarking and characterizing the instances. For example, the LP2-Relax gap, which is defined by the objective gap between the original recourse problem and the problem with LP-relaxed second-stage, quantifies the tightness of the formulation. Refering the report in this section, we hope the users to find instances that fit for their own purpose.

%\begin{itemize}
%	\item Commercial MIP solver (CPLEX) result: optimality gap, integrality gap (LP relaxation gap, tightness), solution time, obj value
%	\item DD solver (DSP) result: optimality gap, duality gap (convex relaxation gap), solution time, obj value
%\end{itemize}

\subsection{Size and sparsity information}
Table \ref{table:instance_size_info_dcap}-\ref{table:instance_size_info_suc} in Section \ref{sec:instance_size_sparsity} report size and sparsity information for each instance. The information is reported both block-wisely and totally (i.e., based on EF). For example, the second-stage size information reports the number of variables in a single-scenario block (block W in Fig. \ref{fig:stagewise_sparsity}). Sparsity is reported in percentage (\%) and the file size is reported in kilobytes (KB).

\subsection{Computational results}
In Table \ref{table:computation_report}, we report computational information investigated from experiments. Unlike the size and sparsity, this should be obtained in highly computational way, which implies that the result is strongly affected depending on the computing environment in which experiments are performed. Throughout the following subsections, we explain about our computational experiments.

\subsubsection{Experimental setting}
\yoc{Experiment plan}
\begin{itemize}
	\item computer: Bebop
	\item running time: 3 hours/instance?
	\item resource: single-node with 36 cores (1 thread/core) per instance
	\item solvers (default parameter setting): CPLEX 12.8, DSP
	\item How to obtain gaps?
		\begin{itemize}
			\item Optimality gap: CPLEX (log file), DSP (log file)
			\item LP2-Relax gap
			\begin{itemize}
				\item $\hat{z}$: the best solution amongst CPLEX, SCIP, DSP
				\item $\hat{z}_{LP2}$: LP-relaxed SMPS instance will be generated (from Siplib.jl) and solve by CPLEX 
			\end{itemize}
			\item REVPI
			\begin{itemize}
				\item $\hat{z}$: the best solution amongst CPLEX, SCIP, DSP
				\item $\hat{z}_{WS}$: \texttt{Siplib.WS()}
			\end{itemize}
			\item RVSS
			\begin{itemize}
				\item $\hat{z}$: the best solution amongst CPLEX, SCIP, DSP
				\item $\hat{z}_{EEV}$: \texttt{Siplib.EEV()}
			\end{itemize}
		\end{itemize}
	\item Standard deviation
	\item experiment goal
\end{itemize}

%Table \ref{table:experimental_setting} summarizes the experimental setting. 
All the experiments were run on \textit{Bebop}, a 1024-node computing cluster at Argonne National Laboratory \cite{bebop}. Each node in Bebop has 128GB DDR4 memory and two sockets for Intel Xeon Processor E5-2695 v4 that consists of 18 cores each of which has single thread hence 36 cores as well as threads per node in total. The base processor frequency is 2.10GHz and maximum turbo frequency is 3.30GHz. Note that we only use 1 node for each instance.

We use two solvers, a general purpose commercial MIP solver \cplex\ (version 12.8) and an open-source DD-based SIP solver \dsp. For solving master/sub problems during DD procedure in \dsp, we also use \cplex. Since \cplex\ does not support \smps\ as its input format, we use \dsp's capability to solve the EF by reading \smps\ files. We fully utilize 36 threads for each single instance as the both two solvers support multi-threads parallel computing. We set the termination condition on CPU time by 3 hours. 

The instances are generated in \smps\ format using the \julia\ package we introduced in Section \ref{sec:package} with default random seed 1. The sizes of the instances are chosen basically based on the original references. Additionally, we include some large-scenario instances when we could not find meaningful result from the small-scenario instances.

% Please add the following required packages to your document preamble:
% \usepackage{multirow}
%\begin{table}[H]
%	\centering
%	\caption{Experimental setting}
%	\label{table:experimental_setting}
%	\resizebox{\textwidth}{!}{%
%	\begin{tabular}{|l|l|}
%		\hline
%		\multirow{4}{*}{Computing environment} & Bebop (ANL 1024-node computing cluster) with each node      \\
%		& - CPU: Intel Xeon Processor E5-2695 v4 (18 cores 36 threads) \\
%		& - Clock speed: 2.10GHz (maximum 3.30GHz)                     \\
%		& - Memory:128GB (45MB Smart cache)                            \\ \hline
%		\multirow{2}{*}{Solver}                & Commercial MIP solver: CPLEX 12.8                                \\
%		& Open-source SIP solver: DSP                                      \\ \hline
%		Threads per instance                 & 36                                                           \\ \hline
%		Time limit                             & 3 hours                                                      \\ \hline
%		Instance size                          & SIPLIB instance sizes + Large scenario instance                  \\ \hline
%	\end{tabular}
%	}
%\end{table}

\subsubsection{Solution report} \label{subsec:sol_report}
We first introduce the notations used to define each column in the report Table \ref{table:computation_report}. The summary for the notations can be found in Table \ref{table:objective-notation}.

First, recall the extensive form RP in Section \ref{sec:sip}:
\begin{subequations}\label{ef}
	\begin{align}
	\textrm{(RP-EF) }\min_{x,\mathrm{y}}\ z(x,\mathrm{y}):=&\ c^{\top}x + \sum_{s=1}^{r}\PP(s) (q_s^{\top}y_s) \tag{\ref{ef:obj}}\\ 
	\mathrm{s.t.}\ &Ax\ge b,  \tag{\ref{ef:b}}\\
	&T_s x+W_s y_s\ge h_s,\quad\forall s\in\{1,\ldots,r\}, \tag{\ref{ef:c}} \\
	&x\in X, \tag{\ref{ef:d}} \\
	&y_s \in Y,\quad\forall s\in\{1,\ldots,r\}. \tag{\ref{ef:e}}
	\end{align}
\end{subequations}

We use the vanilla $z$ to denote the value of the objective function in RHS of (\ref{ef:obj}) such that constraints (\ref{ef:b})-(\ref{ef:e}) are satisfied throughout this section and the report. We use the superscripted notation $z^*$ to denote the true optimal objective value, $\tilde{z}$ for the best lower bound discovered for $z^*$, and $\hat{z}$ for the incumbent (best known integer solution) objective value. Hence, the relationship $\tilde{z}\le z^*\le\hat{z}$ always holds.

$z_{LP2}$ is defined by the objective value of the following partly LP-relaxed problem:
\begin{subequations}\label{ef-lp2}
	\begin{align}
	\textrm{(RP-LP2) }\min_{x,\mathrm{y}}\ z_{LP2}(x,\mathrm{y}):=&\ c^{\top}x + \sum_{s=1}^{r}\PP(s) (q_s^{\top}y_s) \tag{\ref{ef:obj}}\\ 
	\mathrm{s.t.}\ &\textrm{(\ref{ef:b})-(\ref{ef:d})}, \nonumber \\
	& y_s \textrm{ is continuous} ,\quad\forall s\in\{1,\ldots,r\}. \tag{\ref{ef:e}'}
	\end{align}
\end{subequations}
Note that only the integrality in the second-stage variables is relaxed.

$z_{LD}$ is defined by the objective value of the associated Lagrangian dual problem. The Lagrangian dual problem can be derived from RP-EF' (\ref{sip:ef'}) by introducing the nonanticipativity constraints (\ref{eq:SIP_2-2}) into the objective function (\ref{eq:SIP_2-1}) using Lagrange multiplier $\lambda$:
\begin{subequations}\label{ld}
	\begin{align}
	\textrm{(RP-LD) }\max_{\lambda\in\mathbb{R}^{rn_1}}\ z_{LD}(\lambda) :=\min_{\mathrm{x},\mathrm{y}}  &\sum_{s=1}^{r}\left[\PP(s)\left(c^{\top}x_s+q_s^{\top}y_s\right)+\lambda^\top \left(H_s x_s\right)\right] \label{ld:obj}\\ 
	\mathrm{s.t.}\ & (x_s,y_s)\in G_s,\quad \forall s\in\{1,\ldots,r\},	 \label{ld:b}
%	&\lambda\in\mathbb{R}^{rn_1}, \label{ld:c}
	\end{align}
\end{subequations}
where the scenario feasibility set $G_s$ is still defined by (\ref{eq:SIP_2-4}).

$z_{EV}$ is the objective value of the \textit{expected value problem}, which simply considers only the averaged scenario data, $\bar{q}$, $\bar{h}$, $\bar{T}$, $\bar{W}$:
\begin{subequations}\label{ev}
	\begin{align}
	\min_{x,y}\ z_{EV}(x,y):=&\ c^{\top}x + \bar{q}^{\top}y \label{ev:obj}\\ 
	\mathrm{s.t.}\ &Ax\ge b,  \label{ev:b}\\
	&\bar{T} x+\bar{W} y\ge \bar{h},  \label{ev:c}\\
	&x\in X,  \label{ev:d}\\
	&y \in Y. \label{ev:e}
	\end{align}
\end{subequations}
$z_{EEV}$ is then defined by the \textit{expected result of using the first-stage solution of EV}, say $\bar{x}$: 
\begin{subequations}\label{eev}
	\begin{align}
	\min_{\mathrm{y}}\ z_{EEV}(\mathrm{y}):=&\ c^{\top}\bar{x} + \sum_{s=1}^{r}\PP(s) (q_s^{\top}y_s) \label{eev:obj}\\ 
	\mathrm{s.t.}\ &W_s y_s\ge h_s-T_s \bar{x},\quad\forall s\in\{1,\ldots,r\}, \label{eev:b}\\
	&y_s \in Y,\quad\forall s\in\{1,\ldots,r\}. \label{eev:c}
	\end{align}
\end{subequations}

$z_{WS}$ is called the \textit{wait-and-see solution value} defined by
\begin{align} \label{ws}
	z_{WS}:=\sum_{s=1}^r \PP(s)z^*(x,y_s),
\end{align}
where $z^*(x,y_s)$ is the optimal objective value of a \textit{single scenario problem} given a scenario $s$: 
\begin{subequations}
	\begin{align}
	\min_{x,y_s}\ z_{SS(s)}(x,y_s):=&\ c^\top x+q_s^\top y_s \label{ssp:obj}\\
	\mathrm{s.t.}\ &Ax \ge b \label{ssp:b}\\
	&T_s x + W_s y_s\ge h_s,  \label{ssp:c}\\
	&x\in X,  \label{ssp:d}\\
	&y_s \in Y. \label{ssp:e}
	\end{align}
\end{subequations}
Noticeably, the inequality $z^*_{WS}\le z^*\le z^*_{EEV}$ is known to always hold \cite{journal:SKS2013}.

Finally, we define the following \textit{standard deviation of the recourse problem}:
\begin{align} 
\sigma_{RP}&:=\sqrt{\frac{1}{|\mathcal{S}|-1}\sum_{s=1}^{|\mathcal{S}|}\left[z^*(x^*,y_s)-z^*\right]^2}, \label{sd}
\end{align}
where $z^*(x^*,y_s)$ is the optimal objective value of a \textit{single scenario problem with a fixed first-stage solution $x^*$ of RP-EF} and a given scenario $s$. That is,
\begin{subequations}
	\begin{align}
	\min_{y_s}\ z_{SS(s)}(x^*,y_s):=&\ c^\top x^*+q_s^\top y_s \label{ssp_ffs:obj}\\
	\mathrm{s.t.}\ &	W_s y_s\ge h_s-T_s x^*,  \label{ssp_ffs:b}\\
	&y_s \in Y. \label{ssp_ffs:c}
	\end{align}
\end{subequations}

\begin{table}[H]
	\centering
	\caption{Solution notation}
	\label{table:objective-notation}
	\resizebox{\textwidth}{!}{%
	\begin{threeparttable}
%		\begin{tabular}{@{}ll@{}}
		\begin{tabular}{@{}ll@{}}
			\toprule
			Notation  & \multicolumn{1}{c}{Meaning}                                                                                                                    \\ \midrule
			$z$       & Objective value of RP (\ref{ef:obj})-(\ref{ef:e})                                                                                         \\
			$z_{LP2}$ & Objective value of RP with the second-stage only LP-relaxation (\ref{ef:obj})-(\ref{ef:d}) and (\ref{ef:e}')                                              \\
			$z_{LD}$  & Objective value of the Lagrangian dual problem of RP (\ref{ld:obj}) and (\ref{ld:b}) \\ 
			$z_{EV}$ & Objective value of the expected value problem (\ref{ev:obj})-(\ref{ev:d})\\
			$z_{EEV}$ & Objective value of RP using the first-stage solution obtained from (\ref{eev:obj})-(\ref{eev:c})\\
			$z_{SS(s)}$ & Objective value of a single scenario problem given scenario $s$ (\ref{ssp:obj})-(\ref{ssp:e})\\
			$z_{WS}$ & Averaged value of all single scenario problem objectives $z_{SS(s)}$\\
			\bottomrule
		\end{tabular}%
				
		\begin{tablenotes}
			\small
			\item $z^*$: true optimal objective value
			\item $\tilde{z}$: best discovered lower bound for $z^*$ 
			\item $\hat{z}$: incumbent objective value (best  discovered upper bound)
		\end{tablenotes}
	\end{threeparttable}
	}
\end{table}
%\subsubsection {Basic results from the two solvers: \cplex\ and \dsp}
In Table \ref{table:computation_report}, we report objective value, standard deviation, and several gaps. Each gap is useful to characterize the instances. There is no guarantee that we can always obtain the optimal solution for each instance, we report the best results discovered within the limited processing time (3 hours). Table \ref{table:gap} summarizes the definition of each gap using the notation defined in Table \ref{table:objective-notation}.
%\begin{table}[H]
%	\centering
%	\caption{Gap definition}
%	\label{table:gap}
%	\begin{tabular}{@{}lcc@{}}
%		\toprule
%		Gap                  & Absolute value definition  & Reported value  \\ \midrule
%		ROG  & $N/A$ & $\frac{|\hat{z}-\tilde{z}|}{|\hat{z}|+\epsilon}\times 100\%$   \\
%		RIG &  $z^*-z^*_{LP2}$          &        $\frac{\hat{z}}{\hat{z}_{LP2}}$       \\
%		RDG     &      $z^*-z^*_{LD}$     &        $\frac{|\hat{z}-\hat{z}_{LD}|}{|\hat{z}|+\epsilon}\times 100\%$        \\
%		REVPI                 &      $z^*-z^*_{WS}$      &       $\frac{|\hat{z}-\hat{z}_{WS}|}{|\hat{z}|+\epsilon}\times 100\%$         \\ 
%		RVSS	& $z^*_{EEV}-z^*$	&	\\\bottomrule
%	\end{tabular}
%\end{table}


\begin{table}[H]
	\centering
	\caption{Gap definitions}
	\label{table:gap}
	\begin{tabular}{lcc}
		\hline
		Gap   & Target to compare ($x$) & Reported value                                                        \\ \hline
		Optimality Gap (OG)   & $\tilde{z}$, $\hat{z}_{LD}$    & \multirow{4}{*}{$\frac{|\hat{z}-x|}{|\hat{z}|+\epsilon}\times 100\%$} \\
		LP2-Relax Gap (LG)   & $\hat{z}_{LP2}$    &                                                                       \\
%		Duality gap (DG)   &      &                                                                       \\
		Relative EVPI (REVPI) & $\hat{z}_{WS}$     &                                                                       \\
		Relative VSS (RVSS)  & $\hat{z}_{EEV}$    &                                                                       \\ \hline
	\end{tabular}
\end{table}

\begin{quote}
	\noindent\textit{Objective Value} We report the objective values of the recourse problem from the two solvers. This value is $z^*$ if the optimality is attained within the time limit, $\hat{z}$ otherwise. Additionally, we report the standard deviation of the recourse problem ($\sigma_{RP}$) as defined in Equation (\ref{sd}) together.
\end{quote}

\begin{quote}
	\noindent\textit{Optimality Gap (OG)} This gap is only computational concept defined by the relative difference between best upper and lower bounds during an MIP solving procedure. For \dsp\ solver, this gap is obtained from the difference between primal objective and dual objective: $\hat{z}-\hat{z}_{LD}$. We report the relative value obtained from the three solvers (\cplex, \dsp). This can be a basic measure of difficulty of an instance combined with processing time.
\end{quote}

\begin{quote}
	\noindent\textit{LP2-Relax Gap (LG)} We define this gap by the difference in optimal objective values between RP and RP-LP2: $z^*-z^*_{LP2}$. We report the relative incumbent value: $\frac{|\hat{z}-\hat{z}_{LP2}|}{|\hat{\hat{z}}|+\epsilon}\times 100\%$.
	
%	We define a SIP-specific theoritic integrality gap in minimization problem by $\frac{z^*}{z^*_{LP2}}$. Computationally, we report $\frac{\hat{z}}{\hat{z}_{LP2}}$ when the optimality is not attained. Assuming $\hat{z}_{LP2}\approxeq z^*_{LP2}$ in most cases, the reported IG will provide upper bound for the true IG. The LP-relaxation provides an optimistic bound on the integer program's solution. The small this gap implies that the LP-relaxation gives tight bound for the original problem.  %This gap cannot be obtained analytically so we report it based on the computational experiments.% The formulation with tighter LP relaxation is always favorable.
\end{quote}

%\begin{quote}
%\noindent\underline{\textit{Integrality gap (IG)}} We define a SIP-specific integrality gap (IG) in minimization problem by $\frac{z^*}{z^*_{LP2}}$, where $z^*$ is the optimal integral solution and $z^*_{LR2}$ is the optimal fractional solution with LP-relaxed second-stage. This LP relaxation provides an optimistic bound on the integer program's solution. The small this gap implies tightness of the LP relaxation. Since there is no guarantee that we always find the optimal solutions, we report $\frac{\hat{z}}{\hat{z}_{LP2}}$ where $\hat{z}$ and $\hat{z}_{LP2}$ are the incumbent solutions. Assuming $\hat{z}_{LP2}\approxeq z^*_{LP2}$ in most cases, the reported IG will provide upper bound for the true IG. The tightness of this bound highly depends on the quality of the incumbent solution $\hat{z}$.%This gap cannot be obtained analytically so we report it based on the computational experiments.% The formulation with tighter LP relaxation is always favorable.
%\end{quote}
%
%\begin{quote}
%	\noindent\underline{\textit{Duality gap (DG)}} The duality gap in computational optimization is the difference between any dual objective and the incumbent primal objective. This gap also equals to the value of the convex relaxation of the primal problem. It is reported by the open-source DD solver \dsp.
%\end{quote}

%\begin{quote}
%\noindent\textit{Duality gap (DG)} This gap is the difference between primal objective and dual objective: $z^*-z^*_{LD}$. This gap also equals to the value of the convex relaxation of the primal problem. We report the relative incumbent value: $\frac{|\hat{z}-\hat{z}_{LD}|}{|\hat{\hat{z}}|+\epsilon}\times 100\%$. It is reported by the open-source DD solver \dsp. 
%\end{quote}

\begin{quote}
\noindent\textit{Relative EVPI (REVPI)} This gap is the relative version of the \textit{expected value of perfect information} (EVPI) which is defined by $z^*-z^*_{WS}$ theoritically. This measures the maximum amount a decision maker would be ready to pay in return for complete (and accurate) information about the future. We report $\frac{|\hat{z}-\hat{z}_{WS}|}{|\hat{\hat{z}}|+\epsilon}\times 100\%$ using \cplex.
\end{quote}

\begin{quote}
	\noindent\textit{Relative VSS (RVSS)} This gap is the relative version of the \textit{value of the stochastic solution} (VSS) which is defined by $z^*_{EEV}-z^*$. This allows us to obtain the goodness of the expected solution value when the expected values are replaced by the random values for the input variables. We report $\frac{|\hat{z}-\hat{z}_{EEV}|}{|\hat{\hat{z}}|+\epsilon}\times 100\%$ using \cplex.
\end{quote}

\begin{table}[]
	\centering
	\caption{Computational report}
	\label{table:computation_report}
	\resizebox{\textwidth}{!}{%
		\begin{tabular}{|c|l|ll|ll|ll|l|l|l|}
			\hline
			\multicolumn{1}{|l|}{}  &                               & \multicolumn{2}{c|}{Objective value}                                 & \multicolumn{2}{c|}{Optimality Gap (\%)}             & \multicolumn{2}{c|}{Integrality Gap (\%)}     & \multicolumn{1}{c|}{Duality Gap (\%)} & \multicolumn{1}{c|}{REVPI (\%)} & \multicolumn{1}{c|}{RVSS (\%)}\\ \cline{3-11} 
			Problem                 & \multicolumn{1}{c|}{Instance} & \multicolumn{1}{c}{CPLEX (time)} & \multicolumn{1}{c|}{DSP (time)} & \multicolumn{1}{c}{CPLEX} & \multicolumn{1}{c|}{DSP} & \multicolumn{1}{c}{CPLEX} & \multicolumn{1}{c|}{DSP} & \multicolumn{1}{c|}{DSP}              & \multicolumn{1}{c|}{CPLEX}    & \multicolumn{1}{c|}{CPLEX}  \\ \hline
			\multirow{12}{*}{DCAP}  & DCAP\_2\_3\_3\_200            &                                   &                                  &                           &                          &                           &                          &                                       &                       &         \\
			& DCAP\_2\_3\_3\_300            &                                   &                                  &                           &                          &                           &                          &                                       &                               & \\
			& DCAP\_2\_3\_3\_500            &                                   &                                  &                           &                          &                           &                          &                                       &                               & \\
%			& DCAP\_2\_3\_3\_10000          &                                   &                                  &                           &                          &                           &                          &                                       &                               & \\
			& DCAP\_2\_4\_3\_200            &                                   &                                  &                           &                          &                           &                          &                                       &                               & \\
			& DCAP\_2\_4\_3\_300            &                                   &                                  &                           &                          &                           &                          &                                       &                               & \\
			& DCAP\_2\_4\_3\_500            &                                   &                                  &                           &                          &                           &                          &                                       &                               & \\
%			& DCAP\_2\_4\_3\_10000          &                                   &                                  &                           &                          &                           &                          &                                       &                               & \\
			& DCAP\_3\_3\_2\_200            &                                   &                                  &                           &                          &                           &                          &                                       &                               & \\
			& DCAP\_3\_3\_2\_300            &                                   &                                  &                           &                          &                           &                          &                                       &                               & \\
			& DCAP\_3\_3\_2\_500            &                                   &                                  &                           &                          &                           &                          &                                       &                               & \\
%			& DCAP\_3\_3\_2\_10000          &                                   &                                  &                           &                          &                           &                          &                                       &                               & \\
			& DCAP\_3\_4\_2\_200            &                                   &                                  &                           &                          &                           &                          &                                       &                               & \\
			& DCAP\_3\_4\_2\_300            &                                   &                                  &                           &                          &                           &                          &                                       &                               & \\
			& DCAP\_3\_4\_2\_500            &                                   &                                  &                           &                          &                           &                          &                                       &                               & \\
%			& DCAP\_3\_4\_2\_10000          &                                   &                                  &                           &                          &                           &                          &                                       &                               & \\ 
\hline
			\multirow{4}{*}{MPTSPs} & MPTSPs\_D0\_50\_100           &                                   &                                  &                           &                          &                           &                          &                                       &                               & \\
			& MPTSPs\_D1\_50\_100           &                                   &                                  &                           &                          &                           &                          &                                       &                               & \\
			& MPTSPs\_D2\_50\_100           &                                   &                                  &                           &                          &                           &                          &                                       &                               & \\
			& MPTSPs\_D3\_50\_100           &                                   &                                  &                           &                          &                           &                          &                                       &                               & \\
%			& MPTSPs\_D0\_100\_100          &                                   &                                  &                           &                          &                           &                          &                                       &                               & \\
%			& MPTSPs\_D1\_100\_100          &                                   &                                  &                           &                          &                           &                          &                                       &                               & \\
%			& MPTSPs\_D2\_100\_100          &                                   &                                  &                           &                          &                           &                          &                                       &                               & \\
%			& MPTSPs\_D3\_100\_100          &                                   &                                  &                           &                          &                           &                          &                                       &                               & \\ 
\hline
			\multirow{4}{*}{SMKP}   & SMKP\_120\_20                 &                                   &                                  &                           &                          &                           &                          &                                       &                               & \\
			& SMKP\_120\_100                &                                   &                                  &                           &                          &                           &                          &                                       &                               & \\
			& SMKP\_120\_300                &                                   &                                  &                           &                          &                           &                          &                                       &                               & \\
			& SMKP\_120\_500                &                                   &                                  &                           &                          &                           &                          &                                       &                               & \\
%			& SMKP\_120\_1000               &                                   &                                  &                           &                          &                           &                          &                                       &                               & \\ 
\hline
			\multirow{4}{*}{SIZES}  & SIZES\_3                      &                                   &                                  &                           &                          &                           &                          &                                       &         &                       \\
			& SIZES\_5                      &                                   &                                  &                           &                          &                           &                          &                                       &                               & \\
			& SIZES\_10                     &                                   &                                  &                           &                          &                           &                          &                                       &                               & \\
%			& SIZES\_10000                  &                                   &                                  &                           &                          &                           &                          &                                       &                               & \\ 
\hline
			\multirow{8}{*}{SSLP}  & SSLP\_5\_25\_50               &                                   &                                  &                           &                          &                           &                          &                                       &                               & \\
			& SSLP\_5\_25\_100              &                                   &                                  &                           &                          &                           &                          &                                       &                               & \\
%			& SSLP\_5\_25\_2000             &                                   &                                  &                           &                          &                           &                          &                                       &                               & \\
			& SSLP\_5\_50\_50               &                                   &                                  &                           &                          &                           &                          &                                       &                               & \\
			& SSLP\_5\_50\_100              &                                   &                                  &                           &                          &                           &                          &                                       &                               & \\
%			& SSLP\_5\_50\_2000             &                                   &                                  &                           &                          &                           &                          &                                       &                               & \\
			& SSLP\_10\_50\_50              &                                   &                                  &                           &                          &                           &                          &                                       &                               & \\
			& SSLP\_10\_50\_100             &                                   &                                  &                           &                          &                           &                          &                                       &                               & \\
%			& SSLP\_10\_50\_2000            &                                   &                                  &                           &                          &                           &                          &                                       &                               & \\
			& SSLP\_15\_45\_50              &                                   &                                  &                           &                          &                           &                          &                                       &                               & \\
			& SSLP\_15\_45\_100             &                                   &                                  &                           &                          &                           &                          &                                       &                               & \\
%			& SSLP\_15\_45\_2000            &                                   &                                  &                           &                          &                           &                          &                                       &                               & \\ 
\hline
			\multirow{16}{*}{SUC}   & SUC\_FallWD\_10               &                                   &                                  &                           &                          &                           &                          &                                       &        &                        \\
			& SUC\_FallWD\_50               &                                   &                                  &                           &                          &                           &                          &                                       &                               & \\
%			& SUC\_FallWD\_100              &                                   &                                  &                           &                          &                           &                          &                                       &                               & \\
			& SUC\_FallWE\_10               &                                   &                                  &                           &                          &                           &                          &                                       &                               & \\
			& SUC\_FallWE\_50               &                                   &                                  &                           &                          &                           &                          &                                       &                               & \\
%			& SUC\_FallWE\_100              &                                   &                                  &                           &                          &                           &                          &                                       &                               & \\
			& SUC\_WinterWD\_10             &                                   &                                  &                           &                          &                           &                          &                                       &                               & \\
			& SUC\_WinterWD\_50             &                                   &                                  &                           &                          &                           &                          &                                       &                               & \\
%			& SUC\_WinterWD\_100            &                                   &                                  &                           &                          &                           &                          &                                       &                               & \\
			& SUC\_WinterWE\_10             &                                   &                                  &                           &                          &                           &                          &                                       &                               & \\
			& SUC\_WinterWE\_50             &                                   &                                  &                           &                          &                           &                          &                                       &                               & \\
%			& SUC\_WinterWE\_100            &                                   &                                  &                           &                          &                           &                          &                                       &                               & \\
			& SUC\_SpringWD\_10             &                                   &                                  &                           &                          &                           &                          &                                       &                               & \\
			& SUC\_SpringWD\_50             &                                   &                                  &                           &                          &                           &                          &                                       &                               & \\
%			& SUC\_SpringWD\_100            &                                   &                                  &                           &                          &                           &                          &                                       &                               & \\
			& SUC\_SpringWE\_10             &                                   &                                  &                           &                          &                           &                          &                                       &                               & \\
			& SUC\_SpringWE\_50             &                                   &                                  &                           &                          &                           &                          &                                       &                               & \\
%			& SUC\_SpringWE\_100            &                                   &                                  &                           &                          &                           &                          &                                       &                               & \\
			& SUC\_SummerWD\_10             &                                   &                                  &                           &                          &                           &                          &                                       &                               & \\
			& SUC\_SummerWD\_50             &                                   &                                  &                           &                          &                           &                          &                                       &                               & \\
%			& SUC\_SummerWD\_100            &                                   &                                  &                           &                          &                           &                          &                                       &                               & \\
			& SUC\_SummerWE\_10             &                                   &                                  &                           &                          &                           &                          &                                       &                               & \\
			& SUC\_SummerWE\_50             &                                   &                                  &                           &                          &                           &                          &                                       &                               & \\
%			& SUC\_SummerWE\_100            &                                   &                                  &                           &                          &                           &                          &                                       &                               & \\ 
		\hline
		\end{tabular}%
	}
\end{table}

%\begin{quote}
%\noindent\underline{\textit{Symmetry gap (SG)}} An MIP is \textit{symmetric} if its variables can be permuted without changing the structure of the problem. The symmetry is one of the frequently mentioned factors that affects the difficulty of optimization problem. Unfortunately, no universally applicable quantification for the impact of symmetry is available. Hence, we heuristically define the symmetry gap using \cplex\ option (Table \ref{table:cplex_symmetry}). That is,  
%\end{quote}

%\begin{quote}
%	\centering$\frac{|\textrm{\cplex\ solution}-\textrm{\cplex\ solution with the strongest symmetry breaking}|}{|\textrm{\cplex\ solution}|}$
%\end{quote}

%\begin{table}[H]
%	\centering
%	\resizebox{\textwidth}{!}{%
%		\begin{threeparttable}
%			\caption{Symmetry breaking option of CPLEX }
%			\label{table:cplex_symmetry}
%			\begin{tabular}{@{}cl@{}}
%				\toprule
%				Parameter value & \multicolumn{1}{c}{Meaning}                              \\ \midrule
%				-1    & Automatic: let CPLEX choose; default                     \\
%				0     & Turn off symmetry breaking                               \\
%				1     & Exert a moderate level of symmetry breaking              \\
%				2     & Exert an aggressive level of symmetry breaking           \\
%				3     & Exert a very aggressive level of symmetry breaking       \\
%				4     & Exert a highly aggressive level of symmetry breaking     \\
%				5     & Exert an extremely aggressive level of symmetry breaking \\ \bottomrule
%			\end{tabular}
%			\begin{tablenotes}
%				\small
%				\item *Parameter in \texttt{C} API is 	$\texttt{CPXPARAM\_Preprocessing\_Symmetry}$ for $>$12.6.0 and $\texttt{CPX\_PARAM\_SYMMETRY}$ for $\le$12.6.0
%			\end{tablenotes}
%		\end{threeparttable}
%	}
%\end{table}