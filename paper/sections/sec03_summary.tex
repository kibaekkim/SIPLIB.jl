In this section, we explain information about the test problems in summarized manner. This includes problem origin, type, components (variables and constraints), and sparsity for each problem. This section reflects our philosophy in developing \siplibtwo. Detailed problem-specific information is available in Section \ref{sec:prob_desc} for those who are interested in.

\subsection{Origin of the problems}
Table \ref{table:problems} summarizes the description of each test problem in \siplibtwo. The five of them are adopted from \siplib\ \cite{web:SIPLIB1}. We tried to implement the \siplib\ instances as the same as the original references possible. Not all of them, however, are exactly the same due to insufficient information on some parameters. We guess the missing links and develop our own way to implement the problems as long as it does not harm the endemic characteristic. 
\begin{table}[H]
	\centering
	\resizebox{\textwidth}{!}{%
		\begin{threeparttable}
		\caption{Problems in \texttt{SIPLIB 2.0}}
		\label{table:problems}
			\begin{tabular}{@{}llll@{}}
				\toprule
				Problem		  		  & Description                                                        & Main reference              \\ \midrule
				\texttt{DCAP}         & Dynamic capacity planning with stochastic demand (\ref{DCAP})                   & Ahmed and Garcia \cite{journal:AG2004}                          \\
				\texttt{DCLP}		  &	Data center location problem									   & Kim et al. \cite{journal:KYZC2017}								\\	
				\texttt{MPTSPs}       & Multi-path traveling salesman problem with stochastic travel costs (\ref{MPTSPs})& Tadei et al. \cite{journal:TPP2017}                            \\
				\texttt{SIZES}        & Optimal product substitution with stochastic demand (\ref{SIZES})         & Jorjani et al. \cite{journal:JSW1999}          \\
				\texttt{SMKP}		  & Stochastic multiple knapsack problem (\ref{SMKP})                              & Angulo et al. \cite{journal:AAD2014}                            \\
				\texttt{SSLP}         & Stochastic server location problem (\ref{SSLP})                                & Ntaimo and Sen \cite{journal:NS2005}                           \\
				\texttt{SUC}         & Stochastic unit commitment problem	(\ref{SUC})			               & Papavasiliou and Oren \cite{journal:PO2013}                       \\ \bottomrule
			\end{tabular}%
			
%			\begin{tablenotes}
%				\small
%				\item For convenience, we skip the cardinality sign $|\cdot|$ for sets, i.e., for any set $S$, $S$ denotes the number of elements $|S|$ in this table.
%			\end{tablenotes}
		\end{threeparttable}
	}
\end{table}

\subsection{Instance naming rule}
Table \ref{table:naming_rule} shows how we name the instances. We change the original naming convention for consistency and future extension. Some legacy naming rules do not consider the case when the set cardinality becomes larger than 1 digit number. Moreover, since some \siplibtwo\ instances can be generated using more sets other than used in \siplib, we needed to define a new naming convention. For example, we change the instance names of \dcap\ and \smkp\ as below.
\begin{quote}
	\centering dcap$RNT$\_$\mathcal{S}$ $\longrightarrow$ DCAP$\_R\_N\_T\_\mathcal{S}$\\
	smkp$\_\mathcal{S}$ $\longrightarrow$ SMKP$\_I\_\mathcal{S}$
\end{quote}
For \dcap, we just add underbars ``\_'' to delimit set cardinalities. Without delimiter, the instance name causes confusion when set cardinality is greater than or equal to 10. For \smkp, we add new set cardinality $I$ since the fixed number $|I|=120$ can be changed by user if desired.

The capital Roman letters mean the sets defining the problems. In particular, the calligraphic letter $\mathcal{S}$ always denotes the scenario set. For notational convenience, we sometimes skip the cardinality sign $|\cdot|$ for sets, i.e., for set $S$, $S$ itself denotes the number of elements $|S|$ in Table \ref{table:naming_rule} and \ref{table:num_components}. Note that not all sets are used to define an instance. The sets that do not appear in the instance name are fixed by some pre-determined value by the original references so we follow them. For example, in \smkp\ there are four sets in total, $I,J,K,\mathcal{S}$, but the numbers of knapsacks $|J|$ and $|K|$ are fixed by 50 and 5 so do not appear in the instance name.
%\begin{table}[H]
%	\centering
%	\caption{Instance naming rules}
%	\label{table:naming_rule}
%	\resizebox{\textwidth}{!}{%
%		\begin{tabular}{@{}lll@{}}
%			\toprule
%			Problem & Instance name                 & Remark                                                                    					      \\ \midrule
%			\dcap\    & DCAP\_$R$\_$N$\_$T$\_$\mathcal{S}$    &   $R$: number of resources, $N$: number of tasks, $T$: number of time periods, $\mathcal{S}$: number of scenarios        \\
%			\dclp\	 &								&																										\\
%			\mptsps\  & MPTSPs\_$D$\_$N$\_$\mathcal{S}$ &$D$: node distribution strategy, $N$: number of nodes, $\mathcal{S}$: number of scenarios\\
%			\sizes\   & SIZES\_$\mathcal{S}$                            & $\mathcal{S}$: number of scenarios   															\\
%			\smkp\    &   SMKP\_$I$\_$\mathcal{S}$    &   $I$:number of types for item, $\mathcal{S}$: number of scenarios  													 \\
%			\sslp\    &  SSLP\_$I$\_$J$\_$\mathcal{S}$      &    $I$: number of clients, $J$: number of server locations, $\mathcal{S}$: number of scenarios                 				   \\
%			\suc\    & 	SUC\_$D$\_$\mathcal{S}$    &  $D$: day type, $\mathcal{S}$: number of scenarios                                                 						 \\ \bottomrule
%		\end{tabular}%
%	}
%\end{table}

\begin{table}[]
	\centering
	\caption{Instance naming rules}
	\label{table:naming_rule}
	\resizebox{\textwidth}{!}{%
		\begin{tabular}{@{}lll@{}}
			\toprule
			Problem & Instance name                 & Remark                                                                    					      \\ \midrule
			\dcap\    & DCAP\_$R$\_$N$\_$T$\_$\mathcal{S}$    &   $R$: number of resources, $N$: number of tasks, $T$: number of time periods, $\mathcal{S}$: number of scenarios        \\
			\mptsps\  & MPTSPs\_$D$\_$N$\_$\mathcal{S}$ &$D$: node distribution strategy, $N$: number of nodes, $\mathcal{S}$: number of scenarios\\
			\sizes\   & SIZES\_$\mathcal{S}$                            & $\mathcal{S}$: number of scenarios   															\\
			\smkp\    &   SMKP\_$I$\_$\mathcal{S}$    &   $I$:number of types for item, $\mathcal{S}$: number of scenarios  													 \\
			\sslp\    &  SSLP\_$I$\_$J$\_$\mathcal{S}$      &    $I$: number of clients, $J$: number of server locations, $\mathcal{S}$: number of scenarios                 				   \\
			\suc\    & 	SUC\_$D$\_$\mathcal{S}$    &  $D$: day type, $\mathcal{S}$: number of scenarios                                                 						 \\ \bottomrule
		\end{tabular}%
	}
\end{table}

\subsection{Type of the problems}
In \siplibtwo\, we mainly classify each problem by its stage-wise variable types. We consider three types of variable: continuous, binary, and integer. Considering two stages, the possible number of combination is $\left[\sum_{k=1}^3\binom{3}{k}\right]^2=49$ in total. We try to include problems with non-overlapping such combination. Table \ref{table:prob_class} shows the stage-wise components (variable and constraints) of each problem. For the abbreviated notation on the constraints, we refer \miplib\ \cite{MIPLIB}. Although the constraint type is possibly one of the important factors that define the problem characteristic, we decided not to consider it for classification since it can cause too much variety, which we cannot easily capture the insight from the problem type classification. 
\begin{table}[H]
	\centering
	\caption{Components of the problems}
	\label{table:prob_class}
	\begin{threeparttable}
		\begin{tabular}{@{}lllll@{}}
			\toprule
			& \multicolumn{2}{l}{1st stage}                              				  	& \multicolumn{2}{l}{2nd stage}                             			        \\ \midrule
			Problem 	     & Variable                    & Constraint                   	& Variable                    & Constraint                  				    \\ \midrule
			\dcap\ (\ref{dcap:formulation})    & $\mathbb{C}$, $\mathbb{B}$  & \texttt{VBB}                	& $\mathbb{B}$                & \texttt{PAR}, \texttt{M01} 			    		\\
			\dclp\  	 &							   &								& 			 	  &													\\				
			\mptsps\ (\ref{mptsps:formulation})  & $\mathbb{C}$, $\mathbb{B}$  & \texttt{PAR}, \texttt{GEN}		& $\mathbb{B}$                & \texttt{GEN}               						\\
			\sizes\ (\ref{sizes:formulation})   & $\mathbb{I}$ 			   & \texttt{VBD}, \texttt{GEN} 	& $\mathbb{B}$, $\mathbb{I}$  & \texttt{IKN}             						\\
			\smkp\ (\ref{smkp:formulation})   & $\mathbb{B}$                & \texttt{KNA}                	& $\mathbb{B}$                & \texttt{KNA}              						\\
			\sslp\ (\ref{sslp:formulation})   & $\mathbb{B}$                & \texttt{IVK}, \texttt{GEN} 	& $\mathbb{C}$, $\mathbb{B}$  & \texttt{GEN}             						\\
			\suc\ (\ref{SUC:formulation})   & $\mathbb{C}$, $\mathbb{B}$                 & \texttt{VBB}, \texttt{GEN}                            	& $\mathbb{C}$, $\mathbb{B}$  &  \texttt{VBB}, \texttt{GEN}                                  					\\ \bottomrule
		\end{tabular}
		
		\begin{tablenotes}
			\small
			\item *$\mathbb{C}$: continuous, $\mathbb{B}$: binary, $\mathbb{I}$: integer
			\item **Constraint type notation is adopted from \texttt{MIPLIB 2010}. Refer to \ref{sec:miplibconstraint}.
		\end{tablenotes}
	\end{threeparttable}
\end{table}

\subsection{Number of components}
Table \ref{table:num_components} summarizes the number of components (variables and constraints) in each problem from \siplibtwo. The numbers can be calculated based on the cardinality of the sets that define the problems. Although there is no universally effective way to measure the difficulty of MIP yet, the number of components in instance is one of the closely related factor. For example, instances tend to be more difficult as the number of discrete variables increases. For those who want to generate instances with some desired number of components can utilize Table \ref{table:num_components}.

\begin{table}[H]
	\centering
	\resizebox{\textwidth}{!}{%
		\begin{threeparttable}
			\caption{Number of components in each problem}
			\label{table:num_components}
			\begin{tabular}{@{}cccccc@{}}
				\toprule
				&               & \multicolumn{4}{c}{Components}                                                                         \\ \cmidrule(l){3-6} 
				&                & \#Continuous   & \#Binary                           & \#Integer            & \#Constraint              \\ \midrule
				\multirow{3}{*}{\dcap\ (\ref{dcap:notation})}   & 1st stage & $RT$           & $RT$                               & -                    & $RT$                      \\
				& 2nd stage & -              & $\left(1+R\right)NT$               & -                    & $(R+N)T$                  \\ \cmidrule(l){2-6} 
				& Total          & $RT$           & $RT+\left(1+R\right)NT\mathcal{S}$ & -                    & $RT+(R+N)T\mathcal{S}$    \\ \midrule
				\multirow{3}{*}{\mptsps (\ref{mptsps:notation})} & 1st stage & $(N-1)N$          & $(N-1)N$                              & -                    & $N^2+2N-1$                \\
				& 2nd stage & -              & $3(N-1)N$                            & -                    & $(N-1)N$                     \\ \cmidrule(l){2-6} 
				& Total          & $(N-1)N$          & $(N-1)(1+3\mathcal{S})N$              & -                    & $(1+\mathcal{S})N^2+(2-\mathcal{S})N-1$ \\ \midrule
				\multirow{3}{*}{\sizes\ (\ref{sizes:notation})}  & 1st stage & -              & $2N$                               & $2N$                 & $2(1+N)$                  \\
				& 2nd stage & -              & -                                  & $N(N+1)$              & $4N$                     \\ \cmidrule(l){2-6} 
				& Total          & -              & $2N$                               & $2N+N(N+1)\mathcal{S}$ & $2(1+N+2N\mathcal{S})$    \\ \midrule
				\multirow{3}{*}{\smkp\ (\ref{smkp:notation})}   & 1st stage & -              & $2I$                               & -                    & $J$                       \\
				& 2nd stage & -              & $I$                                & -                    & $K$                       \\ \cmidrule(l){2-6} 
				& Total          & -              & $(2+\mathcal{S})I$                 & -                    & $J+K\mathcal{S}$          \\ \midrule
				\multirow{3}{*}{\sslp\ (\ref{sslp:notation})}   & 1st stage & -              & $J$                                & -                    & $1$                       \\
				& 2nd stage & $J$            & $IJ$                               & -                    & $I+J$                     \\ \cmidrule(l){2-6} 
				& Total          & $J\mathcal{S}$ & $(1+I\mathcal{S})J$                & -                    & $1+(I+J)\mathcal{S}$      \\ \midrule
				\multirow{3}{*}{\suc\ (\ref{SUC:notation})}   & 1st stage & 960               &   1000                                 &     -                 &  2208                         \\
				& 2nd stage & 21274               &     2250                               &   -                   & 24780                          \\ \cmidrule(l){2-6} 
				& Total          & $960+21274\mathcal{S}$                &  $1000+2250\mathcal{S}$                                  &  -                    &  $2208+24780\mathcal{S}$                         \\ \bottomrule
			\end{tabular}
			
			\begin{tablenotes}
				\small
				\item *For convenience, we skip the cardinality sign $|\cdot|$ for sets.
				\item **We insert numerical value for the predetermined set, e.g., for \sizes, we use $|T|=2$ and for \mptsps, we use $|K_{ij}|=3$. In \suc, all the sets are predetermined based on the given data except for the scenario set $\mathcal{S}$.
			\end{tablenotes}
		\end{threeparttable}
	}
\end{table}

%
%\begin{table}[H]
%	\centering
%	\resizebox{\textwidth}{!}{%
%		\begin{threeparttable}
%			\caption{Number of components in each problem}
%			\label{table:num_components}
%			\begin{tabular}{@{}cccccc@{}}
%				\toprule
%				&               & \multicolumn{4}{c}{Components}                                                                         \\ \cmidrule(l){3-6} 
%				&                & \#Continuous   & \#Binary                           & \#Integer            & \#Constraint              \\ \midrule
%				\multirow{3}{*}{\dcap\ (\ref{dcap:notation})}   & 1st stage & $RT$           & $RT$                               & -                    & $RT$                      \\
%				& 2nd stage & -              & $\left(1+R\right)NT$               & -                    & $(R+N)T$                  \\ \cmidrule(l){2-6} 
%				& Total          & $RT$           & $RT+\left(1+R\right)NT\mathcal{S}$ & -                    & $RT+(R+N)T\mathcal{S}$    \\ \midrule
%				\multirow{3}{*}{\dclp\ }   & 1st stage &                &                                    &                      &                           \\
%				& 2nd stage &                &                                    &                      &                           \\ \cmidrule(l){2-6} 
%				& Total          &                &                                    &                      &                           \\ \midrule
%				\multirow{3}{*}{\mptsps (\ref{mptsps:notation})} & 1st stage & $(N-1)N$          & $(N-1)N$                              & -                    & $N^2+2N-1$                \\
%				& 2nd stage & -              & $3(N-1)N$                            & -                    & $(N-1)N$                     \\ \cmidrule(l){2-6} 
%				& Total          & $(N-1)N$          & $(N-1)(1+3\mathcal{S})N$              & -                    & $(1+\mathcal{S})N^2+(2-\mathcal{S})N-1$ \\ \midrule
%				\multirow{3}{*}{\sizes\ (\ref{sizes:notation})}  & 1st stage & -              & $2N$                               & $2N$                 & $2(1+N)$                  \\
%				& 2nd stage & -              & -                                  & $N(N+1)$              & $4N$                     \\ \cmidrule(l){2-6} 
%				& Total          & -              & $2N$                               & $2N+N(N+1)\mathcal{S}$ & $2(1+N+2N\mathcal{S})$    \\ \midrule
%				\multirow{3}{*}{\smkp\ (\ref{smkp:notation})}   & 1st stage & -              & $2I$                               & -                    & $J$                       \\
%				& 2nd stage & -              & $I$                                & -                    & $K$                       \\ \cmidrule(l){2-6} 
%				& Total          & -              & $(2+\mathcal{S})I$                 & -                    & $J+K\mathcal{S}$          \\ \midrule
%				\multirow{3}{*}{\sslp\ (\ref{sslp:notation})}   & 1st stage & -              & $J$                                & -                    & $1$                       \\
%				& 2nd stage & $J$            & $IJ$                               & -                    & $I+J$                     \\ \cmidrule(l){2-6} 
%				& Total          & $J\mathcal{S}$ & $(1+I\mathcal{S})J$                & -                    & $1+(I+J)\mathcal{S}$      \\ \midrule
%				\multirow{3}{*}{\suc\ (\ref{SUC:notation})}   & 1st stage & 960               &   1000                                 &     -                 &  2208                         \\
%				& 2nd stage & 21274               &     2250                               &   -                   & 24780                          \\ \cmidrule(l){2-6} 
%				& Total          & $960+21274\mathcal{S}$                &  $1000+2250\mathcal{S}$                                  &  -                    &  $2208+24780\mathcal{S}$                         \\ \bottomrule
%			\end{tabular}
%			
%			\begin{tablenotes}
%				\small
%				\item *For convenience, we skip the cardinality sign $|\cdot|$ for sets.
%				\item **We insert numerical value for the predetermined set, e.g., for \sizes, we use $|T|=2$ and for \mptsps, we use $|K_{ij}|=3$. In \suc, all the sets are predetermined based on the given data except for the scenario set $\mathcal{S}$.
%			\end{tablenotes}
%		\end{threeparttable}
%	}
%\end{table}


%\begin{table}[H]
%	\centering
%	\resizebox{\textwidth}{!}{%
%		\begin{threeparttable}
%			\caption{Number of components in each problem}
%			\label{table:num_components}
%			\begin{tabular}{ccccc}
%				\hline
%				&                & 1st stage  & 2nd stage & Total (including all scenarios)         \\ \hline
%				\multirow{4}{*}{\texttt{DCAP}}   & $\mathbb{C}$ & $RT$       & -         & $RT$                                    \\
%				& $\mathbb{B}$ & $RT$       & $NT(1+R)$ & $RT+NT(1+R)\mathcal{S}$                 \\
%				& $\mathbb{I}$ & -          & -         & -                                       \\
%				& constraint   & $RT$       & $T(R+N)$  & $RT+T(R+N)\mathcal{S}$                  \\ \hline
%				\multirow{4}{*}{\texttt{DCLP}}   & $\mathbb{C}$ &            &           &                                         \\
%				& $\mathbb{B}$ &            &           &                                         \\
%				& $\mathbb{I}$ &            &           &                                         \\
%				& constraint   &            &           &                                         \\ \hline
%				\multirow{4}{*}{\texttt{MPTSPs}} & $\mathbb{C}$ & $N(N-1)$   & -         & $N(N-1)$                                \\
%				& $\mathbb{B}$ & $N(N-1)$   & $NK(N-1)$ & $N(N-1)(1+K\mathcal{S})$                \\
%				& $\mathbb{I}$ & -          & -         & -                                       \\
%				& constraint   & $N^2+2N-1$ & $N(N-1)$  & $N^2(1-\mathcal{S})+N(2-\mathcal{S})-1$ \\ \hline
%				\multirow{4}{*}{\texttt{SIZES}}  & $\mathbb{C}$ & -          & -         & -                                       \\
%				& $\mathbb{B}$ & $2N$       & -         & $2N$                                    \\
%				& $\mathbb{I}$ & $2N$       & $2N^2$    & $2N(1+N\mathcal{S})$                    \\
%				& constraint   & $2(1+N)$   & $4N$      & $2(1+N+2N\mathcal{S})$                  \\ \hline
%				\multirow{4}{*}{\texttt{SMKP}}   & $\mathbb{C}$ & -          & -         & -                                       \\
%				& $\mathbb{B}$ & $2I$       & $I$       & $I(2+\mathcal{S})$                      \\
%				& $\mathbb{I}$ & -          & -         & -                                       \\
%				& constraint   & $J$        & $K$       & $J+K\mathcal{S}$                        \\ \hline
%				\multirow{4}{*}{\texttt{SSLP}}   & $\mathbb{C}$ & -          & $J$       & $J\mathcal{S}$                          \\
%				& $\mathbb{B}$ & $J$        & $IJ$      & $J(1+I\mathcal{S})$                     \\
%				& $\mathbb{I}$ & -          & -         & -                                       \\
%				& constraint   & 1          & $I+J$     & $1+\mathcal{S}(I+J)$                    \\ \hline
%				\multirow{4}{*}{\texttt{SUC}}    & $\mathbb{C}$ & $960$      & $21274$   & $960+21274\mathcal{S}$                  \\
%				& $\mathbb{B}$ & $1000$     & $2250$    & $1000+2250\mathcal{S}$                  \\
%				& $\mathbb{I}$ & -          & -         & -                                       \\
%				& constraint   & $2208$     & $24780$   & $2208+24780\mathcal{S}$                 \\ \hline
%			\end{tabular}
%			
%			\begin{tablenotes}
%				\small
%				\item note 1. For convenience, we skip the cardinality sign $|\cdot|$ for sets.
%				\item note 2. For the predetermined set, we insert numerical value, e.g., for \texttt{SIZES}, we use $|T|=2$ and for \texttt{MPTSPs}, we use $|K_{ij}|=3$.
%			\end{tablenotes}
%		\end{threeparttable}
%	}
%\end{table}

%% Please add the following required packages to your document preamble:
%% \usepackage{multirow}
%\begin{table}[]
%	\centering
%	\caption{My caption}
%	\label{my-label}
%	\begin{tabular}{ccccc}
%		\hline
%		&                & 1st stage  & 2nd stage & Total (including all scenarios)         \\ \hline
%		\multirow{4}{*}{\texttt{DCAP}}   & \#$\mathbb{C}$ & $RT$       & -         & $RT$                                    \\
%		& \#$\mathbb{B}$ & $RT$       & $NT(1+R)$ & $RT+NT(1+R)\mathcal{S}$                 \\
%		& \#$\mathbb{I}$ & -          & -         & -                                       \\
%		& \#constraint   & $RT$       & $T(R+N)$  & $RT+T(R+N)\mathcal{S}$                  \\ \hline
%		\multirow{4}{*}{\texttt{DCLP}}   & \#$\mathbb{C}$ &            &           &                                         \\
%		& \#$\mathbb{B}$ &            &           &                                         \\
%		& \#$\mathbb{I}$ &            &           &                                         \\
%		& \#constraint   &            &           &                                         \\ \hline
%		\multirow{4}{*}{\texttt{MPTSPs}} & \#$\mathbb{C}$ & $N(N-1)$   & -         & $N(N-1)$                                \\
%		& \#$\mathbb{B}$ & $N(N-1)$   & $NK(N-1)$ & $N(N-1)(1+K\mathcal{S})$                \\
%		& \#$\mathbb{I}$ & -          & -         & -                                       \\
%		& \#constraint   & $N^2+2N-1$ & $N(N-1)$  & $N^2(1-\mathcal{S})+N(2-\mathcal{S})-1$ \\ \hline
%		\multirow{4}{*}{\texttt{SIZES}}  & \#$\mathbb{C}$ & -          & -         & -                                       \\
%		& \#$\mathbb{B}$ & $2N$       & -         & $2N$                                    \\
%		& \#$\mathbb{I}$ & $2N$       & $2N^2$    & $2N(1+N\mathcal{S})$                    \\
%		& \#constraint   & $2(1+N)$   & $4N$      & $2(1+N+2N\mathcal{S})$                  \\ \hline
%		\multirow{4}{*}{\texttt{SMKP}}   & \#$\mathbb{C}$ & -          & -         & -                                       \\
%		& \#$\mathbb{B}$ & $2I$       & $I$       & $I(2+\mathcal{S})$                      \\
%		& \#$\mathbb{I}$ & -          & -         & -                                       \\
%		& \#constraint   & $J$        & $K$       & $J+K\mathcal{S}$                        \\ \hline
%		\multirow{4}{*}{\texttt{SSLP}}   & \#$\mathbb{C}$ & -          & $J$       & $J\mathcal{S}$                          \\
%		& \#$\mathbb{B}$ & $J$        & $IJ$      & $J(1+I\mathcal{S})$                     \\
%		& \#$\mathbb{I}$ & -          & -         & -                                       \\
%		& \#constraint   & 1          & $I+J$     & $1+\mathcal{S}(I+J)$                    \\ \hline
%		\multirow{4}{*}{\texttt{SUC}}    & \#$\mathbb{C}$ & $960$      & $21274$   & $960+21274\mathcal{S}$                  \\
%		& \#$\mathbb{B}$ & $1000$     & $2250$    & $1000+2250\mathcal{S}$                  \\
%		& \#$\mathbb{I}$ & -          & -         & -                                       \\
%		& \#constraint   & $2208$     & $24780$   & $2208+24780\mathcal{S}$                 \\ \hline
%	\end{tabular}
%\end{table}


\subsection{Sparsity}
In section \ref{sec:sparsity}, we append tables providing block-wise sparsity that is derived solely based on the set cardinality. Every SIP has a block-diagonal structure in its coefficient matrix of the extensive form (Fig \ref{fig:de_structure}). This characteristic differentiates SIP from the general MIP where the sparsity pattern varies instance by instance. In particular, the block-diagonal structure results always in very high sparsity (i.e., low density of non-zero values) as scenario increases. Unlike general MIP, it does not seem to be meaningful to just report the sparsity information of the extensive form coefficient matrix since decomposition-based algorithms in SIP can efficiently handle the sparsity. 
\kk{Would it be better to present the sparsity structure for one-scenario problem, like the one in .cor file?}
\yoc{I tried to show the repeatedly duplicating patterns by looking at once. I think it would be good to add a subsection in section \ref{sec:prob_desc} to discuss problem-specific sparsity pattern for each problem and present the one-scenario instance sparsity pattern. Thank you.}

\begin{figure}[H]
	\centering
	\subfloat[][\texttt{DCAP\_3\_3\_3\_3}]
	{
		\centering\includegraphics[width=0.45\linewidth]{DCAP_3_3_3_3}
		\label{fig:de_structure_dcap}
	}
	~
	\subfloat[][\texttt{MPTSPs\_D0\_3\_3}]
	{
		\centering\includegraphics[width=0.45\linewidth]{MPTSPs_D0_3_3}
		\label{fig:de_structure_mptsps}
	}
	
	\subfloat[][\texttt{SIZES\_3}]
	{
		\centering\includegraphics[width=0.45\linewidth]{SIZES_3}
		\label{fig:de_structure_sizes}
	}
	~
	\subfloat[][\texttt{SMKP\_120\_3}]
	{
		\centering\includegraphics[width=0.45\linewidth]{SMKP_120_3}
		\label{fig:de_structure_smkp}
	}
	
	\subfloat[][\texttt{SSLP\_5\_10\_3}]
	{
		\centering\includegraphics[width=0.45\linewidth]{SSLP_5_10_3}
		\label{fig:de_structure_sslp}
	}
	~
	\subfloat[][\texttt{SUCW\_FallWD\_1}]
	{
		\centering\includegraphics[width=0.45\linewidth]{SUCW_FallWD_1}
		\label{fig:de_structure_sucw}
	}

	\caption{Block-diagonal structure for each problem in extensive form}
	\begin{minipage}
		{0.65\textwidth}{\footnotesize (\texttt{SUCW} instance is too huge and extremely sparse to plot more than 1 scenario)}
	\end{minipage}
	\label{fig:de_structure}
\end{figure}

As can be seen in Fig \ref{fig:stagewise_sparsity}, there are three different blocks in terms of the structure. Every other block is the duplication of block T or W hence the same sparsity pattern repeats as many as the number of scenarios considered. Block A and W are only related with their own stage while block T is related with both. Block T is called \textit{technology matrix} . \kk{$T$ has been called ``technology matrix'' in the literature. Let's not introduce new term, unless we are proposing new concept related to the name.}
\yoc{Thanks. I change the term throughout the paper and pacakge.}

Dense coefficient matrix usually causes slowdown in decomposition algorithms. For example, the Dual Decomposition based solver \dsp\ shows much slower convergence speed than the centralized solver \cplex\ in problems like \smkp\ which always have low sparsity (nonzero ratio: 50\%-100\%, refer to Table \ref{table:sparsity_SMKP}).
\begin{figure}
	\centering
	\includegraphics[width=0.7\linewidth]{drawings/stagewise_sparsity}
	\caption{Three (structurally) independent blocks in SIP}
	\label{fig:stagewise_sparsity}
\end{figure}









