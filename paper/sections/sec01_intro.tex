%(What is SIPLIB?) 
The \texttt{SIPLIB} \cite{web:SIPLIB1} is an abbreviated term of the Stochastic Integer Programming (SIP) Library firstly contructed in 2002 by Shabbir Ahmed and his colleagues. The library has been providing a collection of test instances to facilitate computational and algorithmic research in SIP. Some new test problems with instances have been added to \texttt{SIPLIB} gradually and now it contains nine different problems in total.  The instances are basically given in the standard \texttt{SMPS} format accompanied with additional information including parameter data, size of the instance in terms of the number of rows, columns, and integers, benchmarking information such as best known objective value or bounds, optimality gap, and solution time.

%(Motivation) 
At the time \texttt{SIPLIB} appeared, it provided enoughly large-sized instances that is reasonable to argue that the performance of algorithm is remarkable if it solves the instances. State-of-the-art in SIP combined with the speedup in computing machinery, however, makes many instances in \texttt{SIPLIB} so easy that we have not enough basis to use them for showing the excellence of newly suggested solution methods. At this point, we are motivated to develop the second version of \texttt{SIPLIB}, say \texttt{SIPLIB 2.0} that provides larger-sized test instances with higher degree of tailorability, e.g., users can easily generate instances of test problems as largely as they want.% in terms of the number of scenarios included.

%(What SIP is and our restriction) 
Stochastic programming (SP) is a framework for modeling optimization problems that involve uncertainty. Whereas optimization problems are typically formulated with known parameters, the problems in real world contain some unknown parameters in many cases. For details on SP, see, e.g., \cite{web:SPS,book:BL2011}. SIP is a branch of SP that indicates any type of SP including at least one integer decision variable. We restrict our focus on two-stage SIP throughout this paper and \texttt{SIPLIB 2.0}. The main reason is that the class of SIP is most widely used to model real world problems. Moreover, two-stage SIP itself has enough difficulties that have not been conquered yet even without any other details like chance-constraints and multi-stages. The main difficulty in solving two-stage SIP is that the second-stage value function is not necessarily convex when there is any integer. Thus, the standard decomposition approaches that work nicely for stochastic \textit{linear} programs, break down when the second stage integer variables are present \cite{journal:AG2004}. Hereinafter, we use the term SIP to indicate the two-stage SIP.

%(SMPS, Julia)
We provide the test sets in two formats: \texttt{SMPS} files (*.cor, *.tim, *.stoch) and \texttt{Julia} files (*.jl). \texttt{SMPS} is widely used to describe stochastic linear and quadratic programs. Once having \texttt{SMPS} files of a problem instance, we can directly solve it using various mixed integer linear program (MILP) solvers such as \texttt{CPLEX}, \texttt{GUROBI}, and \texttt{CBC}. One can also use the existing open-source SIP solvers like \texttt{DSP} \cite{journal:KZ2015} and \texttt{SMI} \cite{web:SMI} given that \texttt{SMPS} files. A drawback of \texttt{SMPS} format is its low readability by human, which we decided to provide \texttt{Julia} files to let users be able to easily read problems and tailor the instances.

%(Convenience of Julia) 
\texttt{Julia} is an open source high-level, high-performance dynamic programming language for numerical computing. It is also known as its good performance, approaching that of statically-compiled languages like \texttt{C} \cite{journal:BEKS2017}. The syntax of \texttt{Julia} is simple and should feel familiar to anyone who has experienced in another high-level languages like \texttt{MATLAB} or \texttt{Python}. A \texttt{Julia} package called \texttt{JuMP} (Julia for Mathematical Programming \cite{web:JuMP}) provides a domain-specific modeling language for mathematical optimization embedded in \texttt{Julia}. \texttt{JuMP} enables us to easily translate mathematical model to \texttt{JuMP.Model}-type object. Some structured mathematical models like SIP can also be translated to the \texttt{JuMP.Model}-type object combined with the package \texttt{StructJuMP} \cite{web:StructJuMP}. Once we have a \texttt{Julia} code for constructing \texttt{JuMP.Model}-type object, it is easy to generate instances whenever we need to modify the original mathematical model. For each problem in \texttt{SIPLIB 2.0}, we provide a \texttt{Julia} script for constructing \texttt{JuMP.Model}-type object. We also provide a \texttt{Julia} script (SmpsWriter.jl) for converting any \texttt{JuMP.Model}-type objects to \texttt{SMPS} files for users' convenience. Those who feel the given instances are not large enough can simply generate more scenario data by just modifying the parameter in \texttt{Julia} script corresponding to the number of scenarios.

%(Contribution of SIPLIP2.0) 
By \texttt{SIPLIB 2.0}, we provide 1) richer collection of test instances for computational and algorithmic research in SIP with benchmarking computational results, 2) not only \texttt{SMPS} files but also utilities for generating/analyzing instances. Hence, the users can obtain as large-sized instances as they need by generating new scenarios and including them into instances. For those who want to utilize the instances in the legacy \texttt{SIPLIB} with strong tailorability provided by \texttt{SIPLIB 2.0}, we include the original \texttt{SIPLIB} instances as well.

%(Contents)
\textcolor{red}{(Contents of the paper, not finished yet) In this paper, we provide a detailed description of the test problems in \texttt{SIPLIB 2.0}, including information on the instances, their origin, formal mathematical models with explanation on notations, and scenario data generation procedure. \texttt{SIPLIB 2.0} classifies problems based on the stages, variables, and constraints. We borrow some classification criteria from the most frequently utilized mixed integer programming (MIP) library, \texttt{MIPLIB 2010} \cite{MIPLIB}, whenever they are applicable to SIP as well.}

\subsection{\texttt{SMPS} format}
\texttt{SMPS} format \cite{SMPS} is a data conventions for the automatic input of multiperiod stochastic linear programs. The input format is based on an old column-oriented format \texttt{MPSX} standard and is designed to promote the efficient conversion of originally deterministic problems by introducing stochastic variants in separate files. 

Three input files are required to specify an SP in \texttt{SMPS} format:
\begin{itemize}
	\item \texttt{.cor}: Core file written in \texttt{MPS} format. This describes the fundamental problem structure and contains the 1st stage data and one 2nd stage scenario data.
	\item \texttt{.tim}: Time file which specifies the location where the 2nd stage begins.
	\item \texttt{.sto}: Stoch file which contains stochastic data of all scenarios except the one included in \texttt{.cor} file.
\end{itemize}

\subsection{Literature review}
The optimizers, not only SIP researchers, have always needed test instances to figure out the performances of newly developed solver/algorithm. For SP, to the best of our knowledge, the first approach is Holmes and Birge's \textit{A portable stochastic programming test set} (POSTS) \cite{POSTS}. POSTS is still available from website and provides a small test set of stochastic programming recourse problems. POSTS is not dedicated only for SIP so 
%(website: Derek Holmes, POSTS, 1993)

%(website: Andy Felt, Jason Sarich, and K. A. Ariyawansa, SLPlib for Stochastic Linear Programming, 2001)

%(paper and website: Ariyawansa and Felt, INFORMS JOC, 2004)

%(website: SIPLIB 1)

%(paper and website: MIPLIB 2010)




