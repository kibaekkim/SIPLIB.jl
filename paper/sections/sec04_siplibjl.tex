\siplibtwo\ is implemented in \julia\ programming language with algebraic modeling packages \jump\ and \structjump. We implement and provide \julia\ package \siplibjl for users to utilize ingredients of \siplibtwo. In this section, we introduce how to use \siplibjl.

To use \siplibjl, we need to perform the following steps.
\begin{itemize}
	\item install \julia\ $\ge$ version 0.6.2
	\item install \julia\ packages: \texttt{Distributions.jl}, \texttt{JuMP.jl}, \texttt{StructJuMP.jl}, \texttt{PyPlot.jl}
	\item change working directory to ``\texttt{$\sim$/Siplib/src/}''
	\item run \julia\
	\item excute \texttt{include("Siplib.jl")}
	\item excute \texttt{using Siplib}
\end{itemize}

\begin{lstlisting}[frame=single,language=julia]
function getJuMPModel(problem::Symbol, param_arr::Any)::JuMP.Model

\end{lstlisting}

\subsection{Generating instances: \texttt{JuMP.Model}-object and \texttt{SMPS} files}
\jumpmodel-type object is an object that contains every information of an instance. Hence, almost every function in \siplibjl\ requires \jumpmodel-type object as one of its input arguments. \siplibjl\ provides two functions to construct the \jumpmodel-type object of an instance.
\begin{lstlisting}[frame=single,language=julia]
function getJuMPModel(problem::Symbol, param_arr::Any)::JuMP.Model
function generateSMPS(problem::Symbol, param_arr::Any, DIR_NAME::String="$(dirname(@__FILE__))/../instance")::JuMP.Model
\end{lstlisting}

The first function \texttt{getJuMPModel} takes \texttt{Symbol}-typed argument \texttt{problem} and associated parameter array \texttt{param\_arr}. Then, it constructs \jumpmodel-type object and return it. For example, the follwing command returns the \jumpmodel\ object of instance DCAP\_3\_4\_2\_100.
\begin{lstlisting}[frame=single,language=julia]
getJuMPModel(:DCAP, [3,4,2,100])
\end{lstlisting}
Keep in mind that the number of elements in \texttt{param\_arr} should match with the problem as in Table \ref{table:numparameter}, otherwise it prints a warning message.

\begin{table}[h]
	\centering
	\caption{\texttt{problem} arguments and corresponding parameter array}
	\label{table:numparameter}
	\resizebox{\textwidth}{!}{%
		\begin{tabular}{@{}ccc@{}}
			\toprule
			\texttt{problem}  & \texttt{param\_arr}     & Remark                                                                                                                                     \\ \midrule
			\texttt{:DCAP}               & \texttt{[R, T, N, $\mathcal{S}$]} & All parameters are integer.                                                                                                                \\
			\texttt{:MPTSPs}             & \texttt{[D, N, $\mathcal{S}$]}    & String $\texttt{D}\in \{\mathrm{``D0"}, \mathrm{``D1"}, \mathrm{``D2"}, \mathrm{``D3"}\}$                                                              \\
			\texttt{:SIZES}              & \texttt{[$\mathcal{S}$]}          & Integer $\mathcal{S}\ge 20$.                                                                                   \\
			\texttt{:SMKP}               & \texttt{[I, $\mathcal{S}$]}       & All parameters are integer.                                                                                                                \\
			\texttt{:SSLP}               & \texttt{[I, J, $\mathcal{S}$]}    & All parameters are integer.                                                                                                                \\
			\texttt{:SUC}                & \texttt{[D, $\mathcal{S}$]}       & \multicolumn{1}{l}{String $\texttt{D}\in \{\mathrm{``FallWD"}, \mathrm{``FallWE"}, \mathrm{``WinterWD"}, \mathrm{``WinterWE"}, $}                                             \\
			\multicolumn{1}{l}{}         & \multicolumn{1}{l}{}            & \multicolumn{1}{r}{$\mathrm{``SpringWD"}, \mathrm{``SpringWE"}, \mathrm{``SummerWD"}, \mathrm{``SummerWE"} \}$} \\ \bottomrule
		\end{tabular}
	}
\end{table}

The second function \texttt{generateSMPS} generates \smps\ files as well as returns \jumpmodel\ object by taking one more argument \texttt{DIR\_NAME} to indicate a directory where the files are stored. The \smps\ files are stored in the default folder ``\texttt{$\sim$/Siplib/instance/}'' unless the argument \texttt{DIR\_NAME} is specified. The file name is automatically generated using the arguments, e.g., \texttt{generateSMPS(:DCAP, [3, 4, 2, 100])} generates three files.
\begin{itemize}
	\item DCAP\_3\_4\_2\_100.cor
	\item DCAP\_3\_4\_2\_100.tim
	\item DCAP\_3\_4\_2\_100.sto
\end{itemize}
Sometimes one might want to generate \smps\ files using pre-declared \texttt{JuMP.Model} object. The function \texttt{writeSMPS} is defined to do such task.
\begin{lstlisting}[frame=single,language=julia]
function writeSMPS(model::JuMP.Model, INSTANCE::String="instance", DIR_NAME::String="$(dirname(@__FILE__))/../instance")
\end{lstlisting}
The function above takes \jumpmodel\ object as input argument and stores \smps\ files into \texttt{DIR\_NAME} folder with file name \texttt{INSTANCE}. The \texttt{String}-type arguments \texttt{INSTANCE} and \texttt{DIR\_NAME} can be omitted since they have default values ``\texttt{instance}" and ``\texttt{$\sim$/Siplib/instance/}."

We also define a conventional function to return the instance name in \texttt{String}-type.
\begin{lstlisting}[frame=single,language=julia]
function getInstanceName(problem::Symbol, param_arr::Any)::String
\end{lstlisting}

\subsection{Pre-analyzing instances: size, sparsity, plot}
\siplibjl\ provides pre-analysis functions for instances. By ``size'', we mean the number of components (continuous, binary, integer, constraint) in an instance. As we discussed in Section \ref{sec:sparsity}, sparsity is analyzed in block-wisely. The size and sparsity information is stored in the object of the following types: \texttt{Size} and \texttt{Sparsity}.

\siplibjl\ also provides functions to plot sparsity pattern in the coefficient matrix. The plots can be drawn in four ways. 
\begin{itemize}
	\item Coefficient matrix of extensive form
	\item First stage-only block (block A)
	\item Second stage-only block (block W)
	\item Complicating block (block T)
\end{itemize}

%\noindent\begin{minipage}{.45\textwidth}
%\begin{lstlisting}[frame=single,language=julia]
%type Size
%	InstanceName::String
%	nCont1::Int
%	nBin1::Int
%	nInt1::Int
%	nCont2::Int
%	nBin2::Int
%	nInt2::Int
%	nCont::Int
%	nBin::Int
%	nInt::Int
%	nRow::Int
%	nCol::Int
%	nNz::Int
%	Size() = new()
%end
%\end{lstlisting}
%\end{minipage}\hfill
%\begin{minipage}{.45\textwidth}
%\begin{lstlisting}[frame=single,language=julia]
%type Sparsity
%	InstanceName::String
%	nRow1::Int
%	nCol1::Int
%	nNz1::Int
%	sparsity1::Float64
%	nRow2::Int
%	nCol2::Int
%	nNz2::Int
%	sparsity2::Float64
%	nRowC::Int
%	nColC::Int
%	nNzC::Int
%	sparsityC::Float64
%	nRow::Int
%	nCol::Int
%	nNz::Int
%	sparsity::Float64
%	Sparsity() = new()
%end
%\end{lstlisting}
%\end{minipage}

\subsubsection{Get size information}
To get the size information of an instance, excute the following function.
\begin{lstlisting}[frame=single,language=julia]
function getSize(model::JuMP.Model, InstanceName::String="")::Size
\end{lstlisting}
The function \texttt{getSize} takes \jumpmodel\ as an input argument and returns \texttt{Size}-type object defined as follows.
\begin{lstlisting}[frame=single,language=julia]
type Size
	InstanceName::String    # instance name
	nCont1::Int             # number of continuous variables in 1st stage
	nBin1::Int              # number of binary variables in 1st stage
	nInt1::Int              # number of integer variables in 1st stage
	nCont2::Int             # number of continuous variables in 2nd stage    
	nBin2::Int              # number of binary variables in 2nd stage
	nInt2::Int              # number of integer variables in 2nd stage    
	nCont::Int              # number of continuous variables in total      
	nBin::Int               # number of binary variables in total      
	nInt::Int               # number of integer variables in total      
	nRow::Int               # number of rows in coefficient matrix in extensive form
	nCol::Int               # number of columns in coefficient matrix in extensive form
	nNz::Int                # number of nonzero values in coefficient matrix in extensive form
	Size() = new()
end
\end{lstlisting}

\subsubsection{Get sparsity information}
To get the sparsity information of an instance, excute the following function.
\begin{lstlisting}[frame=single,language=julia]
function getSparsity(model::JuMP.Model, InstanceName::String="")::Sparsity
\end{lstlisting}
The function \texttt{getSparsity} takes \jumpmodel\ as an input argument and returns \texttt{Sparsity}-type object.
\begin{lstlisting}[frame=single,language=julia]
type Sparsity
	InstanceName::String    # instance name
	nRow1::Int              # number of rows in 1st stage-only block (block A)
	nCol1::Int              # number of columns in 1st stage-only block (block A)
	nNz1::Int               # number of nonzero values in 1st stage-only block (block A)
	sparsity1::Float64      # sparsity ([0,1] scale) of 1st stage-only block (block A)
	nRow2::Int              # number of rows in 2nd stage-only block (block W)
	nCol2::Int              # number of columns in 2nd stage-only block (block W)
	nNz2::Int               # number of nonzero values in 2nd stage-only block (block W)
	sparsity2::Float64      # sparsity ([0,1] scale) of 2nd stage-only block (block W)
	nRowC::Int              # number of rows in complicating block (block T)
	nColC::Int              # number of columns in complicating block (block T)
	nNzC::Int               # number of nonzero values in complicating block (block T)  
	sparsityC::Float64      # sparsity ([0,1] scale) of complicating block (block T)
	nRow::Int               # number of rows in total
	nCol::Int               # number of columns in total
	nNz::Int                # number of nonzero values in total
	sparsity::Float64       # sparsity ([0,1] scale) in total
	Sparsity() = new()
end
\end{lstlisting}
\subsubsection{Plot sparsity patterns}
To plot the sparsity patterns of coefficient matrices, we provide the following functions.
\begin{lstlisting}[frame=single,language=julia]
function plotConstrMatrix(model::JuMP.Model, INSTANCE::String="instance", DIR_NAME::String="$(dirname(@__FILE__))/../plot")

function plotFirstStageBlock(model::JuMP.Model, INSTANCE::String="instance_block_A", DIR_NAME::String="$(dirname(@__FILE__))/../plot")

function plotSecondStageBlock(model::JuMP.Model, INSTANCE::String="instance_block_W", DIR_NAME::String="$(dirname(@__FILE__))/../plot")

function plotComplicatingBlock(model::JuMP.Model, INSTANCE::String="instance_block_T", DIR_NAME::String="$(dirname(@__FILE__))/../plot")

function plotAllBlocks(model::JuMP.Model, INSTANCE::String="instance", DIR_NAME::String="$(dirname(@__FILE__))/../plot")

function plotAll(model::JuMP.Model, INSTANCE::String="instance", DIR_NAME::String="$(dirname(@__FILE__))/../plot")
\end{lstlisting}
The function \texttt{plotConstrMatrix} takes \jumpmodel-type object and plots the constraint matrix of extensive form. For example, the following command lines plot Fig. \ref{fig:plotall_b}.
\begin{lstlisting}[frame=single,language=julia]
param_arr = [2,2,2,2]	                        # declare parameters
problem = :DCAP	                                # declare problem
INSTANCE = getInstanceName(problem, param_arr)	# save instance name
model = getJuMPModel(problem, param_arr)	    # construct JuMP.Model object
plotConstrMatrix(model, INSTANCE)               # plot extensive form constraint matrix   
\end{lstlisting}

The functions \texttt{plotFirstStageBlock}, \texttt{plotSecondStageBlock}, and \texttt{plotComplicatingBlock} take \jumpmodel-type object and plots each block. For example, the following command lines plot Fig. \ref{fig:plotall_a}, \ref{fig:plotall_c}, and \ref{fig:plotall_d}.
\begin{lstlisting}[frame=single,language=julia]
param_arr = [2,2,2,2]	                        # declare parameters
problem = :DCAP	                                # declare problem
INSTANCE = getInstanceName(problem, param_arr)	# save instance name
model = getJuMPModel(problem, param_arr)	    # construct JuMP.Model object
plotFirstStageBlock(model, INSTANCE)               # plot 1st stage block
plotSecondStageBlock(model, INSTANCE)               # plot 2nd stage block
plotComplicatingBlock(model, INSTANCE)               # plot complicating block
\end{lstlisting}

One might want to draw all the plots at once. The following two functions are defined to do that.
\begin{lstlisting}[frame=single,language=julia]
plotAllBlocks(model, INSTANCE)               # plot all blocks A,W,T
plotAll(model, INSTANCE)               # plot all the plots above
\end{lstlisting}
\begin{figure}[H]
	\centering
	\subfloat[][DCAP\_2\_2\_2\_2\_block\_A.pdf]
	{
		\centering\includegraphics[width=0.45\linewidth]{DCAP_2_2_2_2_block_A}
		\label{fig:plotall_a}
	}
	~
	\subfloat[][DCAP\_2\_2\_2\_2.pdf]
	{
		\centering\includegraphics[width=0.45\linewidth]{DCAP_2_2_2_2}
		\label{fig:plotall_b}
	}
	
	\subfloat[][DCAP\_2\_2\_2\_2\_block\_T.pdf]
	{
		\centering\includegraphics[width=0.45\linewidth]{DCAP_2_2_2_2_block_T}
		\label{fig:plotall_c}
	}
	~
	\subfloat[][DCAP\_2\_2\_2\_2\_block\_W.pdf]
	{
		\centering\includegraphics[width=0.45\linewidth]{DCAP_2_2_2_2_block_W}
		\label{fig:plotall_d}
	}

	\caption{Plots drawn by executing function \texttt{plotAll}}
%	\begin{minipage}
%	\end{minipage}
	\label{fig:plotall}
\end{figure}
By executing \texttt{plotAll}, one can obtain all the plots in Fig. \ref{fig:plotall}.
%
%\subsection{Solving instances: interfacing with \texttt{DSP} solver}
%\subsubsection{Extensive form: Invoking standard MIP solver}
%
%\subsubsection{Dual decomposition}
%
%\subsubsection{Benders decomposition}