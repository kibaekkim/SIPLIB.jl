%\yoc{This section is newely written.}
Handling uncertainty is always complicating but unavoidable consideration in real world problems. SIP is one of the useful modeling frameworks that has been effectively addressing some of the issues. Also, unceasing efforts of researchers in this field is making great progress in terms of the range of problems that are conquered. 

Still, we believe that there are many remaining rooms to be pioneered. SIP with realized uncertain data (i.e., extensive form) can be viewed as a special case of MIP having special structure in it. Reversely, SIP before uncertainty realization can be considered to include general MIP since $\textrm{SIP}\equiv\textrm{MIP}\cup\textrm{Uncertainty}$. Whichever the viewpoint, they share many characteristics in common. No one can come up with the universally effective quantification of complexity both in MIP and SIP as of now, however, most researchers would agree with some factors are definitely related, e.g., type of variables, size (rows and columns) of instance, and tightness of the formulation. Some of them cannot be easily measured whereas some can be directly obtained in explicit form. We investigated each of them sometimes analytically and sometimes computationally.

In this study, we also try to provide pivotal platform as well as a milestone for further improvement in SIP by discussing the issues present in SIP and implementing convenient functionalities. Using relatively new language \julia\ hence presumably be unfamiliar, but which is most convenient and strong to the best of our knowledge, is a part of that spirit. 

Although the endemic broad viewpoint of this study restricts deeper investigation on every single different problem, we suggest readers to dig into any one of the topics or problems more intensively. Any further contribution or suggestion for \siplibtwo\ is always welcomed. Better solution reports, more useful functionanlities for the package, interesting problems with \julia\ scripts, more effective analysis methods, and so on.
