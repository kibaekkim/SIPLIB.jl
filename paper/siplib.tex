%%%%%%%%%%%%%%%%%%%%%%% file template.tex %%%%%%%%%%%%%%%%%%%%%%%%%
%
% This is a general template file for the LaTeX package SVJour3
% for Springer journals.          Springer Heidelberg 2010/09/16
%
% Copy it to a new file with a new name and use it as the basis
% for your article. Delete % signs as needed.
%
% This template includes a few options for different layouts and
% content for various journals. Please consult a previous issue of
% your journal as needed.
%
%%%%%%%%%%%%%%%%%%%%%%%%%%%%%%%%%%%%%%%%%%%%%%%%%%%%%%%%%%%%%%%%%%%
%
% First comes an example EPS file -- just ignore it and
% proceed on the \documentclass line
% your LaTeX will extract the file if required
\begin{filecontents*}{example.eps}
%!PS-Adobe-3.0 EPSF-3.0
%%BoundingBox: 19 19 221 221
%%CreationDate: Mon Sep 29 1997
%%Creator: programmed by hand (JK)
%%EndComments
gsave
newpath
  20 20 moveto
  20 220 lineto
  220 220 lineto
  220 20 lineto
closepath
2 setlinewidth
gsave
  .4 setgray fill
grestore
stroke
grestore
\end{filecontents*}
%
\RequirePackage{fix-cm}
%
%\documentclass{svjour3}                     % onecolumn (standard format)
%\documentclass[smallcondensed]{svjour3}     % onecolumn (ditto)
\documentclass[smallextended]{svjour3}       % onecolumn (second format)
%\documentclass[twocolumn]{svjour3}          % twocolumn
%
\smartqed  % flush right qed marks, e.g. at end of proof
%
\usepackage{graphicx}
\usepackage{hyperref}
\usepackage{float}
\usepackage{mathrsfs,amsmath,amssymb,amscd,mathtools}
\usepackage{booktabs}
\usepackage{graphicx}
\usepackage{lscape}
\usepackage{color}
\usepackage[normalem]{ulem}

\newcommand{\stkout}[1]{\ifmmode\text{\sout{\ensuremath{#1}}}\else\sout{#1}\fi}
\DeclareMathOperator*{\PP}{\mathbb{P}}
\DeclareMathOperator*{\EE}{\mathbb{E}}
\DeclarePairedDelimiter\ceil{\lceil}{\rceil}
\DeclarePairedDelimiter\floor{\lfloor}{\rfloor}

\raggedbottom
% \usepackage{mathptmx}      % use Times fonts if available on your TeX system
%
% insert here the call for the packages your document requires
%\usepackage{latexsym}
% etc.
%
% please place your own definitions here and don't use \def but
% \newcommand{}{}
%
% Insert the name of "your journal" with
% \journalname{myjournal}
%
\begin{document}

\title{SIPLIB 2.0
%\thanks{Grants or other notes
%about the article that should go on the front page should be
%placed here. General acknowledgments should be placed at the end of the article.}
}
\subtitle{Stochastic Integer Programming Library version 2.0}

\titlerunning{SIPLIB 2.0}        % if too long for running head

\author{Yongkyu Cho \and 
		Kibaek Kim \and
        Cong Han Lim \and
        James Luedtke \and
        Jeffrey Linderoth
}

%\authorrunning{Short form of author list} % if too long for running head

\institute{Yongkyu Cho, Kibaek Kim \at
           Mathematics and Computer Science Division, Argonne National Laboratory, Lemont, IL 60439, USA \\
           \email{choy@anl.gov; kimk@anl.gov}
           \and
           Cong Han Lim \and James Luedtke \and Jeffrey Linderoth \at
           Department of Industrial and Systems Engineering, University of Wisconsin-Madison Madison, WI 53706, USA \\
%          \and
%           Cong Han Lim \at
           \email{clim9@wisc.edu; jim.luedtke@wisc.edu; linderoth@wisc.edu}
%           \and 
%           James Luedtke \at
%           \email{jim.luedtke@wisc.edu}
%           \and 
%           Jeffrey Linderoth \at
%           \email{linderoth@wisc.edu}
}

\date{Received: date / Accepted: date}

\maketitle

\begin{abstract}
We present a collection of stochastic integer programming problem instances.
\keywords{Stochastic Integer Programming \and Problem Instances}
\end{abstract}

\section{Introduction}\label{sec:intro}
\yoc{I changed the structure and contents of Introduction section so did the flow.}
%(What SIP is and our restriction) 
Stochastic programming (SP) is a framework for modeling optimization problems that involve uncertainty. Whereas optimization problems are typically formulated with known parameters, the problems in real world contain some unknown parameters in many cases. For details on SP, see, e.g., \cite{web:SPS,book:BL2011}. SIP is a branch of SP that indicates any type of SP including at least one integer decision variable. We restrict our focus on two-stage SIP with linear objective throughout this paper. The main reason is that the class of SIP is most widely used to model real world problems. Moreover, two-stage SIP itself has enough difficulties that have not been conquered yet even without any other details like chance-constraints and multi-stages. The main difficulty in solving two-stage SIP is that the second-stage value function is not necessarily convex. Thus, the standard decomposition approaches that work nicely for stochastic \textit{linear} programs, break down when the second stage integer variables are present \cite{journal:AG2004}. Hereinafter, we use the term SIP to indicate the two-stage SIP. 
\kk{This is good. We should elaborate this paragraph more with references and move it to the very beginning of this section.} 
\yoc{Thanks. I moved this paragraph here.}


%(What is SIPLIB?) 
%\siplib\ \cite{web:SIPLIB1} is an abbreviated term of the Stochastic Integer Programming (SIP) Library firstly contructed in 2002 by Shabbir Ahmed and his colleagues. The library has been providing a collection of test instances to facilitate computational and algorithmic research in SIP. Some new test problems with instances have been added to \siplib\ gradually and now it contains nine different problems in total. %The instances are basically given in the standard \smps\ format accompanied with additional information including parameter data, size of the instance in terms of the number of rows, columns, and integers, benchmarking information such as best known objective value or bounds, optimality gap, and solution time. 

%\kk{We may want to start with the description of stochastic (integer) programming, followed by the brief literature review in modeling and algorithms and the computational challenges.}

%\yoc{I reflect this comment by moving and modifying the paragraph.}


%\subsection{Literature review} \label{subsec:literaturereview}
The optimizers, not only SIP researchers, have always needed test instances to figure out the performances of newly developed methods. \miplib\ \cite{MIPLIB} for mixed integer programming (MIP) is a good example of such collection hence the main motivation of this study. \miplib\ comprises 361 instances. The authors collected the instances from various sources and categorized them into 8 groups after the computational experiments. \miplib\ provides instances in \textsf{MPS} format with a test engine developed to run different solvers in a defined way to check the answers for consistency. 

%(website: Derek Holmes, POSTS, 1993)
For stochastic programming, to the best of our knowledge, the first such approach is Holmes and Birge's \textit{portable stochastic programming test set} (POSTS) \cite{POSTS} since 1994. POSTS is still available now and provides a small test set of stochastic programming recourse problems in \smps\ format. POSTS consists of 15 stochastic linear programming problems but not dedicated to SIP. Birge provides some computational results of the POSTS instances \cite{POSTSresults}, but most of them needs to be updated. Moreover, analysis on the instances does not seem to be enough. 

%(website and paper: Andy Felt, Jason Sarich, and K. A. Ariyawansa, Test-Problem Collection for Stochastic Linear Programming, 2001-2004)
Ariyawansa and Felt \cite{AF2004} have constructed a test problem collection for stochastic linear programming since 2001 and published a paper on it. Unlike Holmes and Birge's work, the authors provide an accompanied document explaining short description, mathematical problem statement, and notational reconciliation to a standard problem format for each of the 9 problems. Despite its name, Ariyawansa and Felt's collection also includes 3 problems that contain mixed integer variables. However, still the library is not dedicated to SIP and size of the instances are not large enough to perform intensive computational experiments. 

%(website: SIPLIB 1)
The first SIP-oriented instance collection is the \siplib\ \cite{web:SIPLIB1} constructed in 2002 by Shabbir Ahmed and his colleagues. The instances are basically given in \smps\ format accompanied with simple information on the problem and computational experiment. Although \siplib\ provides basic ingredient to be exploited for SIP research, it has rooms for improvement. First, \siplib\ only provides static instance files and sometimes does not even provide ready-made instance files, which limits usability of the library. Moreover, the precise information for implementing the original problem is often not allowed. Second, we need more problems of various types. Considering three types of variables (continuous, binary, and integer) and two stages, the possible number of combination is $\left[\sum_{k=1}^3\binom{3}{k}\right]^2=49$ in total while \siplib\ provides only 5 such combinations regarding the problems that can be fully implementable based on the open information.
Third, \siplib\ needs to be polished systematically in terms of both pre-analysis on the instances and post-analysis on the benchmark results. Currently, there is no predefined contribution rule for \siplib\ so different problem provides different information. Therefore, interpretation on the new results can be inconsistent highly depending on researchers.

\kk{This may not deserve a subsection. It can be in a part of Section 1. We need the literature review for stochastic programming.} \yoc{Thanks, I move this here. I think the literature review on SP itself is not well blended in Introduction. Instead, Section 2: Stochastic Integer Programming is written to briefly review the SIP. What do you think about it?}

%(Motivation) 
%At the time \siplib\ appeared, it provided enoughly large-sized instances that is reasonable to argue that the performance of algorithm is remarkable if it handles the instances well. State-of-the-art in SIP combined with the speedup in computing machinery, however, makes many instances in \siplib\ trivial so that we have not enough basis to use them for showing the excellence of newly suggested solution methods. At this point, we are motivated to develop the second version of \siplib\, say \siplibtwo\ that provides larger-sized test instances with higher degree of tailorability, e.g., users can easily generate instances of test problems as largely as they want.% in terms of the number of scenarios included.

%At the time \siplib\ appeared, it provided enough instances to be used for computational experiments of SIP methods. State-of-the-art in SIP combined with the speedup in computing machinery, however, requires more test sets as the existing instances have been conquered gradually. As \siplib\ only provides limited testing resources (e.g., static instance files), it has less expandability. 
In summary, we conclude that the current \siplib\ needs to be extended since state-of-the-art in SIP combined with the speedup in computing machinery requires more test sets as the existing instances have been conquered gradually. At this point, we are motivated to develop the second version of \siplib, say \siplibtwo\ that includes richer testing resources. By ``richer", we mean the three things. First, the larger-sized instances of the existing \siplib\ problems. Second, instances of the different problems other than existing ones. Third, higher degree of freedom for the researchers, e.g., users can easily generate instances of the shape/size as they want.
\kk{The justification would be to provide more test instances (not because the existing library has easy instances but because it provides limited testing resources (i.e., smps files only). The functionality we provide to generate large instances is important; but large instances themselves are not necessarily interesting. We should not focus on large vs. small, hard vs. easy, etc.}
\yoc{Thanks. I modify this paragraph to describe our philosophy. And the term ``richer" is firstly introduced and explained here.} 

%(SMPS, Julia)
We provide \siplibtwo\ in two ways: \smps\ files (\texttt{*.cor}, \texttt{*.tim}, \texttt{*.stoch}) and open-source \julia\ package. \smps\ is a file format widely used to describe stochastic program instances. Once \smps\ files of a problem instance are given, we can directly solve it using SIP solvers like \dsp\ \cite{journal:KZ2015} and \textsf{SMI} \cite{web:SMI}. A drawback of \smps\ is low readability by human, which we decided to provide \julia\ scripts to let users be able to easily catch up the problems and tailor the instances if needed.

%(Convenience of Julia) 
\julia\ is an open source high-level, high-performance dynamic programming language for numerical computing. It is also known as nice performance, approaching that of statically-compiled languages like \clang\ \cite{journal:BEKS2017}. The syntax of \julia\ is simple and should feel familiar to anyone who has experienced in another high-level languages like \matlab\ or \python. A \julia\ package called \jump\ (\julia\ for Mathematical Programming \cite{web:JuMP}) provides a domain-specific modeling language for mathematical optimization embedded in \julia. \jump\ enables us to easily translate a paper-written mathematical models to a \texttt{JuMP.Model}-type object. Some structured mathematical models like SIP can also be translated to the \texttt{JuMP.Model}-type object by loading a structured modeling package \structjump\ \cite{web:StructJuMP}. Once we have a \julia\ script for constructing \texttt{JuMP.Model}-type object, it is easy to modify the original model. %For each problem in \siplibtwo, we provide a \julia\ script for constructing \texttt{JuMP.Model}-type object. 
We implement a \julia\ package to provide various functions for handling \siplibtwo\ instances for users' convenience. Those who feel the given instances are not enough can simply generate new instances using the package.


%(Contribution of SIPLIP2.0) 
The contributions of this work can be summarized as follows.
\begin{itemize}
	\item We provide SIP-dedicated testing resources that are richer than the former \siplib.
	\item We collect, implement, summarize, and open all the problem-specific details: mathematical formulation, stochastic data generation, \julia\ scripts.
	\item We provide analysis on the problems and instances (component type, size, sparsity).
	\item We implement an open source \julia\ package for investigating instances: generation, analysis, and test.
	\item We provide well-summarized benchmark computational experiments results.
\end{itemize}
\kk{This is a good summary. Before the summary, these should be properly discussed in this section. ``richer'' is vague. Does ``benchmark'' mean solution reports?}
\yoc{Thanks. For the term ``richer", I add the meaning in the 6th paragraph. For the computational part, I will think about more proper expression.}

%By \texttt{SIPLIB 2.0}, we provide 1) richer collection of test instances for computational and algorithmic research in SIP with benchmarking computational results, 2) not only \texttt{SMPS} files but also utilities for generating/analyzing instances. Hence, the users can obtain as large-sized instances as they need by generating new scenarios and including them into instances. For those who want to utilize the instances in the legacy \texttt{SIPLIB} with strong tailorability provided by \texttt{SIPLIB 2.0}, we include the original \texttt{SIPLIB} instances as well.

%(Contents)
This paper is organized as follows. In Section \ref{sec:sip}, we briefly review SIP. This includes the mathematical formulation with notation, solution methods, and available software packages. In Section \ref{sec:summary}, we provide summary of the test problems. This consists of origin of the problems with brief description, instance naming rule, problem type based on its components (variables and constraints), the number of the components, and sparsity information. Together with the problem-specific description given in Section\ref{sec:prob_desc}, we believe that users can quickly catch up what they need without investigating every detail. We introduce our implementation of a \julia\ package for \siplibtwo\ in Section \ref{sec:siplibjl} with a brief tutorial in Section \ref{sec:tutorial}. In Section \ref{sec:instance_catalog}, we report the computational results of the accompanied \smps\ instances. We conclude this study with provision of future directions to improve this ongoing project in Section \ref{sec:conclusion}. 








\pagebreak

\section{Stochastic integer programming}\label{sec:sip}
In this section, we review the formulation, solution approaches, and software packages for SIP.
\subsection{Formulation}
%We introduce the form of SIP of interest. The notations and dimensional information are summarized in Table \ref{notation:SIP}. 
We present two-stage SIP that is formulated as follows:
\begin{align}
\min_{x\in X}{\left\{c^\top x + \mathcal{Q}(x):\ Ax\ge b\right\}}, \label{eq:rp}
\end{align}
where the first-stage problem data $X\subseteq \mathbb{R}^{n_1-k_1}\times\mathbb{Z}^{k_1}$, $c\in\mathbb{R}^{n_1}$, $A\in\mathbb{R}^{m_1\times n_1}$, and $b\in\mathbb{R}^{m_1}$ are given, and $\mathcal{Q}(x):=\EE_{\pmb{\xi}}[Q(x,\pmb{\xi})]$ is the expected recourse function associated with the random variable $\pmb{\xi}$. \kk{Please use $Q(x,\bxi)$ throughout the paper.} We assume that $\bxi$ is defined on a measurable space with the probability density measure $\PP$. Moreover, to the best interest of this paper, we assume that $\pmb{\xi}$ follows a known probability distribution with the finite countable outcomes $\xi_1,\ldots,\xi_r$, called \textit{scenarios}, and respective probabilities $\PP(1),\ldots,\PP(r)$ (i.e., $\PP(s):=\PP[\pmb{\xi}=\xi_s]$ for each $s\in\mathcal{S}:=\{1,\ldots,r\}$). When the distribution is known to be continuous, we also assume that we can reasonably approximate it by a suitably discretized distribution. 
The recourse function for a given random realization $\xi_s$ is defined by 
\begin{align}
Q(x;\xi_s):=\min_{y\in Y}\left\{q(\xi_s)^\top y : W(\xi_s)y\ge h(\xi_s)-T(\xi_s)x\right\}, \label{eq:rf}
\end{align}
where $Y\subseteq\mathbb{R}^{n_2-k_2}\times\mathbb{Z}^{k_2}$ and $q(\xi_s)\in\mathbb{R}^{n_2}$, $W(\xi_s)\in\mathbb{R}^{m_2\times n_2}$, $h(\xi_s)\in\mathbb{R}^{m_2}$, and $T(\xi_s)\in\mathbb{R}^{m_2\times n_1}$ are scenario-dependent parameters in the second stage. Hereinafter, we use the simplified notation $(T_s,W_s,h_s,q_s)$ for the second-stage data. 
%We summarize the dimensional information in Table \ref{notation:SIP}. 
\kk{Defining $X=\mathbb{R}^{n_1-k_1}\times\mathbb{Z}^{k_1}$ must be sufficient for this work, so is for $Y$.}

With the finite number of scenarios $\calS$, SIP (\ref{eq:rp}) can be formulated to the following \textit{deterministic equivalent form (DEF)} \kk{Changed EF to DEF, please apply this throughout the manuscript.}
\begin{subequations}\label{sip:ef}
\begin{align}
\min_{x,y_s}\ &c^{\top}x + \sum_{s=1}^{r}\PP(s) (q_s^{\top}y_s), \label{ef:obj}\\ 
\mathrm{s.t.}\ &Ax\ge b,  \label{ef:b}\\
	&T_s x+W_s y_s\ge h_s,\quad\forall s\in\{1,\ldots,r\}, \label{ef:c} \\
	&x\in X, \label{ef:d} \\
	&y_s \in Y,\quad\forall s\in\{1,\ldots,r\}. \label{ef:e}
\end{align}
\end{subequations}
DEF is particularly important in the context of this paper, because the problem instances are modeled and written to this form in a structured way. \kk{Table may not be compatible to the journal format. Please check and correct it, if needed. The summary in the table is redundant and thus may be unnecessary; made it commented out.}

%\begin{table}[]
%	\resizebox{\textwidth}{!}
%	{
%		\begin{threeparttable}
%			\caption{Summary of notations in SIP formulation}
%			\label{notation:SIP}
%			\begin{tabular}{ll}
%				\toprule
%				\multicolumn{2}{l}{\textbf{Sets:}} \\ 
%				$X\subseteq\mathbb{R}^{n_1-k_1}\times\mathbb{Z}^{k_1}$	& first-stage polyhedral set (continuous, integer, binary)\\
%				$Y\subseteq\mathbb{R}^{n_2-k_2}\times\mathbb{Z}^{k_2}$	& second-stage polyhedral set (continuous, integer, binary)\\ 
%				$\mathcal{S}=\{1,\ldots,r\}$	& index set of realizable scenarios \\ \midrule
%				%$G_j$	& scenario feasibility set\\ \midrule
%				\multicolumn{2}{l}{\textbf{Scalas:}} \\ 
%	%			$z\in\mathbb{R}$ & optimal objective value of the SIP \\ 
%				$r\in\mathbb{N}$	& total number of scenarios	\\	
%				$s\in\mathcal{S}$	& scenario index 	\\
%				$\PP(s)\in[0,1]$ & probability that scenario $s$ occurs, i.e., $\PP(s):=\PP[\pmb{\xi}=\xi_s]$ \\ \midrule
%				\multicolumn{2}{l}{\textbf{Vectors:}} \\  
%				$x\in\mathbb{R}^{n_1}$	& first-stage decision vector	\\
%				$c\in \mathbb{R}^{n_1}$	& first-stage cost vector\\
%				$b\in\mathbb{R}^{m_1}$	& first-stage RHS vector\\
%				$y_s\in\mathbb{R}^{n_2}$	& second-stage decision vector under scenario $\xi_s$	\\
%				$q_s:= q(\xi_s)\in\mathbb{R}^{n_2}$	& second-stage cost vector \\
%				$h_s:= h(\xi_s)\in\mathbb{R}^{m_2}$	& second-stage RHS vector\\ \midrule
%				%$\mathbf{0}\in\mathbb{R}^{rn_1}$	& vector filled with zeros \\ \midrule
%				\multicolumn{2}{l}{\textbf{Matrices:}} \\  
%				$A\in\mathbb{R}^{m_1\times n_1}$	& first-stage constraint matrix corresponds to decision vector $x$\\
%				$W_s:= W(\xi_s)\in\mathbb{R}^{m_2\times n_2}$	& second-stage constraint matrix corresponds to decision vector $y_s$\\
%				$T_s:= T(\xi_s)\in\mathbb{R}^{m_2\times n_1}$	& second-stage constraint matrix corresponds to decision vector $x$\\ \midrule
%				%$H_j:= H(\xi_j)\in\mathbb{R}^{rm_1\times n_1}$	&	nonanticipativity constraints matrix \\ \midrule
%				\multicolumn{2}{l}{\textbf{Functions:}} \\
%%				$\phi:\mathbb{R}^{m_2}\to\mathbb{R}$	&  optimal objective function of the second-stage program given a scenario\\
%				$Q:\mathbb{R}^{n_1-k_1}\times\mathbb{Z}^{k_1}\to\mathbb{R}$	& recourse function \\
%				$\mathcal{Q}:\mathbb{R}^{n_1-k_1}\times\mathbb{Z}^{k_1}\to\mathbb{R}$	& expected recourse function\\
%				\bottomrule
%			\end{tabular}
%			\begin{tablenotes}
%			\small
%			\item Remark. $\pmb{\xi}$ is the random element that realizes by one of the possible scenarios $\xi_1,\ldots,\xi_r$.		
%			\end{tablenotes}
%		\end{threeparttable}
%	}
%\end{table} 


\subsection{Solution Approaches}
%much of the solution methods in SIP is studied to resolve the difficulty of optimizing the form in Equation \ref{eq:rp}, 
%Much of the solution methods in SIP is studied to resolve the difficulty of optimizing the extensive form. %That is, the sum of the first-stage costs and the expected costs in the second-stage. 
Most computational solution methods in SIP assume that a single scenario evaluation is somehow tractable in a reasonable time. However, we often face the situation in which we need to consider a vast number of scenario realizations. In this case, EF easily becomes a large mixed integer program (MIP), but having a special structure in it. Solving this problem without exploiting the structure can quickly go with computational intractability. In this section, we introduce representative approaches to resolve the difficulties present in solving SIP when many scenarios are considered. \kk{Need to rewrite}

\subsubsection{Stage-wise decomposition}
This type of algorithms naturally optimize the objective function in Equation (\ref{eq:rp}) over the set of feasible solutions for the first-stage. To briefly explain, the problem is decomposed stage-wisely to derive associated master problem and subproblem. Then, the algorithm alternately solves each problem to seek the true optimum of the original problem.
 
To be more specific, at each iteration of the algorithm, it builds an outer linearization of the second-stage recourse function and uses a solution of the first-stage problem plus this linearization \cite{book:BL2011}. This cutting plane technique is called \textit{Benders decomposition} \cite{journal:B1962} in general or \textit{L-shaped method} in stochastic programming, from which a variants of stage-wise decomposition methods have been derived. 

L-shaped method has long history since 1962 even before its successful application to stochastic programming. For SIP, however, the method required to be revised mostly due to the integrality presents in problem. For example, the integer L-shaped method \cite{journal:LL1993} developed a cut generation method that approximates the second-stage recourse function only when the SIP is of pure binary first-stage and mixed integer second-stage. For the problems with the first-stage variables which are not necessarily binary, Car{\o}e and Tind \cite{journal:CT1998} suggested a method to use the duality of the second-stage integer program to generate cuts that build an approximation of the recourse function.

\subsubsection{Scenario-wise decomposition}
The advantage of using this type algorithms is their ability to allow us naturally parallelizing computation hence fully utilize the power of state-of-the-art computing cluster. Assuming the tractability of a single-scenario problem, this type algorithms are known to be one of the most effective solution methods to solve the real world large-scale problems.

To briefly explain the idea of scenario-wise decomposition, we first introduce the following terms for solution systems:
\begin{itemize}
	\item \textit{admissible}: a solution system that satisfies constraints for all scenarios
	\item \textit{implementable} (or \textit{nonanticipative}): a solution system where every scenario-specific solution has the same first-stage decision
	\item \textit{feasible}: a solution system that is both admissible and implementable
\end{itemize}

Scenario-wise decomposition starts by introducing the copies of the first-stage variables into the second-stage. Then, the EF in (\ref{sip:ef}) is written as below.
\begin{subequations} \label{sip:ef'}
	\begin{align}
	\textrm{(EF') }\min_{\mathrm{x},\mathrm{y}}\ z(x,y):=\ &\sum_{s=1}^{r}\PP(s)\left(c^{\top}x_s+q_s^{\top}y_s\right)	\label{eq:SIP_2-1}\\ 
	\mathrm{s.t.}\ &\sum_{s=1}^{r}H_s x_s=0, \label{eq:SIP_2-2} \\
	\ &(x_s,y_s)\in G_s,\quad \forall s\in\{1,\ldots,r\},	\label{eq:SIP_2-3}
	\end{align}
\end{subequations}
where $\mathrm{x}:=\{x_1,\ldots,x_r\}$ and the scenario feasibility set $G_s$ is defined as
\begin{align} 
G_s:=\left\{ (x_s,y_s): \ Ax_s\ge b,\  T_s x_s+W_s y_s\ge h_s,\ (x_s,y_s)\in X\times Y  \right\}. \label{eq:SIP_2-4}
\end{align}
The newly included constraints (\ref{eq:SIP_2-2}) (called \textit{nonanticipativity} constraints) guarantee $x_1=x_r$ and $x_s=x_{s-1}$ for $s=2,\ldots,r$. Here, the matrix $H_s$ is of dimension $l\times n_1$ with a suitable value $l$. %We assume that some first-stage solutions satisfying the first-stage constraints can give rise to infeasibility of RP-EF'. Without this property, the problem is called to have a (relatively) \textit{complete recourse}. 
Based on the RP-EF', there are two representative scenario-wise decomposition algorithms: Dual Decomposition (DD) and Progressive Hedging (PH). 

%We assume that SIP does not necessarily have relatively complete recourse. We recall that without this property there can exist an $\hat{x}\in X$ satisfying $A\hat{x}\ge b$ for which there does not exist a recourse $y\in\mathbb{R}^{m_2}$ satisfying $(\hat{x},y)\in G_s$ for some $s$. In other words, not every choice of the first-stage variables is guaranteed to have feasible recourse for all scenarios.

The main idea of DD algorithm is to relax the nonanticipativity constraints using Lagrange multipliers and then to restore them. Given set of multipliers, the problem is separable by scenarios hence the associated dual function can be optimized scenario-wise independently \cite{journal:CS1999}. The Lagrange multipliers can be updated by various methods, e.g., column generation \cite{journal:LMPS2013,journal:LS2004}, subgradient \cite{journal:CS1999,journal:PO2013,journal:A2013}, and cutting plane \cite{journal:LMPS2013}. The conceptual steps of the DD procedure can be explained as follows:
\begin{enumerate}
	\item Using Lagrangian multiplier (or dual variable) $\lambda$, relax the nonanticipativity constraints (\ref{eq:SIP_2-2}) by introducing them into the objective function (\ref{eq:SIP_2-1}).
	\item Let $k=0$ be the iteration step and set initial value on the $k$-step dual variable $\lambda^{(k)}$.
	\item Scenario-wisely decompose the relaxed problem with fixed $\lambda^{(k)}$ and solve it independently to get the $k$-step solutions.
	\item Update the best upper bound based on the $k$-step solutions.
	\item Go to step 3. with $k=k+1$ and then updating $\lambda^{(k+1)}$ based on the $k$-step solutions if the convergence criterion is yet to be satisfied.
\end{enumerate}
Due to the nonconvexities, there may exist a duality gap between relaxed problem and the original problem so that the primal feasibility needs to be attained by branch-and-bound or heuristic schemes. However, achieving that primal feasibility is not always available when the problem does not have \textit{relatively complete recourse} hence a crucial limitation of DD. Nevertheless, DD is known to be one of the effective methods since it often provides an useful tight lower bound for the original problem.

PH algorithm is developed by Rockafellar and Wets \cite{journal:RW1991} motivated by augmented Lagrangian theory. 
A brief steps of the PH procedure can be summarized as follows \cite{book:pyomo}:
\begin{enumerate}
	\item Let $k=0$ be the iteration step and $w_s^{(k)}=0$ for all $s\in\mathcal{S}$.
	\item For each scenario $s\in\mathcal{S}$, solve the problem (\ref{eq:SIP_2-1}), (\ref{eq:SIP_2-3}), and (\ref{eq:SIP_2-4}) only considering the single scenario to obtain the scenario solutions $(x_s^{(k)},y_s^{(k)})$.
	\item Then, take average of the scenario solutions to obtain an implementable first-stage decision $\bar{x}^{(k)}$.
	\item For each scenario $s$, obtain the scenario solution $(x_s^{(k+1)},y_s^{(k+1)})$ by solving the following minimization problem:
	\begin{align*}
	\min_{x_s,y_s}\left\{c^\top x_s+q_s^\top y_s + {w_s^{(k)}}^\top x_s +\frac{\rho}{2}\norm{x_s-\bar{x}^{(k)}}^2 : (x_s,y_s)\in G_s \right\},
	\end{align*}
	where ${w_s^{(k)}}:={w_s^{(k-1)}}+\rho(x_s^{(k)}-\bar{x}^{(k)})$ is the weight and $\rho$ is an algorithmic parameter for subgradient estimator, which both of them are for penalizing the lack of nonanticipativity.
%	\item For each scenario $s$, solutions are obtained for the problem of minimizing, subject to the problem constraints, the deterministic solution as in Step 1. plus terms that penalize the lack of implementability using a subgradient estimator for the nonanticipativity constraints and a squared penalty term.
	\item Go to step 2. with $k=k+1$ if the convergence criterion is yet to be satisfied. 
\end{enumerate}
A notable limitation of PH algorithm is that it does not always guarantee the convergence. However, PH is advantageous in the sense that it often provides high quality solution within a reasonable number of iterations. 

\subsection{Software Packages}

Despite of the long history of stochastic programming, not many software packages have been developed. In this section we discuss existing modeling languages and solves for SIP.

\subsubsection{Modeling language}

As a deterministic optimization problem, DEF can be written by any algebraic modeling languages (e.g., AMPL~\cite{ampl}, GAMS~\cite{gams}) \kk{please add refs}. Exploiting the block-angular structure embedded in stochastic programs, the problem can be written much more memory-efficiently in SMPS format. However, writing a problem in SMPS files can be cumbersome. Algebraic modeling languages provide a capability of modeling stochastic programming problems, including SIP.

We introduce two algebraic modeling languages for stochastic programming. Both modeling languages are designed to write stochastic programming problems in algebraic form, which allows modelers to write the problems more intuitively than in SMPS format.

\structjump\ is a \julia\ package that implements a parallel algebraic modeling framework for block-structured optimization models \cite{web:StructJuMP}. \structjump\ is an extension of \jump\ modeling package~\cite{journal:JuMP}, which has been recently paid much attention in the optimization community. A key design idea in \structjump\ is to use \jump\ model objects to effectively represent the block-angular structure of stochastic programming problem. In particular, the structure-exploit design of \structjump\ can efficiently distribute a large-scale problem data to multiple compute nodes in distributed computing settings (e.g., high-performance computing clusters).

\pysp\ is a stochastic programming extension of a \python-based algebraic modeling package \pyomo\ \cite{book:pyomo} that supports variants of mathematical optimization. In \pysp, users can express stochastic programming problems by directly extending deterministic models with specification of random parameters. A key advantage of \pysp\ is the capability of directly extending an existing deterministic model to a stochastic counterpart by specifying random parameters. \kk{What about distributed memory? Need more}

\subsubsection{Solver}
Several commercial or noncommercial solvers are available for SIP. In this subsection, we mainly introduce some noncommercial solvers: \pipssbb, \scip, \dsp, and \pysp.

\pipssbb\ \cite{web:PIPS-SBB} is a distributed-memory parallel SIP solver that implements parallelized branch-and-bound (B\&B) algorithm and reads \smps\ format. The embedded scheme is designed mainly to overcome the possible memory insufficiency resulted from the B\&B procedure for a large number of scenarios. However, the computational performance in terms of speed is known to have a long way to go before it is competitive with commercial MIP solvers \cite{proceeding:MOR2016}.

\scip\ \cite{SCIP} is one of the fast non-commercial MIP and MINLP solvers that provides various optimization framework such as branching, cutting plane separation, propagation, pricing, and Benders decomposition. The latest version of \scip\ has been extended towards reading \smps\ format hence now available as an SIP solver. \scip\ currently provides two ways to solve SIP instances.
\begin{itemize}
	\item Invoking MIP solver to solve EF
	\item L-shaped method
\end{itemize}
Although \scip\ recently started to support SIP, it is still preliminary phase as an SIP-oriented solver. For example, \scip\ is yet to provide any scenario-wise decomposition algorithm that fully utilizes the special structure presents in SIP.

\dsp\ \cite{web:DSP} is an open-source SIP solver implemented in \cpp. \dsp\ also reads \smps\ files as an input format and provides three ways to solve SIP instances.
\begin{itemize}
	\item Invoking standard MIP solver to solve EF
	\item L-shaped method
	\item Dual Decomposition algorithm
\end{itemize}
\dsp\ is known as a remarkable DD-based solver and consistently improves its computational performance \cite{journal:KZ2015}. \dsp\ provides parallel implementations for the two decomposition-based algorithms, which allow users fully exploit computing clusters and multi-core processors. In addition, \dsp\ provides \julia\ interface to improve its usability.

\pysp\ package, which we have introduced in the previous subsection, includes not only modeling framework but solver capability providing three ways to solve SIP instances. 
\begin{itemize}
	\item Invoking standard MIP solver to solve EF
	\item L-shaped method
	\item Progressive Hedging algorithm
\end{itemize}
\pysp's PH implementation gives an effective heuristic for approximating general multi-stage SIP \cite{journal:WWH2012}. \pysp\ also supports the distributed execution of the optimization which leverages the distributed solver capabilities that are provided in \pyomo\ library.


\pagebreak
 
\section{Summary of the test sets}\label{sec:summary}
\textcolor{red}{In this section, we explain information about the test problems and corresponding instances in summarized manner. This includes type, components, and instances for each problem. More detailed problem-specific explanation is available in Section \ref{sec:prob_desc}.}
\subsection{Type of problems}
\begin{table}[H]
	\centering
	\caption{Types of the problems in \texttt{SIPLIB 2.0}}
	\label{table:prob_type}
	\resizebox{\textwidth}{!}{%
		\begin{tabular}{@{}llll@{}}
			\toprule
			Problem name 		  & Description                                                        & Main reference              \\ \midrule
			\texttt{DCAP}         & Dynamic capacity planning with stochastic demand                   & Ahmed and Garcia \cite{journal:AG2004}                          \\
			\texttt{DCLP}		  &	Data center location problem									   & Kim et al. \cite{journal:KYZC2017}								\\	
			\texttt{MPTSPs}       & Multi-path traveling salesman problem with stochastic travel costs & Tadei et al. \cite{journal:TPP2017}                            \\
			\texttt{SIZES}        & Optimal product substitution with stochastic demand                & Jorjani et al. \cite{journal:JSW1999}          \\
			\texttt{SMKP}		  & Stochastic multiple knapsack problem                               & Angulo et al. \cite{journal:AAD2014}                            \\
			\texttt{SSLP}         & Stochastic server location problem                                 & Ntaimo and Sen \cite{journal:NS2005}                           \\
			\texttt{SUCW}         & Wind power stochastic unit commitment				               & Papavasiliou and Oren \cite{journal:PO2013}                       \\ \bottomrule
		\end{tabular}%
	}
\end{table}

\begin{table}[H]
	\centering
	\caption{Constraint type legend \cite{MIPLIB}}
	\label{table:constraint_type}
	\begin{tabular}{@{}lll@{}}
		\toprule
		Type           & Description        & Constraint form                                                                                           \\ \midrule
		\texttt{AGG} & Aggregation        & $a_i x_i+a_k x_k=b,\ x_i , x_k$ int. or cont., $a_i , a_k , b \in \mathbb{R}$                  \\
		\texttt{VBD} & Variable bound     & $x_i \le a_k x_k + b$ or $x_i\ge a_k x_k + b$, $x_i, x_k$ int. or cont., $a_k, b\in\mathbb{R}$ \\
		\texttt{PAR} & Set partition      & $\sum x_i = 1,$ $x_i$ binary                                                                   \\
		\texttt{PAC} & Set packing        & $\sum x_i \le 1,$ $x_i$ binary                                                                 \\
		\texttt{COV} & Set cover          & $\sum x_i \ge 1,$ $x_i$ binary                                                                 \\
		\texttt{CAR} & Cardinality        & $\sum x_i = b,$ $x_i$ binary, $b\in\mathbb{N}$                                                 \\
		\texttt{EQK} & Equality knapsack  & $\sum a_i x_i = b,$ $x_i$ binary, $a_i , b\in\mathbb{N}$                                       \\
		\texttt{BIN} & Bin packing        & $\sum a_i x_i  + a_k x_k \le a_k,$ $x_i$ binary, $a_i , a_k\in\mathbb{N}$                      \\
		\texttt{IVK} & Invariant knapsack & $\sum x_i \le b,$ $x_i$ binary, $b\in\mathbb{N}$                                               \\
		\texttt{KNA} & Knapsack           & $a_i x_i \le b$, $x_i$ binary, $a_i,b\in\mathbb{N}$                                            \\
		\texttt{IKN} & Integer knapsack   & $a_i x_i \le b$, $x_i\ge 0$ integer, $a_i,b\in\mathbb{N}$                                      \\
		\texttt{M01} & Mixed binary       & $\sum a_i x_i + \sum p_j s_j \le$ or $=b$, $x_i$ binary, $s_j$ cont., $a_i, p_j\in\mathbb{R}$  \\
		\texttt{GEN} & General            & All other constraint types                                                                     \\ \bottomrule
	\end{tabular}
\end{table}

\begin{table}[H]
	\centering
	\caption{Components of the problems in \texttt{SIPLIB 2.0} ($\mathbb{C},\mathbb{B},\mathbb{I}$: continuous, binary, integer variables)}
	\label{table:prob_class}
	\begin{tabular}{@{}lllll@{}}
		\toprule
		& \multicolumn{2}{l}{1st stage}                              				  	& \multicolumn{2}{l}{2nd stage}                             			        \\ \midrule
		Problem 	     & Variable                    & Constraint                   	& Variable                    & Constraint                  				    \\ \midrule
		\texttt{DCAP}    & $\mathbb{C}$, $\mathbb{B}$  & \texttt{VBB}                	& $\mathbb{B}$                & \texttt{PAR}, \texttt{M01} 			    		\\
		\texttt{DCLP}	 &							   &								& $\mathbb{C}$			 	  &													\\				
		\texttt{MPTSPs}  & $\mathbb{C}$, $\mathbb{B}$  & \texttt{PAR}, \texttt{GEN}		& $\mathbb{B}$                & \texttt{GEN}               						\\
		\texttt{SIZES}   & $\mathbb{I}$ 			   & \texttt{VBD}, \texttt{GEN} 	& $\mathbb{B}$, $\mathbb{I}$  & \texttt{IKN}             						\\
		\texttt{SMKP}    & $\mathbb{B}$                & \texttt{KNA}                	& $\mathbb{B}$                & \texttt{KNA}              						\\
		\texttt{SSLP}    & $\mathbb{B}$                & \texttt{IVK}, \texttt{GEN} 	& $\mathbb{C}$, $\mathbb{B}$  & \texttt{GEN}             						\\
		\texttt{SUCW}    & $\emptyset$                 &                              	& $\mathbb{C}$, $\mathbb{B}$  &                             					\\ \bottomrule
	\end{tabular}
\end{table}

% Please add the following required packages to your document preamble:
% \usepackage{booktabs}

% \usepackage{graphicx}
\begin{table}[H]
	\centering
	\caption{Number of components in each problem (for convenience, we skip the cardinality sign $|\cdot|$ for sets, i.e., for any set $S$, $S$ denotes $|S|$)}
	\label{table:num_components}
	\resizebox{\textwidth}{!}{%
		\begin{tabular}{@{}cccccc@{}}
			\toprule
			&                & \multicolumn{4}{c}{Components}                                                                         \\ \cmidrule(l){3-6} 
			&                & \#Continuous   & \#Binary                           & \#Integer            & \#Constraint              \\ \midrule
			\multirow{3}{*}{\texttt{DCAP}}   & 1$^{st}$ stage & $RT$           & $RT$                               & -                    & $RT$                      \\
			& 2$^{nd}$ stage & -              & $\left(1+R\right)NT$               & -                    & $(R+N)T$                  \\ \cmidrule(l){2-6} 
			& Total          & $RT$           & $RT+\left(1+R\right)NT\mathcal{S}$ & -                    & $RT+(R+N)T\mathcal{S}$    \\ \midrule
			\multirow{3}{*}{\texttt{DCLP}}   & 1$^{st}$ stage &                &                                    &                      &                           \\
			& 2$^{nd}$ stage &                &                                    &                      &                           \\ \cmidrule(l){2-6} 
			& Total          &                &                                    &                      &                           \\ \midrule
			\multirow{3}{*}{\texttt{MPTSPs}} & 1$^{st}$ stage & $N^2$          & $N^2$                              & -                    & $N^2+2N-1$                \\
			& 2$^{nd}$ stage & -              & $N^2 K$                            & -                    & $N^2$                     \\ \cmidrule(l){2-6} 
			& Total          & $N^2$          & $(1+\mathcal{S}K)N^2$              & -                    & $(1+\mathcal{S})N^2+2N-1$ \\ \midrule
			\multirow{3}{*}{\texttt{SIZES}}  & 1$^{st}$ stage & -              & $NT$                               & $NT$                 & $(1+N)T$                  \\
			& 2$^{nd}$ stage & -              & -                                  & $N^2 T$              & $2NT$                     \\ \cmidrule(l){2-6} 
			& Total          & -              & $NT$                               & $(1+N\mathcal{L})NT$ & $(1+N+2N\mathcal{L})T$    \\ \midrule
			\multirow{3}{*}{\texttt{SMKP}}   & 1$^{st}$ stage & -              & $2I$                               & -                    & $J$                       \\
			& 2$^{nd}$ stage & -              & $I$                                & -                    & $K$                       \\ \cmidrule(l){2-6} 
			& Total          & -              & $(2+\mathcal{S})I$                 & -                    & $J+K\mathcal{S}$          \\ \midrule
			\multirow{3}{*}{\texttt{SSLP}}   & 1$^{st}$ stage & -              & $J$                                & -                    & $1$                       \\
			& 2$^{nd}$ stage & $J$            & $IJ$                               & -                    & $I+J$                     \\ \cmidrule(l){2-6} 
			& Total          & $J\mathcal{S}$ & $(1+I\mathcal{S})J$                & -                    & $1+(I+J)\mathcal{S}$      \\ \midrule
			\multirow{3}{*}{\texttt{SUCW}}   & 1$^{st}$ stage &                &                                    &                      &                           \\
			& 2$^{nd}$ stage &                &                                    &                      &                           \\ \cmidrule(l){2-6} 
			& Total          &                &                                    &                      &                           \\ \bottomrule
		\end{tabular}%
	}
\end{table}

\subsection{Instance catalog}

% Please add the following required packages to your document preamble:
% \usepackage{booktabs}
% \usepackage{graphicx}
\begin{table}[H]
	\centering
	\caption{Problem-specific instance naming rules}
	\label{table:instance_naming_rule}
	\resizebox{\textwidth}{!}{%
		\begin{tabular}{@{}lll@{}}
			\toprule
			Problem & Instance name                 & Description                                                                    					      \\ \midrule
			\texttt{DCAP}    & \texttt{DCAP\_$R$\_$N$\_$T$\_$\mathcal{S}$}    &   $R$: number of resources, $N$: number of tasks, $T$: number of time periods, $\mathcal{S}$: number of scenarios        \\
			\texttt{DCLP}	 &								&																										\\
			\texttt{MPTSPs}  & \texttt{MPTSPs\_$D$\_$N$\_$\mathcal{S}$} &$D$: node distribution strategy, $N$: number of nodes, $\mathcal{S}$: number of scenarios\\
			\texttt{SIZES}   & \texttt{SIZES\_$\mathcal{L}$}                            & $\mathcal{L}$: number of scenario   															\\
			\texttt{SMKP}    &   \texttt{SMKP\_$I$\_$\mathcal{S}$}    &   $I$:number of types for item, $\mathcal{S}$: number of scenarios  													 \\
			\texttt{SSLP}    &   \texttt{SSLP\_$I$\_$J$\_$\mathcal{S}$}      &    $I$: number of clients, $J$: number of server locations, $\mathcal{S}$: number of scenarios                 				   \\
			\texttt{SUCW}    & 	\texttt{SUCW\_$D$\_$\mathcal{S}$}    &  $D$: day type, $\mathcal{S}$: number of scenarios                                                 						 \\ \bottomrule
		\end{tabular}%
	}
\end{table}
\pagebreak

\section{How to run a test, generate new instance, and convert to \texttt{SMPS}}\label{sec:tutorial}
We explain the structure of \texttt{SIPLIB 2.0}. We explain procedure to generate new instances with user-generated scenario data using \texttt{Julia} scripts. We explain how to convert \texttt{JuMP.Model}-type object to \texttt{SMPS} files.
\pagebreak

\section{Implementation of SMPS Writer}\label{sec:smps_writer}
We describe our Julia implementation, how to model SIP and generate SMPS files..
\pagebreak

\section{Solution report}\label{sec:solution}
In this section, we report 
%\begin{itemize}
%  \item Coin-SMI (\url{https://github.com/coin-or/Smi}): compile with CPLEX; can read SMPS and solve in extensive form.
%  \item DSP: Benders/dual decomposition
%  \item PySP: Progressive Hedging
%\end{itemize}
%
%- problem characteristics
%
%- preliminary results using CPLEX

\pagebreak

\section{Concluding remarks}
\textcolor{red}{(not finished yet) Any further contribution or suggestions for \texttt{SIPLIB 2.0} are always welcomed. Better solutions than discovered so far, more functions, more problems with \texttt{Julia} scripts for instance generation, more effective classification rules, etc.}
\pagebreak

%\begin{acknowledgements}
%\end{acknowledgements}

\appendix
\section*{Appendices}
\section{Problem descriptions} \label{sec:prob_desc}

In this section, we introduce details for each problem in \texttt{SIPLIB 2.0}. We also explain data generation procedures. Due to limited access to the original data in reference papers, we selectively choose the methods from several available references and modify some of them without harming validity. Also, we guess some parameters about scenario generation to make the procedure clear.  %This does not mean that users need to know the whole procedure in order to generate new scenario data. Those who feel the given instances are not large enough can simply generate more scenario data by just modifying the parameter corresponding to the number of scenarios.

\subsection{\texttt{DCAP}: Dynamic capacity planning with stochastic demand}
\texttt{DCAP} is the problem of determining a capacity expansion schedule for a set of resources, and the assignment of resource capacity to task with stochastic requirement over a multi-period planning horizon. In \texttt{SIPLIB}, 12 instances are available in \texttt{SMPS} format with the largest instance comprises of 500 scenarios correspond to the size of 9,012 rows and 18,018 columns. We refer to \cite{journal:AG2004} for writing \texttt{Julia} scripts.
%\texttt{DCAP} is a problem of deciding the capacity expansion schedule for $m$ resources over $T$ time periods in order to satisfy the processing requirements of $n$ tasks.
\subsubsection{\texttt{DCAP}: Mathematical formulation}
We consider the problem of deciding the capacity expansion schedule for $|R|$ resources over $|T|$ time periods to satisfy the processing requirements of $|N|$ tasks where $R$, $T$, and $N$ denote set of resources, set of time periods, and set of tasks, respectively. We define decision variables: the first-stage continuous variable $x_{it}$ for the capacity acquisition of resource $i$ in period $t$ and the second-stage binary variable $y_{ijt}^s$ to indicate whether resource $i$ is assigned to task $j$ in period $t$ under scenario $s$. Additional first-stage binary variable $u_{it}$ is for logical constraint whether or not we decided to acquire more capacity of resource $i$ in period $t$. Hence, for all resource $i\in R$ and time $t\in T$, $u_{it}=1$ if $x_{it}>0$, $u_{it}=0$ otherwise. 

Under the definition of the decision variables, the extensive form of \texttt{DCAP} is written below and the summarized notation is available in Table \ref{dcap:notation}.

\begin{subequations}
	\begin{align}
	(\texttt{DCAP})\ \textrm{min}\ &\sum_{t\in T}\sum_{i\in R}\left(\alpha_{it}x_{it}+\beta_{it}u_{it}\right)+\sum_{s\in\mathcal{S}}\PP(s)\sum_{t\in T}\sum_{i\in R\cup\{0\}}\sum_{j\in N}c_{ijt}^{s}y_{ijt}^s	\label{dcap:obj} \\
	\textrm{s.t.}\ & \textcolor{red}{x_{it}\le \textrm{M}u_{it}},\quad\forall i\in R,\ \forall t\in T,	\label{dcap:b}\\
	&\sum_{j\in N}d_{jt}^s y_{ijt}^s\le\sum_{\tau=1}^{t}x_{i\tau},\quad\forall i\in R,\ \forall t\in T,\ \forall s\in\mathcal{S},\label{dcap:c}\\
	&\sum_{i\in R\cup \{0\}}y_{ijt}^s=1,\quad\forall j\in N,\ \forall t\in T,\ \forall s\in\mathcal{S}, \label{dcap:d} \\
	&x_{it}\ge 0,\quad\forall i\in R,\ \forall t\in T,\label{dcap:e} \\
	&u_{it}\in\{0,1\}, \quad\forall i\in R,\ \forall t\in T,\label{dcap:f}\\
	&y_{ijt}^s\in\{0,1\},\quad\forall i\in R\cup\{0\},\ \forall j\in N,\ \forall t\in T,\ \forall s\in\mathcal{S},\label{dcap:g}
	\end{align}
\end{subequations}
The objective function (\ref{dcap:obj}) is to minimize total expected cost for the capacity expansion schedule. The first double summation denotes the expansion cost for resource $i$ in period $t$ where $\alpha_{it}$ and $\beta_{it}$ are the variable and fixed cost, respectively. The second term in the objective function represents the expected assignment cost in period $t$ over all scenario $s\in\mathcal{S}$. Note that a dummy resource $i=0$ is included with infinite capacity. The cost $c_{0jt}^s$ denotes the penalty of failing to assign a resource to task $j$. The dummy resource enforces the \textit{complete recourse property}, which ensures that there is a feasible second-stage assignment in all periods and all scenarios for any capacity acquisition schedule \cite{journal:AG2004}. Constraint (\ref{dcap:b}) is the logical constraint containing a suitably large value M \textcolor{red}{(e.g., we set $\textrm{M}\leftarrow \textrm{M}_{t}\equiv \max_{s\in\mathcal{S}} \sum_{j\in N} d_{jt}^s$)} to define the cost for capacity expansion. Constraint (\ref{dcap:c}) reflects that the processing requirement of all tasks assigned to a resource in any period cannot exceed the installed capacity in that period under all scenarios. Constraint (\ref{dcap:d}) guarantees that each task needs to be assigned to exactly one resource in each period under all scenarios. Finally, constraints (\ref{dcap:e})-(\ref{dcap:g}) restrict the space from which the variables take values.


\begin{table}[H]
	\caption{Notations for \texttt{DCAP}}
	\label{dcap:notation}
	\resizebox{\textwidth}{!}
	{
		\begin{tabular}{ll}
			\toprule
			\multicolumn{2}{l}{\textbf{Index sets:}} \\
			$R$ & index set of resources ($i\in R\cup\{0\}$ where $0$ is a dummy resource with infinite capacity) \\ 
			$N$ & index set of tasks ($j\in N$)\\ 
			$T$ & index set of time periods ($t\in T$)\\
			$\mathcal{S}$ & index set of scenarios ($s\in \mathcal{S}$) \\ \midrule
			\multicolumn{2}{l}{\textbf{Parameters:}} \\
			$\alpha_{it}$ & variable cost for expanding capacity of resource $i$\\ 
			$\beta_{it}$ & fixed cost for expanding capacity of resource $i$ \\ 
			$c_{ijt}^{s}$ & cost of processing task $j$ using resource $i$ in period $t$ under scenario $s$ \\ 
			$d_{jt}^s$	& processing requirement for task $j$ in period $t$ under scenario $s$		\\
			$\PP(s)$ & \textrm{the probability of occurence of scenario $s$} \\ \midrule
			\multicolumn{2}{l}{\textbf{Decision variables:}} \\
			$x_{it}$ ($1^{\textrm{st}}$ stage) & capacity acquisition amount of resource $i$ in period $t$ \\ 
			$u_{it}$ ($1^{\textrm{st}}$ stage)& 1 if capacity of resource $i$ is expanded in period $t$, 0 otherwise \\ 
			$y_{ijt}^s$ ($2^{\textrm{nd}}$ stage)& 1 if resource $i$ is assigned to task $j$ in period $t$ under scenario $s$, 0 otherwise\\
			\bottomrule
		\end{tabular}
	}
\end{table} 

\subsubsection{\texttt{DCAP}: Data generation}
There are four factors that define the instance of \texttt{DCAP}: $|R|$, $|N|$, $|T|$, and $|\mathcal{S}|$. Once we decide the factors by, $n_R=|R|$, $n_N=|N|$, $n_T=|T|$, and $n_\mathcal{S}=|\mathcal{S}|$, each instance is named by \texttt{DCAP\_$n_R$\_$n_N$\_$n_T$\_$n_\mathcal{S}$}. \textcolor{red}{Since no original parameter is available, we randomly generate the parameters as long as they are valid.} Let $U$ be a continuous uniform random variable: $U\sim Unif(0,1)$. Then, the parameters are generated as follows:
\begin{align*}
	\alpha_{it}&=5U+5,\quad\forall i\in R,\ \forall t\in T, \\
	\beta_{it} &=40U+10,\quad\forall i\in R,\ \forall t\in T, \\
	c_{ijt}^s  &=5U+5,\quad\forall i\in R,\ \forall j\in N,\ \forall t\in T,\ \forall s\in\mathcal{S}, \\
	c_{0jt}^s  &=500U+500,\quad\forall j\in N,\ \forall t\in T,\ \forall s\in\mathcal{S}, \\
	d_{jt}^s   &=U+0.5,\quad\forall j\in N,\ \forall t\in T,\ \forall s\in\mathcal{S}.
\end{align*}
\textcolor{red}{For the first-stage parameters $\alpha_{it}$ and $\beta_{it}$, we fix the random seed to make it deterministic.}

\subsection{\texttt{DCLP}: Data center location problem}

\subsubsection{\texttt{DCLP}: Mathematical formulation}

\subsubsection{\texttt{DCLP}: Data generation}


\subsection{\texttt{MPTSPs}: Mutli-path Traveling Salesman Problem with Stochastic Travel Times}
\texttt{MPTSPs} is a variant of the travelling salesman problem (TSP) where a set of paths exists between any two nodes and each path is chracterized by a random travel time. 

In \texttt{SIPLIB}, only limited data (e.g., number of nodes, coordinates of nodes, generated travel times) are provided and no \texttt{SMPS} file is available. We mainly refer to \cite{journal:PGM2017} for deriving the mathematical formulation. Due to the malfunction of subtour breaking constraints in the reference model, we refer to another paper \cite{journal:LSD1990} to break subtours. Combining the two references, we construct the forthcoming single commodity flow-based formulation \texttt{MPTSPs} that is used for \texttt{SIPLIB 2.0}. 
\subsubsection{\texttt{MPTSPs}: Mathematical formulation}
We consider a two-stage SIP with recourse. The travel time oscillation $e_{ij}^k$ by using path $k$ between nodes $i$ and $j$. We present each realization (scenario) of random travel time oscillation by $e_{ijk}^{s}$ where $s$ indicates the scenario. In \texttt{MPTSPs}, at the first stage, the decision-maker does not have any information about the travel time oscillation. The tour paths among the nodes, however, should be determined before the complete information is available. The first stage decision variable $y_{ij}$ is represented by the selection of nodes $i$ and $j$ to be visited in a tour. In the second stage where the random travel time $c_{ijk}^{s}$ are avilable, the paths $k$ between each couple of nodes $i$ and $j$ under scenario $s$, $x_{ijk}^{s}$ can be calculated. 

Let $N$ and $K_{ij}$, respectively, bethe finite set of nodes of the graph and the set of paths between the pair of nodes $i,j\in N$. We denote with $\mathcal{S}$ the set of scenarios with associated equally distributed probability of each scenario $\PP(s)$, i.e., $\PP(s)\equiv 1/|\mathcal{S}|$. Each path $k\in K_{ij}$ between nodes $i,j\in N$ is chracterized by a non-negative estimation of the mean unit travel time $\bar{c}_{ij}$ and a non-negative unit random travel time $c_{ijk}^{s}$ under the scenario $s\in S$. Let $e_{ijk}^{s}\equiv c_{ijk}^{s}-\bar{c}_{ij}$ be the error on the travel time estimated for the path $k\in K_{ij}$ under time scenario $s\in S$.

The first stage binary variables $y_{ij}=1$ if node $j\in N$ is visited right after node $i\in N$, 0 otherwise. The second stage binary variables $x_{ijk}^{s}=1$ if path $k\in K_{ij}$ between nodes $i,j\in N$ is selected at the second stage, 0 otherwise. We have one more set of first stage variables $\phi_{ij}$ which is introduced to break the subtours \cite{journal:LSD1990}. The non-negative continuous variables $\phi_{ij}$ describe the flow of a single commodity to node 1 from every other nodes (without loss of generality, 1 is the starting node). 

The extensive form of \texttt{MPTSPs} is as follows and the notations used are summrized in Table \ref{mptsps:notation}.
\begin{subequations}
	\begin{align}
	(\texttt{MPTSPs})\ \textrm{min}\ &\sum_{i\in N}\sum_{j\in N}\bar{c}_{ij}y_{ij}+\sum_{s\in \mathcal{S}}\PP(s)\sum_{i\in N}\sum_{j\in N}\sum_{k\in K_{ij}}e_{ijk}^{s}x_{ijk}^{s} \label{mptsps:obj} \\ 
	\textrm{s.t.}\ &\sum_{j\in N:j\neq i}y_{ij}=1,\quad\forall i\in N, \label{mptsps:con1} \\ 
	&\sum_{i\in N:i\neq j}y_{ij}=1,\quad\forall j\in N,\label{mptsps:con2} \\ 
	&\sum_{j\in N}\phi_{lj}-\sum_{i\in N\backslash\{1\}}\phi_{il}=1,\quad\forall l\in N\backslash\{1\}, \label{mptsps:con3}  \\ 
	&\phi_{ij}\le \left(|N|-1\right)y_{ij},\quad\forall i\in N\backslash\{1\},\ \forall j\in N,  \label{mptsps:con4} \\ 
	&\sum_{k\in K_{ij}}x_{ijk}^{s}=y_{ij},\quad\forall i\in N,\ \forall j\in N,\ \forall s\in \mathcal{S}, \label{mptsps:con5} \\ 
	&y_{ij}\in \{0,1\},\quad \forall i\in N,\ \forall j\in N, \label{mptsps:con6} \\ 
	&\phi_{ij} \ge 0, \quad \forall i\in N,\ \forall j\in N. \label{mptsps:con7} \\
	&x_{ijk}^{s}\in\{0,1\},\quad\forall i\in N,\ \forall j\in N,\ \forall k\in K_{ij},\ \forall s\in \mathcal{S}, \label{mptsps:con8} 
	\end{align}
\end{subequations}
The first sum in the objective function (\ref{mptsps:obj}) represents the first stage travel cost, while the second sum represents the recourse action, consisting in choosing the best path $k\in K_{ij}$ under scenario $s\in\mathcal{S}$. The constraints (\ref{mptsps:con1}) and (\ref{mptsps:con2}) form the assignment constraints and ensure that each node is visited only once. Given the fixed values of $y_{ij}$, constraint (\ref{mptsps:con3}) and (\ref{mptsps:con4}) form a network flow problem, and therefore the $\phi_{ij}$ values will be integer. In case the solutions of the above formulation contain at least one subtour, the constraints (\ref{mptsps:con3}) and (\ref{mptsps:con4}) are violated. Moreover, no tour can exist that does not contain node 1 by the two constraints. For more explanation on the subtour breaking mechanism accompanied with rigorous proof, refer to \cite{journal:GG1978}. The constraint (\ref{mptsps:con5}) guarantees that path $k$ between nodes $i$ and $j$ can be chosen at stage 2 only if nodes $i$ and $j$ were part of the tour fixed at stage 1. Finally, the constraints (\ref{mptsps:con6})-(\ref{mptsps:con8}) restrict the space from which the variables take values.

\begin{table}[H]
	\caption{Notations for \texttt{MPTSPs}}
	\label{mptsps:notation}
	\resizebox{\textwidth}{!}
	{
		\begin{tabular}{ll}
			\toprule
			\multicolumn{2}{l}{\textbf{Index sets:}} \\
			$N$ & \textrm{index set of nodes ($i,j,l\in N$)} \\ 
			$K_{ij}$ & \textrm{index set of paths between nodes $i$ and $j$ ($k\in K_{ij}$)} \\ 
			$\mathcal{S}$ & \textrm{index set of scenarios ($s\in \mathcal{S}$)}\\ \midrule
			\multicolumn{2}{l}{\textbf{Parameters:}} \\
			$c_{ijk}^{s}$ & \textrm{unit random travel time of path $k$ between nodes $i,j$ under scenario $s$} \\ 
			$\bar{c}_{ij}$ & \textrm{estimation of the mean unit travel time (expectation of $c_{ijk}^{s}$ over all $s$ and $k$)} \\ 
			$e_{ijk}^{s}$ & \textrm{the error on the travel time on estimated for arc $(i,j)$ and path $k$ under scenario $s$} \\ 
			$\PP(s)$ & \textrm{the probability of occurence of scenario $s$} \\  \midrule
			\multicolumn{2}{l}{\textbf{Decision variables:}} \\
			$\phi_{ij}$ ($1^{\textrm{st}}$ stage) & \textrm{the nonnegative real-valued flow on arc $(i,j)$}\\
			$y_{ij}$ ($1^{\textrm{st}}$ stage)& \textrm{1 if path $k$ between nodes $i,j\in N$ is selected at the second stage, 0 otherwise} \\  
			$x_{ijk}^{s}$ ($2^{\textrm{nd}}$ stage) & \textrm{1 if node $j$ is visited just after node $i$, 0 otherwise} \\ 
			\bottomrule
		\end{tabular}
	}
\end{table} 


\subsubsection{\texttt{MPTSPs}: Data generation}
We follow the scenario generation methods described through the references \cite{journal:MPT2014,journal:PGM2017,journal:TPP2017}. For \texttt{MPTSPs}, there are three mainly distinguished characteristics for each instance: the nodes partition strategy ($D\in\{D0,D1,D2,D3\}$, explanation on each strategy is forthcoming), the number of nodes ($|N|\in\{2,3,\ldots\}$), and the number of scenarios ($|S|\in\{1,2,\ldots\}$). Another important charicteristic $|K_{ij}|\in\{1,2,3,\ldots\}$ is the number of paths for each edge which is fixed by 3 as a default following \cite{journal:TPP2017}. Once we decide $D$, $|N|$, and $|S|$ by $D=Dx$, $|N|=n_N$, and $|\mathcal{S}|=n_\mathcal{S}$, each instance is named by \texttt{MPTSPs\_$Dx$\_$n_N$\_$n_\mathcal{S}$}.

The nodes are distributed in a circle with radius equal to $r$ km. We use Cartesian coordinate system where the geometric center of the circle is $(r,r)$. The nodes are distinguished by two subsets: \textit{central} and \textit{suburban}. If the Euclidean distance between a node and the geometric center is less than or equal to the half of the radius ($r/2$), then the node is of \textit{central} type. Otherwise, if the Euclidean distance is greater than the half of the radius, the node is of \textit{suburban} type. Each arc between any two nodes $i$ and $j$ is either \textit{homogeneous} or \textit{heterogeneous}. If the two nodes are of the same type of node, i.e., both are \textit{central} or both are \textit{suburban}, the type of the arc is \textit{homogeneous}. Otherwise, the type of the arc is \textit{heterogeneous}. Later, the travel time of each path between two nodes are affected by the type of arc. 

The nodes are generated by one of the following distribution strategies:
\begin{itemize}
	\item $D0$: All the nodes are \textit{central}.
	\item $D1$: All the nodes are \textit{suburban}.
	\item $D2$: 3/4 of the nodes are \textit{central} and the remaining 1/4 are \textit{suburban}.
	\item $D3$: 1/2 of the nodes are \textit{central} and the remaining 1/2 are \textit{suburban}.
\end{itemize}

Given $D,|N|$ and $|S|$, the next procedure can be summarized as follows:
\begin{enumerate}
	\item Generate $|N|$ nodes based on the predetermined strategy $D$. Then, the nodes are generated by acceptance-rejection procedure with uniform random number generation. Again following \cite{journal:TPP2017}, we fix $r=7$\textit{km}. 
	\item Calculate Euclidean distances between the nodes ($EC_{ij}$).
	\item We guess and fix the deterministic velocity profile by 40\textit{km/h} for the \textit{central} nodes and 80\textit{km/h} for the \textit{suburban} nodes: $v_{cntr}=40$ and $v_{sbrb}=80$.
	\item Generate random travel times ($c_{ijk}^{s}$) for each scenario $s$.
	\begin{itemize}
		\item The velocity for traveling arc $(i,j)$ is affected by its arc type.
		%\item Each arc has $|K_{ij}|$ paths (we have fixed it $|K_{ij}|=3$ for all $i,j$).
		\item If the arc is \textit{homogeneous}, the random travel time of all the paths are generated only based on the corresponding velocity profile.
		\item If the arc is \textit{heterogeneous}, $\ceil*{\frac{|K_{ij}|}{3}}$ paths are generated based on $v_{cntr}=40$ and the remaining paths are generated based on $v_{sbrb}=80$. 
		\item The velocities are distributed by $Unif(\frac{v}{2},2v)$ for $v=v_{cntr},v_{sbrb}$.
		\item In summary, if the arc $(i,j)$ is \textit{homogeneous}, 
		\begin{align*}
		c_{ijk}^{s}\sim\left\{ \begin{array}{ll} \frac{EC_{ij}}{Unif(\frac{v_{cntr}}{2},2v_{cntr})} & \textrm{if $i,j$ are both \textit{central},} \\
		\frac{EC_{ij}}{Unif(\frac{v_{sbrb}}{2},2v_{sbrb})} & \textrm{if $i,j$ are both \textit{suburban},}	\end{array} \right.\ \forall k\in K_{ij}.
		\end{align*}
		\item Otherwise, if $(i,j)$ is \textit{heterogeneous},
		\begin{align*}
		c_{ijk}^{s}\sim\left\{ \begin{array}{ll} \frac{EC_{ij}}{Unif(\frac{v_{cntr}}{2},2v_{cntr})} & \textrm{for $k\in\left\{1,\ldots,\ceil*{\frac{|K_{ij}|}{3}}\right\}$,} \\
		\frac{EC_{ij}}{Unif(\frac{v_{sbrb}}{2},2v_{sbrb})} & \textrm{for $k\in\left\{\ceil*{\frac{|K_{ij}|}{3}}+1,\ldots,|K_{ij}|\right\}$.}	\end{array} \right.
		\end{align*}
	\end{itemize}
	\item Finally, we multiply 3600 for each component of $c_{ijk}^{s}$ to convert the unit from \textit{hours} to \textit{seconds}.
\end{enumerate}


\subsection{\texttt{SIZES}: Selection of an optimal subset of sizes}
\texttt{SIZES} is a simplified version of the cutting-stock problem with multi-period stochastic demand. We only consider the two-periods (i.e., two stage) model to follow \cite{journal:JSW1999} as well as the SIP of interest discussed in Section \ref{sec:sip}. The first period demand is deterministic whereas demand for the second period are stochastic. 

In \texttt{SIPLIB}, only three instances are available in \texttt{SMPS} files. We refer to the mathematical formulation in \cite{journal:JSW1999} to construct \texttt{JuMP.Model}. Due to some unclear explanations (or typo), we slightly modify the formulation and use it for \texttt{SIPLIB 2.0}.

\subsubsection{\texttt{SIZES}: Mathematical formulation}
% here needs to be modified. I change the formulation.
\textcolor{red}{
	Suppose a product is available in a finite number $|N|$ of sizes where 1 is the index of the smallest size and $|N|$ is the index of the largest size. Further, suppose size $i$ is substitutable for size $j$ if $i>j$, i.e., larger-sized items may fulfill demand for smaller sizes. Unlike typical cutting-stock problem, an item cannot be substituted into several pieces. 
	Let $p_i$ be the unit production cost for size $i$. Generally $p_i>p_j$ for $i>j$. Let $s$ be the setup cost for producing units of any size and $r$ be the unit penalty cost of meeting demand for size $j$ with a larger size $i$. Let $d_{jt}^l$ be the stochastic demand for size $j$ at time $t$ under scenario $l$. Let $c_t^l$ be the stochastic production capacity at time $t$ under scenario $l$. $\PP(l)$ is the equiprobable probability of occurence for scenario $l$.
	We introduce three decision variables. The first-stage integer variable $y_{it}$ is the number of units of sizes $i$ produced at time $t$. The second-stage integer variable $x_{ijt}^l$ denotes the number of units of size $i$ cut to meet demand for smaller size $j$ at time $t$ under scenario $l$. The second-stage binary variable $z_i$ denotes whether or not we produce size $i$ item at time $t$ under scenario $l$. 
	Based on the above definitions, \texttt{SIZES} can be fomulated by the following extensive form.
}
%\begin{subequations}
%	\begin{align}
%	(\texttt{SIZES})\ \textrm{min}\ &\sum_{t\in T}\sum_{i\in N} p_i y_{it} + \sum_{l\in \mathcal{L}} \PP(l)\sum_{t\in T}\left(\sum_{i\in N} sz_{it}^l+r\sum_{i\in N\backslash\{1\}}\sum_{j=1}^{i-1}x_{ijt}^l\right) \label{sizes:obj}\\
%	\textrm{s.t.}\ &\sum_{i\in N}y_{it}\le c_{t}^l,\quad \forall t\in T,\ \forall l\in \mathcal{L}, \label{sizes:b}\\
%	&\sum_{i=j}^{|N|} x_{ijt}^l \ge d_{jt}^l,\quad\forall j\in N,\ \forall t\in T,\  \forall l\in \mathcal{L},\label{sizes:c}\\
%	&\sum_{t'=1}^{t}\sum_{j=1}^{i}x_{ijt'}^l\le\sum_{t'=1}^{t}y_{it'}, \quad\forall i\in N,\ \forall t\in T,\ \forall l\in \mathcal{L},\label{sizes:d}\\
%	&y_{it}\le c_{t}^l z_{it}^l,\quad\forall i \in N,\ \forall t\in T,\ \forall l\in \mathcal{L},\label{sizes:e}\\
%	&y_{it}\in\mathbb{Z}_+,\quad \forall j\in N,\ \forall t\in T,\label{sizes:f}\\
%	&x_{ijt}^l\in\mathbb{Z}_+,\quad\forall i\in N,\ \forall j\in N,\ \forall t\in T,\ \forall l\in \mathcal{L},\label{sizes:g}\\
%	&z_{it}^l\in\{0,1\},\quad\forall i\in N,\ \forall t\in T,\ \forall l\in \mathcal{L}.\label{sizes:h}
%	\end{align}
%\end{subequations}

\begin{subequations}
	\begin{align}
	(\texttt{SIZES})\ \textrm{min}\ &\sum_{t\in T}\sum_{i\in N} \left(p_i y_{it}+sz_{it} \right) + 
	\sum_{l\in \mathcal{L}} \PP(l)\sum_{t\in T}\sum_{i\in N\backslash\{1\}}\sum_{j=1}^{i-1}rx_{ijt}^l \label{sizes:obj}\\
	\textrm{s.t.}\ &\sum_{i\in N}y_{it}\le c_{t},\quad \forall t\in T, \label{sizes:b}\\
	&y_{it}\le c_{t} z_{it},\quad\forall i \in N,\ \forall t\in T,\label{sizes:e}\\
	&\sum_{t'=1}^{t}\sum_{i=j}^{|N|} x_{ijt'}^l \ge d_{jt}^l,\quad\forall j\in N,\ \forall t\in T,\  \forall l\in \mathcal{L},\label{sizes:c}\\
	&\sum_{t'=1}^{t}\sum_{j=1}^{i}x_{ijt'}^l\le\sum_{t'=1}^{t}y_{it'}, \quad\forall i\in N,\ \forall t\in T,\ \forall l\in \mathcal{L},\label{sizes:d}\\
	&y_{it}\in\mathbb{Z}_+,\quad \forall j\in N,\ \forall t\in T,\label{sizes:f}\\
	&z_{it}\in\{0,1\},\quad\forall i\in N,\ \forall t\in T,\label{sizes:h}\\
	&x_{ijt}^l\in\mathbb{Z}_+,\quad\forall i\in N,\ \forall j\in N,\ \forall t\in T,\ \forall l\in \mathcal{L}.\label{sizes:g}
	\end{align}
\end{subequations}
\textcolor{red}{
	The first sum of the objective function (\ref{sizes:obj}) is the unit costs for producing items for all time periods. The second term corresponds to the expectation of the setup costs and penalty costs for substituting items. Constraint (\ref{sizes:b}) ensures the production for each period cannot exceed the capacity under all scenarios. Constraint (\ref{sizes:c}) and (\ref{sizes:d}) guarantee the demand for each item can be met for all time periods and for all scenarios. In particular, constraint (\ref{sizes:d}) means the demand can be met by the items that are produced in the previous periods. Constraint (\ref{sizes:e}) enforces the production limits. Constraints (\ref{sizes:f})-(\ref{sizes:h}) are binary or integer restrictions of decision variables.
}
\begin{table}[H]
	\caption{Notations for \texttt{SIZES}}
	\label{sizes:notation}
	\resizebox{\textwidth}{!}
	{
		\begin{tabular}{ll}
			\toprule
			\multicolumn{2}{l}{\textbf{Index sets}} \\
			$N$ & \textrm{index set of items ($i,j\in N$)} \\ 
			$T$ & \textrm{index set of time periods ($t\in T$)} \\ 
			$\mathcal{L}$ & \textrm{index set of scenarios ($l\in\mathcal{L}$)}\\ \midrule
			\multicolumn{2}{l}{\textbf{Parameters}} \\
			$p_{i}$ & unit production cost for item $i$\\
			$s$	& setup cost for producing any item\\
			$r$ & unit cutting cost\\ 
			$c_{t}$ & production capacity at time $t$\\
			$d_{it}^l$ &	demand for item $i$ at time $t$ under scenario $l$\\
			$\PP(l)$ & the probability of occurence of scenario $l$\\ \midrule
			\multicolumn{2}{l}{\textbf{Decision variables}} \\
			$y_{it}$ ($1^{\textrm{st}}$ stage)  & number of units of size $i$ produced at time $t$ \\
			$z_{it}$ ($1^{\textrm{nd}}$ stage)& 1 if we produce size $i$ at time $t$, 0 otherwise\\
			$x_{ijt}^l$ ($2^{\textrm{nd}}$ stage) & number of units of size $i$ cut to meet demand for smaller size $j$ at time $t$ under scenario $l$\\ 
			\bottomrule
		\end{tabular}
	}
\end{table} 

%\begin{table}[H]
%	\caption{Notations for \texttt{SIZES}}
%	\label{sizes:notation}
%	\resizebox{\textwidth}{!}
%	{
%		\begin{tabular}{ll}
%			\toprule
%			\multicolumn{2}{l}{\textbf{Index sets}} \\
%			$N$ & \textrm{index set of items ($i,j\in N$)} \\ 
%			$T$ & \textrm{index set of time periods ($t\in T$)} \\ 
%			$\mathcal{L}$ & \textrm{index set of scenarios ($l\in\mathcal{L}$)}\\ \midrule
%			\multicolumn{2}{l}{\textbf{Parameters}} \\
%			$d_{it}^l$ &	demand for item $i$ at time $t$ under scenario $l$\\
%			$p_{i}$ & unit production cost for item $i$\\
%			$s$	& setup cost for producing any item\\
%			$r$ & unit cutting cost\\ 
%			$c_{t}^l$ & production capacity at time $t$ under scenario $l$\\
%			$\PP(l)$ & the probability of occurence of scenario $l$\\ \midrule
%			\multicolumn{2}{l}{\textbf{Decision variables}} \\
%			$y_{it}$ ($1^{\textrm{st}}$ stage)  & number of units of size $i$ produced at time $t$ \\
%			$x_{ijt}^l$ ($2^{\textrm{nd}}$ stage) & number of units of size $i$ cut to meet demand for smaller size $j$ at time $t$ under scenario $l$\\ 
%			$z_{it}^l$ ($2^{\textrm{nd}}$ stage)& 1 if we produce size $i$ at time $t$ under scenario $l$, 0 otherwise\\
%			\bottomrule
%		\end{tabular}
%	}
%\end{table} 

\subsubsection{\texttt{SIZES}: Data generation}
\textcolor{red}{
	Instances of \texttt{SIZES} are generated based on the one-period data given in Table \ref{sizes:data}. Note that although the table includes sleeve length data, we do not use this information for \texttt{SIZES}. Following \cite{journal:JSW1999}, we set the stochstic parameter $c_t^l=200,000$ to be deterministic for all $t\in T$ and $l\in\mathcal{L}$, hence only the demand parameter ($d_i^l$) is stochastic throughout the scenarios. The stochastic demand data in generated based on Table \ref{sizes:data}. First, we decide the number of scenarios to be generated by $n_\mathcal{L}=|\mathcal{L}|$. Then, the demand data is specified by a vector of multipliers: one multiplier for each scenario that is multiplied times the demand vector from Table \ref{sizes:data}. For example, if $n_\mathcal{L}=3$, the instance is defined by $(0.7,1,1.3)$. Or if $n_\mathcal{L}=5$, the instance is defined by $(0.6,0.8,1,1.2,1.4)$. Since \texttt{SIPLIB} provides instances with $n_\mathcal{L}\le 20$, we recommend the users to use \texttt{SIPLIB 2.0} to only generate instances with $n_\mathcal{L} \ge 20$. In \texttt{SIPLIB 2.0}, the multiplier vector is defined by the equally split set of subintervals between $[0.5,1.5]$, e.g., when $n_\mathcal{L}=20$, the multiplier vector is $(0.5,0.55,0.6,\ldots,1.4,1.45,1.5)$. With larger value of $n_\mathcal{L}$, we will have vector with finer granularity. After that, the instance is named by \texttt{SIZES\_$n_\mathcal{L}$}.
}
\begin{table}[]
	\centering
	\caption{Base data for \texttt{SIZES} scenarios \cite{journal:JSW1999}}
	\label{sizes:data}
	\begin{tabular}{cccc}
		\hline
		$i$  & sleeve length & unit production cost ($p_i$) & demand ($d_i$) \\ \hline
		1  & 25            & 0.748                & 2500   \\
		2  & 30            & 0.7584               & 7500   \\
		3  & 35            & 0.7688               & 12500  \\
		4  & 40            & 0.7792               & 10000  \\
		5  & 45            & 0.7896               & 35000  \\
		6  & 50            & 0.8                  & 25000  \\
		7  & 55            & 0.8014               & 15000  \\
		8  & 60            & 0.8208               & 12500  \\
		9  & 65            & 0.8312               & 12500  \\
		10 & 70            & 0.8416               & 5000   \\ \hline
		\multicolumn{4}{c}{unit cutting cost ($u$): \$0.008}     \\
		\multicolumn{4}{c}{setup cost ($s$): \$453}              \\ 
		\multicolumn{4}{c}{production capacity ($c_t$): 200,000} \\ \hline
	\end{tabular}
\end{table}


\subsection{\texttt{SMKP}: Stochastic multiple knapsack problem}
\texttt{SMKP} is a class of stochastic multiple binary knapsack problems. Unlike typical knapsack problems where the objective is to maximize total profits under the restriction of the weight capacity of each knapsack, \texttt{SMKP} is to minimize total weights while satisfying a certain required profit for each knapsack. 

\texttt{SIPLIB} provides 30 instances of \texttt{SMKP} in total. The first-stage problems contain 240 binary variables and 50 knapsack constraints. The second-stage problems have 120 binary variables and 5 knapsack constraints. Each instance has 20 scenarios. 
We refer \cite{journal:AAD2014} for writing \texttt{Julia} script for \texttt{SIPLIB 2.0} and explain the model throughout the following subsections.
\subsubsection{\texttt{SMKP}: Mathematical formulation}
We have three types of items $x$, $z$, and $y$ where the first two types are of the first-stage and the last one is of the second-stage with stochastic scenarios. For each type, we have $|I|$ number of items where $I$ is the index set of the items. Hence, we define the binary variables $x_i$, $z_i$, and $y_i^s$ which are equal to 1 if the $i^{\mathrm{th}}$ item is decided to be included ($s$ denotes scenario so only appears in $y$-type variables). We consider two types of knapsacks: one associated with $x$-type and $z$-type items (say \texttt{xz}-knapsack) and the other one with $x$-type and $y$-type items (say \texttt{xy}-knapsack). \texttt{xz}-knapsacks are indexed by $j\in J$ and \texttt{xy}-knapsacks are indexed by $k\in K$.  Each knapsack has its own minimum level of profit that should be satisfied by the items of the associated types, e.g., the profit of the $j^{\mathrm{th}}$ \texttt{xz}-type knapsack is calculated based on the inclusion or exclusion of $x$-type and $z$-type items and should satisfy a certain requirement $b_j$. Bear in mind that the inclusion or exclusion of a certain item $i$ affects all the associated knapsacks.
 
Each parameter $c_i$, $d_i$, and $q_i^s$ denotes the gain of weight when including items of type $x$, $z$, and $y$, respectively. Here, $c_i$ and $d_i$ are deterministic and $q_i^s$ is stochastic. Parameters $a_{ji}$, $e_{ji}$, $t_{ki}$, and $w_{ki}$ are all deterministic and denote the profits for including items in the knapsacks. The RHS parameters $b_j$ and $h_k$ are the minimum levels of profit requirements for \texttt{xz}-knapsacks and \texttt{xy}-knapsacks, respectively.

The extensive form of \texttt{SMKP} is as follows and the notations used are summrized in Table \ref{smkp:notation}.
\begin{subequations}
	\begin{align}
	(\texttt{SMKP})\ \textrm{min}\ &\sum_{i \in I}\left(c_i x_i + d_i z_i\right) + \sum_{s\in\mathcal{S}}\PP(s)\sum_{i\in I}q_i^s y_i^s \label{smkp:obj}\\
	\textrm{s.t.}\ &  \sum_{i\in I}a_{ji}x_{i} + \sum_{i \in I}e_{ji}z_i\ge b_j,\quad\forall j\in J, \label{smkp:b}\\
	&  \sum_{i\in I} t_{ki}x_i + \sum_{i\in I}w_{ki} y_i^s\ge h_k,\quad\forall k\in K,\ \forall s\in\mathcal{S}, \label{smkp:c}\\
	&  x_i\in\{0,1\},\quad\forall i\in I, \label{smkp:d}\\
	&  z_i\in\{0,1\},\quad\forall i\in I, \label{smkp:e}\\
	&  y_i^s\in\{0,1\},\quad\forall i\in I,\ \forall s\in \mathcal{S}. \label{smkp:f}
	\end{align}
\end{subequations}

The objective (\ref{smkp:obj}) is to minimize the expected value of the total weights. Constraint (\ref{smkp:b}) ensures the minimum levels of profit requirements for all \texttt{xz}-knapsacks are satisfied. Constraint (\ref{smkp:c}) guarantees the minimum levels of profit requirements are satisfied for all \texttt{xy}-knapsacks under every scenario. Constraints (\ref{smkp:d})-(\ref{smkp:f}) are binary restriction of the decision variables.

\begin{table}[H]
	\caption{Notations for \texttt{SMKP}}
	\label{smkp:notation}
	\resizebox{\textwidth}{!}
	{
		\begin{tabular}{ll}
			\toprule
			\multicolumn{2}{l}{\textbf{Index sets:}} \\
			$I$ &  index set of items for each type ($i\in I$)\\ 
			$J$ &  index set of \texttt{xz}-knapsacks ($j \in J$)\\ 
			$K$ &  index set of \texttt{xy}-knapsacks ($k \in K$)\\
			$\mathcal{S}$ & index set of scenarios ($s\in \mathcal{S}$)\\ \midrule
			\multicolumn{2}{l}{\textbf{Parameters:}} \\
			$c_{i}$ 	& weight of the $i^{\mathrm{th}}$ $x$-type item 		\\ 
			$d_{i}$ 	& weight of the $i^{\mathrm{th}}$ $z$-type item  		\\ 
			$q_{i}^{s}$ & weight of the $i^{\mathrm{th}}$ $y$-type item under scenario $s$ 		\\ 
			$a_{ji}$	& profit of the $j^{\mathrm{th}}$ \texttt{xz}-knapsack for including $i^{\mathrm{th}}$ $x$-type item		\\
			$e_{ji}$	& profit of the $j^{\mathrm{th}}$ \texttt{xz}-knapsack for including $i^{\mathrm{th}}$ $z$-type item		\\
			$t_{ki}$	& profit of the $k^{\mathrm{th}}$ \texttt{xy}-knapsack for including $i^{\mathrm{th}}$ $x$-type item	\\
			$w_{ki}$	& profit of the $k^{\mathrm{th}}$ \texttt{xy}-knapsack for including $i^{\mathrm{th}}$ $y$-type item		\\
			$b_j$		& minimum required profit for the $j^{\mathrm{th}}$ \texttt{xz}-knapsack		\\
			$h_k$		& minimum required profit for the $k^{\mathrm{th}}$ \texttt{xy}-knapsack		\\
			$\PP(s)$ 	& \textrm{the probability of occurence of scenario $s$} \\ \midrule
			\multicolumn{2}{l}{\textbf{Decision variables:}} \\
			$x_{i}$ ($1^{\textrm{st}}$ stage) & 1 if the $i^{\mathrm{th}}$ $x$-type item is decided to be included, 0 otherwise \\ 
			$z_{i}$ ($1^{\textrm{st}}$ stage) & 1 if the $i^{\mathrm{th}}$ $z$-type item is decided to be included, 0 otherwise \\ 
			$y_{i}^s$ ($2^{\textrm{nd}}$ stage)& 1 if the $i^{\mathrm{th}}$ $y$-type item is decided to be included under scenario $s$, 0 otherwise  \\
			\bottomrule
		\end{tabular}
	}
\end{table} 

\subsubsection{\texttt{SMKP}: Data generation}
There are two factors that define the instance of \texttt{SMKP}: $|I|$ and $|\mathcal{S}|$. The sizes for another sets are fixed by $|J|=50$ and $|K|=5$ following \cite{journal:AAD2014}. Once we decide the factors by $n_I=|I|$, $n_S=|\mathcal{S}|$, $n_T=|T|$, and $n_\mathcal{S}=|\mathcal{S}|$, each instance is named by \texttt{SMKP\_$n_I$\_$n_\mathcal{S}$}. Again directly following \cite{journal:AAD2014}, we randomly generate the parameters. Let $U$ be a discrete uniform random variable: $U\sim Unif[1,100]$. Then, the parameters are generated as follows:
\begin{align*}
c_i		&=	U,\quad\forall i\in I, \\
d_i		&=	U,\quad\forall i\in I, \\
q_i^s	&= 	U,\quad\forall i\in I,\ \forall s\in\mathcal{S},\\
a_{ji}	&=	U,\quad\forall j\in J,\ \forall i\in I, \\
e_{ji}	&=	U,\quad\forall j\in J,\ \forall i\in I, \\
t_{ki}	&=	U,\quad\forall k\in K,\ \forall i\in I, \\
w_{ki}	&=	U,\quad\forall k\in K,\ \forall i\in I,  \\
b_j		&=	\frac{3}{4}\sum_{i\in I}\left(a_{ji}+e_{ji}\right),\quad\forall j\in J, \\
h_k		&=	\frac{3}{4}\sum_{i\in I}\left(t_{ji}+w_{ji}\right),\quad\forall k\in K. 
\end{align*}
\textcolor{red}{For the first-stage parameters, we fix the random seed to make it deterministic.}

\subsection{\texttt{SSLP}: Stochastic server location problem}
\texttt{SSLP} is a class of problem that finds the optimal location of servers and the optimal allocation of clients to servers which maximizes the expected net income under uncertain presents of clients. \texttt{SSLP} finds applications in a variety of domains such as network design for electric power, internet services, telecommunications, and water distribution. \texttt{SIPLIB} provides 12 instances with varying number of clients, server locations, and scenarios in \texttt{SMPS} format. The largest instance includes 10 server locations, 50 clients, and 2,000 scenarios which corresponds to 120,001 constraints, 1,000,010 binary variables, and 20,000 continuous variables.

We refer to \cite{journal:NS2005} for mathematical formulation and data generation forthcoming through the following subsections.


\subsubsection{\texttt{SSLP}: Mathematical formulation}
Let $I$, $J$, $Z$, and $\mathcal{S}$ be index sets for the clients, servers, zones, and scenarios. For $i\in I$, $j\in J$, $z\in Z$, and $s\in\mathcal{S}$, we define the notations in Table \ref{sslp:notation}.

Suppose that we place a server at location $j$. Then, the allocation costs $c_j$ and the server will provide capacity to serve up to $u$ amount of resource to clients. The revenue earned by serving client $i$ from location $j$ is denoted by $q_{ij}$. We have also a shortage cost (penalty) $q_{0j}$ for each unit of demand that remains unserved among the clients assigned to server $j$. If client $i$ is served by a server at location $j$, it uses $d_{ij}$ units of resource from the server. We allow only one server to be installed at each location and each client can only be served by one server. There is a requirement that a minimum number of servers to be located in a zone $z$, and is denoted by $w_z$. 

The first-stage binary variables $x_j$ decide whether or not a server is located at location $j$. The second-stage binary variables $y_{ij}^s$ are referred to as recourse decision under scenario $s$ and associated with the decision on serving client $i$ by server $j$. The variables $y_{ij}^s$ will be implemented in the future, when scenario $s$ is finally observed.

Based on the above, the extensive form of \texttt{SSLP} can be stated as follows:
\begin{subequations}
	\begin{align}
	(\texttt{SSLP})\ \textrm{min}\ &	\sum_{j\in J}c_j x_j - \sum_{s\in\mathcal{S}}\PP(s)\left(\sum_{i\in I}\sum_{j\in J}q_{ij}^s y_{ij}^s - \sum_{j\in J}q_{0j}^s y_{0j}^s \right) \label{sslp:obj} \\ 
	\textrm{s.t.}\ &\sum_{j\in J}x_j\le v,\label{sslp:b}\\ 
	&\textcolor{red}{\sum_{j\in J_z}x_j\ge w_z,\quad\forall z\in Z,}\label{sslp:c}\\
	&\sum_{i\in I}d_{ij}y_{ij}^s - y_{0j}^s\le ux_j,\quad\forall j\in J,\ \stkout{\textcolor{red}{\forall i\in I,}}\ \forall s\in\mathcal{S}, \label{sslp:d} \\
	&\sum_{j\in J}y_{ij}^s=h_i^s,\quad\forall i\in I,\ \forall s\in \mathcal{S}, \label{sslp:e}\\
	&x_j\in\{0,1\},\quad\forall j\in J,\label{sslp:f}\\
	&y_{ij}^s\in\{0,1\},\quad\forall i\in I,\ j\in J,\ s\in\mathcal{S}, \label{sslp:g}\\
	&y_{0j}^s\ge 0,\quad\forall j\in J,\ \forall s\in\mathcal{S} \label{sslp:h}.
	\end{align}
\end{subequations}
The objective function (\ref{sslp:obj}) is to maximize total expected revenue of locating servers and serving customers by the servers. Constraint (\ref{sslp:b}) satisfies the requirement that only up to a total of $v$ available servers can be installed. The zonal requirements that specify how many servers are needed in each zone are given by constraint (\ref{sslp:c}). Constraint (\ref{sslp:d}) ensures that a server located at site $j$ can serve only up to its capacity $u$. The variable $y_{0j}^s$ is introduced in the constraint (\ref{sslp:d}) to accomodate any overflows that are not served due to limitations in server capacity. These overflows result in a loss of revenue at a rate of $q_{0j}^s$. The inclusion of an artificial variable may allow a client to be assigned to servers that are not located. However, penalty costs associated with such an assignment may result in such high costs as to preclude it in an optimal solution, unless server capacity is so limited that some clients have to be turned away \cite{journal:NS2005}. Constraint (\ref{sslp:e}) guarantees that each client is served by only one server. Constraint (\ref{sslp:f}) and (\ref{sslp:g}) are binary restrictions on the decision variables. Finally, constraint (\ref{sslp:h}) is the non-negativity requirement on the overflow variables.
\begin{table}[H]
	\caption{Notations for \texttt{SSLP}}
	\label{sslp:notation}
	\resizebox{\textwidth}{!}
	{
		\begin{tabular}{ll}
			\toprule
			\multicolumn{2}{l}{\textbf{Index sets:}} \\
			$J$ 		  & index set of server locations ($j\in J$)\\ 
			$I$ 		  & index set of clients ($i\in I$)  \\ 
			$Z$ 		  & index set of zones ($z\in Z$) \\
			$\mathcal{S}$ & index set of scenarios ($s\in \mathcal{S}$)	\\ \midrule
			\multicolumn{2}{l}{\textbf{Parameters:}} \\
			$c_j$		& cost of locating a server at location $j$	\\
			$q_{ij}^s$	& revenue from client $i$ being served by server at location $j$ under scenario $s$	\\
			$q_{0j}^s$	& rate of revenue loss for overflows that are not served due to limited server capacity under scenario $s$	\\
			$d_{ij}$	& resource demand of client $i$ from server at location $j$	\\
			$u$			& server capacity	\\
			$v$			& upper bound on the total number of servers that can be located	\\
			$w_z$		& minimum number of servers to be located in zone $z$	\\
			$J_z$		& subset of server locations that belong to zone $z$	\\
			$h_i^s$		& 1 if client $i$ is present under scenario $s$, 0 otherwise	\\
			$\PP(s)$ 	& probability of occurence for scenario $s$\\ \midrule
			\multicolumn{2}{l}{\textbf{Decision variables:}} \\
			$x_j$ ($1^{\textrm{st}}$ stage)  	 & 1 if a server is located at site $j$, 0 otherwise \\
			$y_{ij}^s$ ($2^{\textrm{nd}}$ stage) & 1 if client $i$ is served by a server at location $j$ under scenario $s$, 0 otherwise\\
			$y_{0j}^s$ ($2^{\textrm{nd}}$ stage) & non-negative amount of overflows that are not served due to limitations in server $j$'s capacity	\\
			\bottomrule
		\end{tabular}
	}
\end{table} 

\subsubsection{\texttt{SSLP}: Data generation}
For each instance of \texttt{SSLP}, we determine the number of potential server locations $n_J=|J|$, the number of clients $n_I=|I|$, and the number of scenarios $n_\mathcal{S}=|\mathcal{S}|$. Then, the instance is named by \texttt{SSLP\_$n_J$\_$n_I$\_$n_\mathcal{S}$}. The client-server revenue are set to be 1 per unit of client demand. Some of deterministic parameters are randomly generated from the discrete uniform distribution while scenario data are generated from the Bernoulli distribution. In summary, the parameters are generated as follows:
\begin{align*}
c_j	&=Unif[40,80],\quad\forall j\in J,\\
q_{ij}^s	&= d_{ij},\quad\forall i\in I,\ \forall j\in J,\ \forall s\in\mathcal{S},\\
q_{0j}^s	&=	1000,\quad\forall j\in J,\ \forall s\in\mathcal{S},\\
d_{ij}	&= Unif[0,25],\quad\forall i\in I,\ \forall j\in J,\\
h_i^s	&= Bernoulli(0.5),\quad\forall i\in I,\ \forall s\in \mathcal{S},\\
v 		&= |J|	\\
\textcolor{red}{u	}&\textcolor{red}{=		}\\
\textcolor{red}{w_z	}&\textcolor{red}{=		}\\
\textcolor{red}{J_z	}&\textcolor{red}{=	}
\end{align*}



\subsection{\texttt{SUCW}: Stochastic unit commitment problem with wind power}
The unit commitment (UC) problem is a production cost model (PCM) that plans power system operations over an exteded time horizon. \texttt{SUCW} is a stochastic version of UC for studying the impact of incorporating highly uncertain power generation of large-scale wind turbines with transmission constraints and system component failures. We refer to \cite{journal:PO2013} and \cite{journal:KZ2015} for mathematical models.
\subsubsection{\texttt{SUCW}: Mathematical formulation}
A two-stage stochastic unit commitment model \texttt{SUCW} is presented, where we make decision on slow generators in the first stage and make the commitment decisions for fast generators and the power dispatch decision in the second stage. In \texttt{SUCW}, we also incorporate ramping constraints, reserve constraints and transmission line capacity constraints. We assume the piecewise linear convex cost function for the power generation.
\begin{subequations}
	\begin{align}
	\min \ & \sum_{\sigma \in \mathcal{S}} \PP(\sigma) \sum_{t\in T} \sum_{g\in G}\left( C^\text{fx}_g x_{gt}^\sigma + C^\text{up}_g u_{gt}^\sigma  + C^\text{dn}_g v_{gt}^\sigma  + \sum_{k\in K} C^\text{mar}_{gk} q_{gkt}^\sigma\right) \label{sucw:obj}\\
	\text{s.t.} \
	%% Constraints for commitment, start-up, and shut-down
	& 1 - x_{g(t-1)}^\sigma \geq u_{gt}^\sigma , \quad \forall \sigma \in \mathcal{S},\ g\in G,\ t\in T, \label{sucw:b} \\
	& x_{g(t-1)}^\sigma \geq v_{gt}^\sigma , \quad \forall \sigma \in \mathcal{S},\ g\in G,\ t\in T, \label{sucw:c} \\
	& x_{gt}^\sigma - x_{g(t-1)}^\sigma = u_{gt}^\sigma  - v_{gt}^\sigma , \quad \forall \sigma \in \mathcal{S},\ g\in G,\ t\in T, \label{sucw:d} \\
	%% Constraints for minimum down/up time
	& x_{gt}^\sigma \geq \sum_{\tau=\max\{1,t-UT_g+1\}}^t u_{g\tau}^\sigma, \quad \forall \sigma \in \mathcal{S},\ g\in G,\ t\in T, \label{sucw:e} \\
	& 1 - x_{gt}^\sigma \geq \sum_{\tau=\max\{1,t-DT_g+1\}}^t u_{g\tau}^\sigma,\quad \forall \sigma \in \mathcal{S},\ g\in G,\ t\in T, \label{sucw:f} \\
	%% Ramping constraint
	& -RD_g \leq p_{gt}^\sigma - p_{g(t-1)}^\sigma \leq RU_g - s_{gt}^\sigma, \quad \forall \sigma \in \mathcal{S},\ g\in G,\ t\in T, \label{sucw:g} \\
	& s_{gt}^\sigma \leq RC_g x_{gt}^\sigma, \quad \forall \sigma \in \mathcal{S},\ g\in G,\ t\in T, \label{sucw:h} \\
	%% Spinning reserve constraint
	& \sum_{g\in G} s_{gt}^\sigma \geq SR_t, \quad \forall \sigma \in \mathcal{S},\ t\in T,\ \label{sucw:i} \\
	%% Minimum/maximum power generation level
	& p_{gt}^\sigma = P^\text{min}_g x_{gt}^\sigma + \sum_{k\in K} q_{gkt}^\sigma, \quad \forall \sigma \in \mathcal{S},\ g\in G, t\in T,\ \label{sucw:j} \\
	& p_{gt}^\sigma + s_{gt}^\sigma \leq P^\text{max}_g x_{gt}^\sigma, \quad \forall \sigma \in \mathcal{S},\ g\in G,\ t\in T, \label{sucw:k} \\
	%% Marginal power generation cost
	& q_{gkt}^\sigma \leq Q^\text{max}_{gk} x_{gt}^\sigma, \quad \forall \sigma \in \mathcal{S},\ g\in G,\ k\in K,\ t\in T, \label{sucw:l} \\
	%% System balance
	& \sum_{g\in G} p_{gt}^\sigma = \sum_{n\in N} D_{nt}^\sigma - \sum_{w\in W} W_{wt}^\sigma, \quad \forall \sigma \in \mathcal{S},\ t\in T,\ \label{sucw:m} \\
	%% maximum power flow on each transmission lines
	& -F^\text{max}_l \leq \sum_{g\in G} LSF_{lg} p_{gt}^\sigma - \sum_{n\in N} LSF_{ln} D_{nt}^\sigma \notag	\\
	&\quad\quad\quad\quad\quad + \sum_{w\in W} LSF_{lw} W_{wt}^\sigma \leq F^\text{max}_l,\quad \forall \sigma \in \mathcal{S},\ l\in L,\ t\in T, \label{sucw:n} \\
	%% Non-anticipativity constraint
	& x_{gt}^{\sigma_1} = x_{gt}^{\sigma_2},\ u_{gt}^{\sigma_1} = u_{gt}^{\sigma_2},\ v_{gt}^{\sigma_1} = v_{gt}^{\sigma_2},\quad \forall \sigma_1,\sigma_2 \in \mathcal{S},\ g\in G_s,\ t\in T, \label{sucw:o} \\
	%% Initial conditions
	& x_{g0}^\sigma = X^\text{init}_g, \quad \forall \sigma \in \mathcal{S},\ g\in G, \label{sucw:p} \\
	& x_{gt}^\sigma = 1, \quad \forall \sigma \in \mathcal{S},\ g\in G,\ t\in \{1, \dots, UT^\text{init}_g\}, \label{sucw:q} \\
	& x_{gt}^\sigma = 0, \quad \forall \sigma \in \mathcal{S},\ g\in G,\ t\in \{1, \dots, DT^\text{init}_g\}, \label{sucw:r} \\
	& p_{g0}^\sigma = P^\text{init}_g, \quad \forall \sigma \in \mathcal{S},\ g\in G,\ \label{sucw:s} \\
	%% bound constraints
	%& x_{gt}^\sigma \in \{0,1\},\quad 0\leq u_{gt}^\sigma , v_{gt}^\sigma  \leq 1, \quad \forall \sigma \in \mathcal{S},\ g\in G,\ t\in T, \label{eq:suc:cons19} \\
	& \textcolor{red}{x_{gt}^\sigma , u_{gt}^\sigma , v_{gt}^\sigma \in \{0,1\},\quad \forall \sigma \in \mathcal{S},\ g\in G,\ t\in T, }\label{sucw:t} \\
	& p_{gt}^\sigma , q_{gkt}^\sigma,s_{gt}^\sigma \geq 0, \quad \forall \sigma \in \mathcal{S},\ g\in G,\ t\in T. \label{sucw:u} 
	\end{align}
\end{subequations}
The objective function (\ref{sucw:obj}) is to minimize the expected value of the sum of operating, start-up, shut-down, and production cost. Constraints (\ref{sucw:b})-(\ref{sucw:d}) are for logical relationships amongst the commitment, start-up, and shut-down decisions. Constraints (\ref{sucw:e}) and (\ref{sucw:f}) represent the minimum downtime and uptime of generators in each time period. Constraints (\ref{sucw:g}) and (\ref{sucw:h}) are ramping restrictions, and (\ref{sucw:i}) is a spinning reserve constraint. Constraints (\ref{sucw:j}) and (\ref{sucw:k}) ensure the amount of power generation is restricted by its minimum and maximum limits. Constraint (\ref{sucw:l}) represents the piecewise linearized power generation cost. Constraint (\ref{sucw:m}) guarantees the balanced flow. Constraint (\ref{sucw:n}) is for the transmission line flow. Constraint (\ref{sucw:o}) is a \textit{nonanticipativity} constraint that enforces the decisions do not change over scenarios. Constraints (\ref{sucw:p})-(\ref{sucw:s}) represent the initial conditions of generators and production level. Finally, constraints (\ref{sucw:t}) and (\ref{sucw:u}) restrict the space from which the decision variables can take values by binary or non-negative continuous. 


\begin{table}[H]
	\centering
	\caption{Notations for the \texttt{SUCW}}
	\begin{tabular}{ll}
		\toprule
		\multicolumn{2}{l}{\textbf{Index sets:}} \\
		$G$ & index set of all generators ($g\in G$)\\
		$G_s$ & index set of slow generators ($g\in G_s$)\\
		$G_f$ & index set of fast generators ($g\in G_f$)\\
		$K$ & index set of linear segments of the piece-wise linear power generation cost ($k\in K$)\\
		$L$ & index set of transmission lines ($l\in L$) \\
		$N$ & index set of buses ($n\in N$)\\
		$T$ & index set of time periods ($t\in T$)\\
		$W$ & index set of wind power generators ($w\in W$)\\
		$\mathcal{S}$ & index set of scenarios ($\sigma\in\mathcal{S}$) 		\\ \midrule
		\multicolumn{2}{l}{\textbf{Parameters:}} \\
		$C^\text{up}_g$ & start-up cost of generator $g$ \\
		$C^\text{dn}_g$ & shut-down cost of generator $g$ \\
		$C^\text{fx}_g$ & fixed cost of operating the generator $g$ \\
		$C^\text{mar}_{gk}$ & $k^\textrm{th}$ marginal cost of production of generator $g$ \\
		$X^\text{init}_g$ & initial on/off status of generator $g$ \\
		$UT^\text{init}_g$ & initial minimum uptime of generator $g$ \\
		$UT_g$ & minimum uptime of generator $g$ \\
		$DT^\text{init}_g$ & initial minimum downtime of generator $g$ \\
		$DT_g$ & minimum downtime of generator $g$ \\
		$RU_g$ & ramp-up limit of generator $g$ \\
		$RD_g$ & ramp-down limit of generator $g$ \\
		$RC_g$ & ramping capacity of generator $g$ \\
		$P^\text{init}_g$ & initial power output of generator $g$ \\
		$P^\text{min}_g$ & minimum power output of generator $g$ \\
		$P^\text{max}_g$ & maximum power output of generator $g$ \\
		$Q^\text{max}_{gk}$ & maximum power output of generator $g$ with the $k^\textrm{th}$ marginal cost \\
		$SR_t$ & spinning reserve required at time $t$ \\
		$F^\text{max}_l$ & maximum power flow of transmission line $l$ \\
		$LSF_{ln}$ & load-shift factor of transmission line $l$ with respect to bus $n$ \\
		$\PP(\sigma)$ & probability of scenario $\sigma$ \\
		$D_{nt}^\sigma$ & demand load at bus $n$ at time $t$ in scenario $\sigma$ \\
		$W_{wt}^\sigma$ & wind power generation from generator $w$ at time $t$ in scenario $\sigma$ \\ \midrule
		\multicolumn{2}{l}{\textbf{Decision variables:}} \\
		$x_{gt}^\sigma$ & on/off indicator of generator $g$ at time $t$ in scenario $\sigma$\\
		$u_{gt}^\sigma $ & start-up indicator of generator $g$ at time $t$ in scenario $\sigma$\\
		$v_{gt}^\sigma $ & shut-down indicator of generator $g$ at time $t$ in scenario $\sigma$ \\
		$p_{gt}^\sigma$ & power output of generator $g$ at time $t$ in scenario $\sigma$ \\
		$q_{gkt}^\sigma$ & power output of generator $g$ at time $t$ with the $k^\textrm{th}$ marginal cost in scenario $\sigma$\\
		\textcolor{red}{$s_{gt}^\sigma$} & \textcolor{red}{spinning reserve of generator $g$ at time $t$ in scenario $\sigma$} \\
		\hline
	\end{tabular}
\end{table}


\subsubsection{\texttt{SUCW}: Data generation}
For \texttt{SUCW} instances, thermal power generators are scheduled over a day. The schedules are subjected to uncertainty in wind power. We use a modified IEEE 188-bus system with 54 generators, 118 buses, and 186 transmission lines provided in \cite{journal:LLMS2014}. 17 of the 54 generators are assumed to be allowed to start on demand (second-stage) while the other generators should be scheduled in advance (first-stage). We consider 3 identical wind farms. Each of them consts of  120 wind turbines. We used real wind speed data predicted from the observations of 31 weather stations in Illinois.

\textcolor{red}{Amongst the parameters, only the wind power generation data $W_{wt}^\sigma$ is stochastic subjected to be generated randomly.}

\input{sec00_references} % This will be replaces by a bibtex format file later.

\end{document}
% end of file template.tex

