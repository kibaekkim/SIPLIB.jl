%%%%%%%%%%%%%%%%%%%%%%% file template.tex %%%%%%%%%%%%%%%%%%%%%%%%%
%
% This is a general template file for the LaTeX package SVJour3
% for Springer journals.          Springer Heidelberg 2010/09/16
%
% Copy it to a new file with a new name and use it as the basis
% for your article. Delete % signs as needed.
%
% This template includes a few options for different layouts and
% content for various journals. Please consult a previous issue of
% your journal as needed.
%
%%%%%%%%%%%%%%%%%%%%%%%%%%%%%%%%%%%%%%%%%%%%%%%%%%%%%%%%%%%%%%%%%%%
%
% First comes an example EPS file -- just ignore it and
% proceed on the \documentclass line
% your LaTeX will extract the file if required
\begin{filecontents*}{example.eps}
%!PS-Adobe-3.0 EPSF-3.0
%%BoundingBox: 19 19 221 221
%%CreationDate: Mon Sep 29 1997
%%Creator: programmed by hand (JK)
%%EndComments
gsave
newpath
  20 20 moveto
  20 220 lineto
  220 220 lineto
  220 20 lineto
closepath
2 setlinewidth
gsave
  .4 setgray fill
grestore
stroke
grestore
\end{filecontents*}
%
\RequirePackage{fix-cm}
%
%\documentclass{svjour3}                     % onecolumn (standard format)
%\documentclass[smallcondensed]{svjour3}     % onecolumn (ditto)
\documentclass[smallextended]{svjour3}       % onecolumn (second format)
%\documentclass[twocolumn]{svjour3}          % twocolumn
%
\smartqed  % flush right qed marks, e.g. at end of proof
%
\usepackage{graphicx}
\usepackage{hyperref}
\usepackage{float}
\usepackage{mathrsfs,amsmath,amssymb,amscd,mathtools}
%
% \usepackage{mathptmx}      % use Times fonts if available on your TeX system
%
% insert here the call for the packages your document requires
%\usepackage{latexsym}
% etc.
%
% please place your own definitions here and don't use \def but
% \newcommand{}{}
%
% Insert the name of "your journal" with
% \journalname{myjournal}
%
\begin{document}

\title{SIPLIB 2.0
%\thanks{Grants or other notes
%about the article that should go on the front page should be
%placed here. General acknowledgments should be placed at the end of the article.}
}
\subtitle{Stochastic Integer Programming Library version 2.0}

\titlerunning{SIPLIB 2.0}        % if too long for running head

\author{Yongkyu Cho \and 
		Kibaek Kim \and
        Cong Han Lim \and
        James Luedtke \and
        Jeffrey Linderoth
}

%\authorrunning{Short form of author list} % if too long for running head

\institute{Yongkyu Cho, Kibaek Kim \at
           Mathematics and Computer Science Division, Argonne National Laboratory, Lemont, IL 60439, USA \\
           \email{choy@anl.gov; kimk@anl.gov}
           \and
           Cong Han Lim \and James Luedtke \and Jeffrey Linderoth \at
           Department of Industrial and Systems Engineering, University of Wisconsin-Madison Madison, WI 53706, USA \\
%          \and
%           Cong Han Lim \at
           \email{clim9@wisc.edu; jim.luedtke@wisc.edu; linderoth@wisc.edu}
%           \and 
%           James Luedtke \at
%           \email{jim.luedtke@wisc.edu}
%           \and 
%           Jeffrey Linderoth \at
%           \email{linderoth@wisc.edu}
}

\date{Received: date / Accepted: date}
% The correct dates will be entered by the editor


\maketitle

\begin{abstract}
We present a collection of stochastic integer programming problem instances.
\keywords{Stochastic Integer Programming \and Problem Instances}
% \PACS{PACS code1 \and PACS code2 \and more}
% \subclass{MSC code1 \and MSC code2 \and more}
\end{abstract}

\section{Introduction}
\begin{itemize}
	\item What SIP is?\\
	Stochastic integer programming (SIP) is ... The main difficulty in solving stochastic integer programs is that the second-stage value function is not necessarily convex, but only lower semicontinuous (l.s.c.). Thus, the standard decomposition approaches that work nicely for stochastic \textit{linear} programs, break down when second stage integer variables are present (Ahmed and Garcia, 2004). In this research study, we focus our emphasis on 2-stage SIP.
	\item SIPLIB?
	\begin{itemize}
		\item MIPLIBv5 (last modified 2017): http://miplib.zib.de/
		\item Shabbir's SIPLIB (last modified 2015): https://www2.isye.gatech.edu/~sahmed/siplib/
		\item Felt et al's SLPlib (last modified 2001): https://www4.uwsp.edu/math/afelt/slpinput/download.html
		\item Holmes's POSTS (the most recent reference 1994): http://users.iems.northwestern.edu/~jrbirge/html/dholmes/post.html
	\end{itemize}
	\item Motivation for SIPLIBv2\\
	We need more..
	\item Power of Julia language for large-scale optimization problems
	\item Contribution\\
	By SIPLIB 2.0, we mainly provide 1) richer collection of test problems for computational and algorithmic research in 2-stage SIP with benchmark experimental results, 2) not only SMPS files but also \textit{Julia} files formatted in \textit{StructJuMP} that are easily readable/modifiable.
\end{itemize}


\section{Stochastic Integer Programming}
The form of SIP is varying 
\subsection{Formulation}
\subsubsection{2-Stage Recourse Programs}


\subsection{Algorithms}
\subsubsection{Stage-wise Decomposition Algorithm}
\subsubsection{Scenario-wise Decompostion Algorithm}
Benders, dual, ...

\subsection{Software Libraries}
\subsubsection{Modeling Languages}
\subsubsection{Solvers}

\section{Test Sets Description}

We introduce the set of problem instances. The instances are available in SMPS and Julia (StructJuMP) file format. characteristics, categorization
\subsection{Mutli-Path Traveling Salesman Problem with Stochastic Travel Times (MPTSPS)}
\subsubsection{Problem Class}
\begin{table}[H]
	\centering
	%\caption{My caption}
	\label{mptsps-class}
	\begin{tabular}{|c|c|c|}
		\hline
		& 1$^{st}$ stage & 2$^{nd}$ stage \\ \hline
		Variables   & Bin & Bin               \\ \hline
		Constraints &                &                \\ \hline
	\end{tabular}
\end{table}
\subsubsection{Notation}
\begin{table}[H]
	\caption{Notations corresponding to problem}
	\label{notation}
	\resizebox{\textwidth}{!}
	{
		\begin{tabular}{ll}
			\hline 
			\textbf{Index sets} &  \\ 
			$T$ & \textrm{index set of time slots $(t=1,\ldots,|T|)$} \\ 
			$A$ & \textrm{index set of applications $(i=1,\ldots,|A|)$} \\ 
			$S$ & \textrm{index set of servers $(j=1,\ldots,|S|)$}\\
			$F_j$ & \textrm{index set of frequency options of server $j\in S$ $(f=0,\ldots,|F_j|)$} \\ 
			\textbf{Parameters} &   \\ 
			$\lambda_{it}$ & \textrm{average workload of application $i\in A$ that arrives in time interval $t\in T$} \\ 
			$U_j$  & \textrm{maximum number of applications installable to server $j\in S$} \\ 
			$C_{jf}$ & \textrm{processing capacity of server $j\in S$ under frequency $f\in F_j$} \\ 
			$\beta_{jft}$ & \textrm{cost incurred when server $j\in S$ runs at frequency $f\in F_j$ in time interval $t\in T$} \\ 
			$\rho$ & \textrm{target load for all servers (surrogate for quality of service)}\\ 
			\textbf{Decision variables} &  \\ 
			$a_{ij}$ (virtualization) & \textrm{1 if application $i\in A$ is installed to server $j\in S$, 0 otherwise} \\ 
			$x_{jft}$ (server provisioning) & \textrm{1 if server $j\in S$ runs at frequency $f\in F_j$ during time interval $t\in T$, 0 otherwise} \\ 
			$r_{ijt}$ (routing) & \textrm{fraction of workloads of application $i\in A$ assigned to server $j\in S$ in time  interval $t\in T$}\\
			\hline
		\end{tabular}
	}
\end{table} 

\subsubsection{Formulation}
\begin{align}
(\textrm{MIP}_O): \textrm{minimize    } &\sum_{j\in S}\sum_{f\in F_j}\sum_{t\in T}\beta_{jft}x_{jft},\\
\textrm{subject to} \nonumber\\ 
&\sum_{i\in A}a_{ij} \le U_j, \quad \forall j\in S,\\
&\sum_{j\in S}r_{ijt}=1, \quad \forall i\in A, \forall t\in T,\\
&r_{ijt} \le a_{ij}, \quad \forall i\in A, \forall j\in S, \forall t\in T, \\
&\sum_{i\in A}\lambda_{it}r_{ijt} \le \rho \sum_{f\in F_j}C_{jf}x_{jft}, \quad \forall j\in S, \forall t\in T,\\
&\sum_{f\in F_j}x_{jft}=1, \quad \forall j\in S, \forall t\in T, \\
&\sum_{i\in S}r_{ijt} \le U_j(1-x_{j0t}), \quad \forall j\in S, \forall t\in T, \\
&a_{ij}, x_{jft}\in \{0,1\}, \quad \forall j\in S, \forall f\in F_j, \forall t\in T  \\
&0 \le r_{ijt} \le 1, \quad \forall j\in S, \forall t\in T
\end{align}
\section{Implementation of SMPS Writer}

We describe our Julia implementation, how to model SIP and generate SMPS files..

\section{Solution Report}

\section{Concluding Remarks}

%\begin{acknowledgements}
%If you'd like to thank anyone, place your comments here
%and remove the percent signs.
%\end{acknowledgements}

% BibTeX users please use one of
\bibliographystyle{spbasic}      % basic style, author-year citations
%\bibliographystyle{spmpsci}      % mathematics and physical sciences
%\bibliographystyle{spphys}       % APS-like style for physics
%\bibliography{}   % name your BibTeX data base
\begin{thebibliography}{9} 
	\bibitem{example}
	Author1 and Author2, paper paper paper paper, Journal Title 68 (2011), 1207--1221.
\end{thebibliography}
\end{document}
% end of file template.tex

