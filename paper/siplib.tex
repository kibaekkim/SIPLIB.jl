%%%%%%%%%%%%%%%%%%%%%%% file template.tex %%%%%%%%%%%%%%%%%%%%%%%%%
%
% This is a general template file for the LaTeX package SVJour3
% for Springer journals.          Springer Heidelberg 2010/09/16
%
% Copy it to a new file with a new name and use it as the basis
% for your article. Delete % signs as needed.
%
% This template includes a few options for different layouts and
% content for various journals. Please consult a previous issue of
% your journal as needed.
%
%%%%%%%%%%%%%%%%%%%%%%%%%%%%%%%%%%%%%%%%%%%%%%%%%%%%%%%%%%%%%%%%%%%
%
% First comes an example EPS file -- just ignore it and
% proceed on the \documentclass line
% your LaTeX will extract the file if required
\begin{filecontents*}{example.eps}
%!PS-Adobe-3.0 EPSF-3.0
%%BoundingBox: 19 19 221 221
%%CreationDate: Mon Sep 29 1997
%%Creator: programmed by hand (JK)
%%EndComments
gsave
newpath
  20 20 moveto
  20 220 lineto
  220 220 lineto
  220 20 lineto
closepath
2 setlinewidth
gsave
  .4 setgray fill
grestore
stroke
grestore
\end{filecontents*}
%
\RequirePackage{fix-cm}
%
%\documentclass{svjour3}                     % onecolumn (standard format)
%\documentclass[smallcondensed]{svjour3}     % onecolumn (ditto)
\documentclass[smallextended]{svjour3}       % onecolumn (second format)
%\documentclass[twocolumn]{svjour3}          % twocolumn
%
\smartqed  % flush right qed marks, e.g. at end of proof
%
\usepackage{graphicx}
\usepackage{hyperref}
\usepackage{float}
\usepackage{mathrsfs,amsmath,amssymb,amscd,mathtools}
\usepackage{booktabs}
\usepackage{graphicx}
\usepackage{lscape}

\DeclareMathOperator*{\PP}{\mathbb{P}}
\DeclareMathOperator*{\EE}{\mathbb{E}}
\DeclarePairedDelimiter\ceil{\lceil}{\rceil}
\DeclarePairedDelimiter\floor{\lfloor}{\rfloor}
% \usepackage{mathptmx}      % use Times fonts if available on your TeX system
%
% insert here the call for the packages your document requires
%\usepackage{latexsym}
% etc.
%
% please place your own definitions here and don't use \def but
% \newcommand{}{}
%
% Insert the name of "your journal" with
% \journalname{myjournal}
%
\begin{document}

\title{SIPLIB 2.0
%\thanks{Grants or other notes
%about the article that should go on the front page should be
%placed here. General acknowledgments should be placed at the end of the article.}
}
\subtitle{Stochastic Integer Programming Library version 2.0}

\titlerunning{SIPLIB 2.0}        % if too long for running head

\author{Yongkyu Cho \and 
		Kibaek Kim \and
        Cong Han Lim \and
        James Luedtke \and
        Jeffrey Linderoth
}

%\authorrunning{Short form of author list} % if too long for running head

\institute{Yongkyu Cho, Kibaek Kim \at
           Mathematics and Computer Science Division, Argonne National Laboratory, Lemont, IL 60439, USA \\
           \email{choy@anl.gov; kimk@anl.gov}
           \and
           Cong Han Lim \and James Luedtke \and Jeffrey Linderoth \at
           Department of Industrial and Systems Engineering, University of Wisconsin-Madison Madison, WI 53706, USA \\
%          \and
%           Cong Han Lim \at
           \email{clim9@wisc.edu; jim.luedtke@wisc.edu; linderoth@wisc.edu}
%           \and 
%           James Luedtke \at
%           \email{jim.luedtke@wisc.edu}
%           \and 
%           Jeffrey Linderoth \at
%           \email{linderoth@wisc.edu}
}

\date{Received: date / Accepted: date}
% The correct dates will be entered by the editor


\maketitle

\begin{abstract}
We present a collection of stochastic integer programming problem instances.
\keywords{Stochastic Integer Programming \and Problem Instances}
% \PACS{PACS code1 \and PACS code2 \and more}
% \subclass{MSC code1 \and MSC code2 \and more}
\end{abstract}

\section{Introduction}
(What is SIPLIB?) The \texttt{SIPLIB} \cite{web:SIPLIB1} is an abbreviated term of the Stochastic Integer Programming (SIP) LIBrary firstly contructed in 2002 by Shabbir Ahmed and his colleagues. The library has been providing a collection of test instances to facilitate computational and algorithmic research in SIP. Some new test problems with instances have been added to \texttt{SIPLIB} gradually and now it contains nine different problems in total.  The instances are basically given in the standard \texttt{SMPS} format accompanied with additional information including parameter data, size of the instance in terms of the number of rows, columns, and integers, benchmarking information such as best known objective value or bounds, optimality gap, and solution time.

(Motivation) At the time \texttt{SIPLIB} appeared, it provided enoughly large-sized instances that is reasonable to argue that the performance of algorithm is remarkable if it solves the instances. The state-of-art in SIP combined with the speedup in computing machinery, however, makes many instances of \texttt{SIPLIB} trivial so that we have no more basis to use \texttt{SIPLIB} for showing the excellence of new solution methods. At this point, we are motivated to develop the second version of \texttt{SIPLIB}, say \texttt{SIPLIB 2.0} that provides larger-sized test instances with higher degree of tailorability, e.g., users can expand the size of instance as much as they want in terms of the number of scenarios included.

(What SIP is and our restriction) Stochastic programming (SP) is a framework for modeling optimization problems that involve uncertainty. Whereas optimization problems are typically formulated with known parameters, the problems in real world contain some unknown parameters in many cases. For details on SP, see, e.g., \cite{web:SPS,book:BL2011}. SIP is a branch of SP that indicates any type of SP including at least one integer decision variable. The integers can be placed anywhere in general SIP. However, we restrict our focus on two-stage SIP that contains integer variables (including binary) in its second stage throughout this paper and \texttt{SIPLIB 2.0}. The main reason is that the class of SIP is most widely used to model real world problems. Moreover, two-stage SIP itself has enough difficulties that have not been conquered yet even without any other details like chance-constraints and multi-stages. The main difficulty in solving two-stage SIP is that the second-stage value function is not necessarily convex, but only lower semicontinuous. Thus, the standard decomposition approaches that work nicely for stochastic \textit{linear} programs, break down when the second stage integer variables are present \cite{journal:AG2003}. Hereinafter, we use the term SIP to indicate the two-stage SIP that contains integer variables in its second stage.

(SMPS, Julia)
We provide the test sets in two formats: \texttt{SMPS} files (*.cor, *.tim, *.stoch) and \texttt{Julia} files (*.jl). \texttt{SMPS} is widely used to describe stochastic linear and quadratic programs. Once having \texttt{SMPS} files of a problem instance, we can directly solve it using various mixed integer linear program (MILP) solvers such as \texttt{CPLEX}, \texttt{GUROBI}, and \texttt{CBC}. One can also use the existing open-source SIP solvers like \texttt{DSP} \cite{journal:KZ2015}, \texttt{PySP} \cite{journal:WWH2012}, and \texttt{SMI} \cite{web:SMI} given that \texttt{SMPS} files. A drawback of \texttt{SMPS} format is its low readability by human, which we decided to provide \texttt{Julia} files to let users be able to easily read problems and tailor the instances.

(Convenience of Julia) \texttt{Julia} is an open source high-level, high-performance dynamic programming language for numerical computing. It is also known as its good performance, approaching that of statically-compiled languages like \texttt{C} \cite{journal:BEKS2017}. The syntax of \texttt{Julia} is simple and should feel familiar to anyone who has experienced in another high-level languages like \texttt{MATLAB} or \texttt{Python}. A \texttt{Julia} package called \texttt{JuMP} (Julia for Mathematical Programming \cite{web:JuMP}) provides a domain-specific modeling language for mathematical optimization embedded in \texttt{Julia}. \texttt{JuMP} enables us to easily translate mathematical model to \texttt{JuMP.Model}-type object. Some structured mathematical models like SIP can also be translated to the \texttt{JuMP.Model}-type object combined with the package \texttt{StructJuMP} \cite{web:StructJuMP}. Once we have a \texttt{Julia} code for constructing \texttt{JuMP.Model}-type object, it is easy to generate instances whenever we need to modify the original mathematical model. For each problem in \texttt{SIPLIB 2.0}, we provide a \texttt{Julia} script for constructing \texttt{JuMP.Model}-type object. We also provide a \texttt{Julia} script (SmpsWriter.jl) for converting any \texttt{JuMP.Model}-type objects to \texttt{SMPS} files for users' convenience. Those who feel the given instances are not large enough can simply generate more scenario data by just modifying the parameter in \texttt{Julia} script corresponding to the number of scenarios.

(Contribution of SIPLIP2.0) By \texttt{SIPLIB 2.0}, we provide 1) richer collection of test instances for computational and algorithmic research in SIP with benchmarking computational results, 2) not only \texttt{SMPS} files but also \texttt{Julia} files that are easily readable/tailorable. Hence, the users can obtain as large-sized instances as they need by generating new scenarios and including them into instances. For those who want to utilize the instances in the legacy \texttt{SIPLIB} with strong tailorability provided by \texttt{SIPLIB 2.0}, we include the original \texttt{SIPLIB} instances as well.

\section{Stochastic integer programming}
In this section, we explain general description of SIP. This includes formal mathematical formulation, existing general solution methods to solve the SIPs, and currently available software libraries.
\subsection{Formulation}
In this subsection, we introduce the form of SIP of interest. The notations and dimensional information are summarized in Table \ref{notation:SIP}. We are interested in finding solution for two-stage SIP of the form: 
\begin{align}
z:=\min_{x\in X}{\left\{c^\top x + \mathcal{Q}(x):\ Ax\ge b\right\}}, \label{eq:SIP_1}
\end{align}
where $\mathcal{Q}(x):=\EE_{\pmb{\xi}}\left[ \phi\left( h(\pmb{\xi})-T(\pmb{\xi})x \right) \right]$ is the recourse function given the random variable (r.v.) $\pmb{\xi}$. We assume that $\pmb{\xi}$ follows a known discrete probability distribution with the finite realizations, called \textit{scenarios}, $\xi_1,\cdots,\xi_r$ and respective nonnegative probabilities $p_1,\cdots,p_r$, i.e., $p_i:=\PP[\pmb{\xi}=\xi_i]$ for $i\in\{1,\ldots,r\}$. When the distribution is continuous, we can aproximate it by a suitably discretized distribution. The real-valued map $\phi:\mathbb{R}^{m_2}\to\mathbb{R}$ is the optimal value of the second-stage problem defined by
\begin{align}
\phi(s;\xi):=\min_{y\in Y}\left\{ q(\xi)^\top y:\ W(\xi)y \ge s \right\},\ s\in\mathbb{R}^{m_2},
\end{align}
where the r.v. $\pmb{\xi}$ is realized by an arbitrary scenario $\xi$.
The sets $X\subseteq \mathbb{R}^{n_1}$ and $Y\subseteq\mathbb{R}^{n_2}$ represent integer or binary restrictions on a subset of the decision variables $x$ and $y$, respectively. 
The first-stage problem data comprise $A$, $b$, and $c$. The second-stage data are given by $T(\xi_j)$, $W(\xi_j)$, $h(\xi_j)$, and $q(\xi_j)$ (for dimensional information refer to Table \ref{notation:SIP}). Hereinafter, we use the simplified notations $(T_j,W_j,h_j,q_j)$.
The SIP (\ref{eq:SIP_1}) can be rewritten in the extensive form
\begin{subequations}
\begin{align}
z=\min_{x_j,y_j}\ &\sum_{j=1}^{r}p_j\left(c^{\top}x_j+q_j^{\top}y_j\right)	\label{eq:SIP_2-1}\\ 
\mathrm{s.t.}\ &\sum_{j=1}^{r}H_j x_j=0 \label{eq:SIP_2-2} \\
\ &(x_j,y_j)\in G_j,\quad \forall j\in\{1,\ldots,r\},	\label{eq:SIP_2-3}
\end{align}
\end{subequations}
where the scenario feasibility set $G_j$ is defined as
\begin{align}
G_j:=\left\{ (x_j,y_j): \ Ax_j\ge b,\  T_j x_j+W_j y_j\ge h_j,\ (x_j,y_j)\in X\times Y  \right\}.
\end{align}
The nonanticipativity constraints in (\ref{eq:SIP_2-2}) stand for the equations $x_1=x_r$ and $x_j=x_{j-1}$ for $j=2,\ldots,r$, and $H_j$ is a suitable $rn_1\times n_1$ matrix. We assume that SIP does not necessarily have relatively complete recourse. We recall that without this property there can exist an $\hat{x}\in X$ satisfying $A\hat{x}\ge b$ for which there does not exist a recourse $y\in\mathbb{R}^{m_2}$ satisfying $(\hat{x},y)\in G_j$ for some $j$. In other words, not every choice of the first-stage variables is guaranteed to have feasible recourse for all scenarios.
\begin{table}[H]
	\caption{Summary of notations in SIP formulation}
	\label{notation:SIP}
	\resizebox{\textwidth}{!}
	{
		\begin{tabular}{ll}
			\toprule
			\textbf{Scalas}	&	\\  \midrule
			$\pmb{\xi}$	& r.v. denoting scenario that can have value in the set $\{\xi_1,\cdots,\xi_r\}$ 	\\			
			$z\in\mathbb{R}$ & optimal objective value of the SIP \\ 
			$r\in\mathbb{N}$	& number of scenarios	\\	
			$j\in\{1,\cdots,r\}$	& index denoting scenarios	\\
			$p_j\in[0,1]$ & probability that the scenario $j$ happens, i.e., $\PP[\pmb{\xi}=\xi_j]$ \\ \midrule
			\textbf{Sets} &  \\  \midrule
			$X\subseteq\mathbb{R}^{n_1}$	& first-stage polyhedral set (real, integer, binary)\\
			$Y\subseteq\mathbb{R}^{n_2}$	& second-stage polyhedral set (real, integer, binary)\\
			$G_j$	& scenario feasibility set\\ \midrule
			\textbf{Vectors} &   \\  \midrule
			$x,x_j\in\mathbb{R}^{n_1}$	& first-stage decision vector	\\
			$c\in \mathbb{R}^{n_1}$	& first-stage cost vector\\
			$b\in\mathbb{R}^{m_1}$	& first-stage RHS vector\\
			$y,y_j\in\mathbb{R}^{n_2}$	& second-stage decision vector	\\
			$q_j\equiv q(\xi_j)\in\mathbb{R}^{n_2}$	& second-stage cost vector \\
			$h_j\equiv h(\xi_j)\in\mathbb{R}^{m_2}$	& second-stage RHS vector\\
			$\mathbf{0}\in\mathbb{R}^{rn_1}$	& vector filled with zeros \\ \midrule
			\textbf{Matrices} &  \\  \midrule
			$A\in\mathbb{R}^{m_1\times n_1}$	& first-stage constraint matrix corresponds to the decision vector $x_j$\\
			$W_j\equiv W(\xi_j)\in\mathbb{R}^{m_2\times n_2}$	& second-stage constraint matrix corresponds to the decision vector $y_j$\\
			$T_j\equiv T(\xi_j)\in\mathbb{R}^{m_2\times n_1}$	& second-stage constraint matrix corresponds to the decision vector $x_j$\\
			$H_j\equiv H(\xi_j)\in\mathbb{R}^{rm_1\times n_1}$	&	nonanticipativity constraints matrix \\ \midrule
			\textbf{Functions}	&	\\ \midrule
			$\phi:\mathbb{R}^{m_2}\to\mathbb{R}$	& second stage program optimal value given the realization of scenario $\xi_j$	\\
			$\mathcal{Q}:\mathbb{R}^{n_1}\to\mathbb{R}$	& recourse function (the expectation of $\phi\left( h(\pmb{\xi})-T(\pmb{\xi})x \right)$ over the r.v. $\pmb{\xi}$) 	\\
			\hline
		\end{tabular}
	}
\end{table} 


\subsection{Solution methods}
\subsubsection{Stage-wise decomposition algorithm}
\subsubsection{Scenario-wise decompostion algorithm}

\subsection{Software libraries}
\subsubsection{Modeling languages}
\subsubsection{Solvers}


\section{The test sets}

\subsection{The type of problems}
\begin{table}[H]
	\centering
	\caption{The type of problems in \texttt{SIPLIB 2.0}}
	\label{table:prob_type}
	\resizebox{\textwidth}{!}{%
		\begin{tabular}{@{}llll@{}}
			\toprule
			Problem name & Description                                                        & Reference             & Is in \texttt{SIPLIB}? \\ \midrule
			\texttt{DCAP}         & Dynamic capacity planning with stochastic demand                   & Ahmed and Garcia      & Yes                      \\
			\texttt{MPTSPs}       & Multi-path traveling salesman problem with stochastic travel costs & \cite{journal:MPT2014,journal:PGM2017,journal:TPP2017}        & Yes                      \\
			\texttt{\texttt{SMKP}}         & Stochastic multiple knapsack problem                               & Angulo et al.         & Yes                      \\
			\texttt{SIZES}        & Optimal product substitution with stochastic demand                & Jorjani et al.        & Yes                      \\
			\texttt{SSLP}         & Stochastic server location problem                                 & Ntaimo and Sen        & Yes                      \\
			\texttt{WECC}         & Wind power stochastic unit commitment				              & Papavasiliou and Oren & No                       \\ \bottomrule
		\end{tabular}%
	}
\end{table}

\begin{table}[H]
	\centering
	\caption{The components of problems in \texttt{SIPLIB 2.0}}
	\label{table:prob_class}
	\begin{tabular}{@{}lllll@{}}
		\toprule
		& \multicolumn{2}{l}{1st stage}                              & \multicolumn{2}{l}{2nd stage}                             \\ \midrule
		Problem & Variable                    & Constraint                   & Variable                    & Constraint                  \\ \midrule
		\texttt{DCAP}    & \texttt{real}, \texttt{bin} &                              & \texttt{bin}                & \texttt{PAR}, \texttt{M01}1 \\
		\texttt{MPTSPs}  & \texttt{bin}, \texttt{real} & \texttt{PAR}2, \texttt{GEN}1 & \texttt{bin}                & \texttt{GEN}2               \\
		\texttt{SMKP}    & \texttt{bin}                & \texttt{KNA}1                & \texttt{bin}                & \texttt{KNA}1               \\
		\texttt{SIZES}   & \texttt{real}, \texttt{int} & \texttt{VBD}1, \texttt{GEN}2 & \texttt{real}, \texttt{int} & \texttt{IKN}1               \\
		\texttt{SSLP}    & \texttt{bin}                & \texttt{IVK}1, \texttt{GEN}1 & \texttt{real}, \texttt{int} & \texttt{GEN}2               \\
		\texttt{WECC}    &                             &                              &                             &                             \\ \bottomrule
	\end{tabular}

\end{table}

\subsection{The instance catalog}

\subsubsection{Instance naming rule}

% Please add the following required packages to your document preamble:
% \usepackage{booktabs}
% \usepackage{graphicx}
\begin{table}[H]
	\centering
	\caption{Problem-specific instance naming rules}
	\label{my-label}
	\resizebox{\textwidth}{!}{%
		\begin{tabular}{@{}lll@{}}
			\toprule
			Problem & Instance name                 & Description                                                                                                          \\ \midrule
			\texttt{DCAP}    &                              &                                                                                                                     \\
			\texttt{MPTSPs}  & \texttt{MPTSPs\_D}x\texttt{\_N}y\texttt{\_S}z &node distribution strategy \texttt{D}x, number of nodes y, and number of scenarios z \\
			\texttt{SMKP}    &                              &                                                                                                                     \\
			\texttt{SIZES}   &                             &                                                                                                                     \\
			\texttt{SSLP}    &                              &                                                                                                                     \\
			\texttt{WECC}    &                              &                                                                                                                     \\ \bottomrule
		\end{tabular}%
	}
\end{table}
\section{How to run a test, generate new instance, and convert to \texttt{SMPS}}

We explain the structure of \texttt{SIPLIB 2.0}. We explain procedure to generate new instances with user-generated scenario data using \texttt{Julia} scripts. The problem-specific descriptions are given in Section \ref{sec:prob_desc}. We explain how to convert \texttt{JuMP.Model}-type object to \texttt{SMPS} files.

\section{Implementation of SMPS Writer}

We describe our Julia implementation, how to model SIP and generate SMPS files..


\section{Problem descriptions} \label{sec:prob_desc}

In this section, we introduce each problem in \texttt{SIPLIB 2.0}.  For each problem, we provide various size of instances as a default. We also explain the scenario data generation procedures. Due to limited access to the original data and inconsistencies present in reference papers, we selectively choose the methods from the references and modify some of them without harming validity. Also, we guess some parameters about scenario generation to make the procedure clear.  %This does not mean that users need to know the whole procedure in order to generate new scenario data. Those who feel the given instances are not large enough can simply generate more scenario data by just modifying the parameter corresponding to the number of scenarios.

\subsection{\texttt{SIZES}: Selection of an optimal subset of sizes}
\subsubsection{Mathematical formulation}
\begin{table}[H]
	\caption{Notations for \texttt{SIZES}}
	\label{notation}
	\resizebox{\textwidth}{!}
	{
		\begin{tabular}{ll}
			\toprule
			\textbf{Index sets} &  \\ 
			$N$ & \textrm{index set of items ($i,j\in N$)} \\ 
			$T$ & \textrm{index set of time periods ($t\in T$)} \\ 
			$\mathcal{L}$ & \textrm{index set of scenarios ($l\in\mathcal{L}$)}\\
			\textbf{Parameters} &   \\ 
			$d_{its}$ &	demand for item $i$ at time $t$ under scenario $l$\\
			$p_{i}$ & unit production cost for item $i$\\
			$s$	& setup cost for producing any item $i$\\
			$r$ & unit cutting cost\\ 
			$\PP(l)$ & the probability of occerence of scenario $l$\\
			\textbf{Variables} &  \\ 
			$y_{it}$ (1st stage) & number of units of size $i$ produced at time $t$ \\
			$x_{ijtl}$ (2nd stage)& number of units of size $i$ cut to meet demand for smaller size $j$ at time $t$ under scenario $l$\\ 
			$z_{itl}$ (2nd stage)& 1 if we produce size $i$ at time $t$ under scenario $l$, 0 otherwise\\
			\hline
		\end{tabular}
	}
\end{table} 

\begin{subequations}
	\begin{align}
	(\texttt{SIZES})\ \textrm{min}\ &\sum_{t\in T}\sum_{i\in N} p_i y_{it} + \sum_{l\in \mathcal{L}} \PP(s)\sum_{t\in T}\left[\sum_{i\in N} sz_{its}+r\sum_{i\in N\backslash\{1\}}\sum_{j=1}^{i-1}x_{ijtl}\right] \\
	\textrm{s.t.}\ &\sum_{i\in N}y_{it}\le c_{tl},\quad \forall t\in T,\ \forall l\in \mathcal{L}, \\
	&\sum_{i=j}^N x_{ijtl} \ge d_{itl},\quad\forall j\in N,\ \forall t\in T,\  \forall l\in \mathcal{L},\\
	&\sum_{t'\in T}\sum_{j=1}^{i}x_{ijt'l}\le\sum_{t'\in T}y_{it'}, \quad\forall j\in N,\ \forall t\in T,\ \forall l\in \mathcal{L},\\
	&y_{it}\le Mz_{itl},\quad\forall i \in N,\ \forall t\in T,\ \forall l\in \mathcal{L},\\
	&x_{ijtl}\in\mathbb{Z}_+,\quad\forall i\in N,\ \forall j\in N,\ \forall t\in T,\ \forall l\in \mathcal{L},\\
	&y_{it}\in\mathbb{Z}_+,\quad \forall j\in N,\ \forall t\in T,\\
	&z_{itl}\in\{0,1\},\quad\forall i\in N,\ \forall t\in T,\ \forall l\in \mathcal{L}.
	\end{align}
\end{subequations}

\subsubsection{Scenario data generation}

\subsection{\texttt{MPTSPs}: Mutli-path Traveling Salesman Problem with Stochastic Travel Times}
\subsubsection{Mathematical formulation}
\begin{table}[H]
	\caption{Notations for \texttt{MPTSPs}}
	\label{notation}
	\resizebox{\textwidth}{!}
	{
		\begin{tabular}{ll}
			\toprule
			\textbf{Index sets} &  \\ 
			$N$ & \textrm{index set of nodes ($i,j,l\in N$)} \\ 
			$K_{ij}$ & \textrm{index set of paths between nodes $i$ and $j$ ($k\in K_{ij}$)} \\ 
			$\mathcal{S}$ & \textrm{index set of scenarios ($s\in \mathcal{S}$)}\\
			\textbf{Parameters} &   \\ 
			$c_{ij}^{ks}$ & \textrm{unit random travel time of path $k$ between nodes $i,j$ under scenario $s$} \\ 
			$\bar{c}_{ij}$ & \textrm{estimation of the mean unit travel time (expectation of $c_{ij}^{ks}$ over all $s$ and $k$)} \\ 
			$e_{ij}^{ks}$ & \textrm{the error on the travel time estimated for path $k$ under scenario $s$} \\ 
			$\PP(s)$ & \textrm{the probability of occurence of scenario $s$} \\ 
			\textbf{Variables} &  \\ 
			$\phi_{ij}$ (1st stage) & \textrm{the nonnegative real-valued flow on arc $(i,j)$}\\
			$x_{ij}^{ks}$ (1st stage) & \textrm{1 if node $j$ is visited just after node $i$, 0 otherwise} \\ 
			$y_{ij}$ (2nd stage)& \textrm{1 if path $k$ between nodes $i,j\in N$ is selected at the second stage, 0 otherwise} \\ 

			\hline
		\end{tabular}
 	}
\end{table} 

\begin{subequations}
\begin{align}
(\texttt{MPTSPs})\ \textrm{min}\ &\sum_{i\in N}\sum_{j\in N}\bar{c}_{ij}y_{ij}+\sum_{s\in \mathcal{S}}\PP(s)\sum_{i\in N}\sum_{j\in N}\sum_{k\in K_{ij}}e_{ij}^{ks}x_{ij}^{ks} \\
\textrm{s.t.}\ &\sum_{j\in N:j\neq i}y_{ij}=1,\quad\forall i\in N,\\
&\sum_{i\in N:i\neq j}y_{ij}=1,\quad\forall j\in N,\\
&\sum_{j\in N}\phi_{lj}-\sum_{i\in N: i\neq 1}\phi_{il}=1,\quad\forall l\in N\backslash\{1\},\\
&\phi_{ij}\le \left(|N|-1\right)y_{ij},\quad\forall i\in N\backslash\{1\},\ \forall j\in N,\\
&\sum_{k\in K_{ij}}x_{ij}^{ks}=y_{ij},\quad\forall i\in N,\ \forall j\in N,\ \forall s\in \mathcal{S}, \\
&x_{ij}^{ks}\in\{0,1\},\quad\forall i\in N,\ \forall j\in N,\ \forall k\in K_{ij},\ \forall s\in \mathcal{S},\\
&y_{ij}\in \{0,1\},\quad \forall i\in N,\ \forall j\in N,\\
&\phi_{ij} \ge 0, \quad \forall i\in N,\ \forall j\in N.
\end{align}
\end{subequations}

\subsubsection{Scenario data generation}
We follow the scenario generation methods described through the references \cite{journal:MPT2014,journal:PGM2017,journal:TPP2017}. For \texttt{MPTSPs}, there are three mainly distinguished characteristics for each instance: the nodes partition strategy ($D\in\{D0,D1,D2,D3\}$, explanation on each strategy is forthcoming), the number of nodes ($|N|\in\{2,3,\ldots\}$), and the number of scenarios ($|S|\in\{1,2,\ldots\}$). Another important charicteristic $|K_{ij}|\in\{1,2,3,\ldots\}$ is the number of paths for each edge which is fixed by 3 as a default following \cite{journal:TPP2017}. Once we decide $D$, $|N|$, and $|S|$ by $D=D\textrm{x}$, $|N|=\textrm{y}$, and $|S|=\textrm{z}$, each instance is named by \texttt{MPTSPs\_Dx\_Ny\_Sz}.

The nodes are distributed in a circle with radius equal to $r$ km. We use Cartesian coordinate system where the geometric center of the circle is $(r,r)$. The nodes are distinguished by two subsets: \textit{central} and \textit{suburban}. If the Euclidean distance between a node and the geometric center is less than or equal to the half of the radius ($r/2$), then the node is of \textit{central} type. Otherwise, if the Euclidean distance is greater than the half of the radius, the node is of \textit{suburban} type. Each arc between any two nodes $i$ and $j$ is either \textit{homogeneous} or \textit{heterogeneous}. If the two nodes are of the same type of node, i.e., both are \textit{central} or both are \textit{suburban}, the type of the arc is \textit{homogeneous}. Otherwise, the type of the arc is \textit{heterogeneous}. Later, the travel time of each path between two nodes are affected by the type of arc. 

The nodes are generated by one of the following distribution strategies:
\begin{itemize}
	\item $D0$: All the nodes are \textit{central}.
	\item $D1$: All the nodes are \textit{suburban}.
	\item $D2$: 3/4 of the nodes are \textit{central} and the remaining 1/4 are \textit{suburban}.
	\item $D3$: 1/2 of the nodes are \textit{central} and the remaining 1/2 are \textit{suburban}.
\end{itemize}

Given $D,|N|$ and $|S|$, the next procedure can be summarized as follows:
\begin{enumerate}
	\item Generate $|N|$ nodes based on the predetermined strategy $D$. Then, the nodes are generated by acceptance-rejection procedure with uniform random number generation. Again following \cite{journal:TPP2017}, we fix $r=7$\textit{km}. 
	\item Calculate Euclidean distances between the nodes ($EC_{ij}$).
	\item We guess and fix the deterministic velocity profile by 40\textit{km/h} for the \textit{central} nodes and 80\textit{km/h} for the \textit{suburban} nodes: $v^c=40$ and $v^s=80$.
	\item Generate random travel times ($c_{ij}^{ks}$) for each scenario $s$.
	\begin{itemize}
		\item The velocity for traveling arc $(i,j)$ is affected by its arc type.
		%\item Each arc has $|K_{ij}|$ paths (we have fixed it $|K_{ij}|=3$ for all $i,j$).
		\item If the arc is \textit{homogeneous}, the random travel time of all the paths are generated only based on the corresponding velocity profile.
		\item If the arc is \textit{heterogeneous}, $\ceil*{\frac{|K_{ij}|}{3}}$ paths are generated based on $v^c=40$ and the remaining paths are generated based on $v^s=80$. 
		\item The velocities are distributed by $Unif(\frac{v}{2},2v)$ for $v=v^c,v^s$.
		\item In summary, if the arc $(i,j)$ is \textit{homogeneous}, 
		\begin{align*}
		c_{ij}^{ks}\sim\left\{ \begin{array}{ll} \frac{EC_{ij}}{Unif(\frac{v^c}{2},2v^c)} & \textrm{if $i,j$ are both \textit{central},} \\
		\frac{EC_{ij}}{Unif(\frac{v^s}{2},2v^s)} & \textrm{if $i,j$ are both \textit{suburban}}	\end{array} \right.\ \forall k\in K_{ij}.
		\end{align*}
		\item Otherwise, if $(i,j)$ is \textit{heterogeneous},
		\begin{align*}
		c_{ij}^{ks}\sim\left\{ \begin{array}{ll} \frac{EC_{ij}}{Unif(\frac{v^c}{2},2v^c)} & \textrm{for $k\in\left\{1,\ldots,\ceil*{\frac{|K_{ij}|}{3}}\right\}$,} \\
		\frac{EC_{ij}}{Unif(\frac{v^s}{2},2v^s)} & \textrm{for $k\in\left\{\ceil*{\frac{|K_{ij}|}{3}}+1,\ldots,|K_{ij}|\right\}$.}	\end{array} \right.
		\end{align*}
	\end{itemize}
	\item Finally, we multiply 3600 for each component of $c_{ij}^{ks}$ to convert the unit from \textit{hours} to \textit{seconds}.
\end{enumerate}
\section{Solution report}

\section{Concluding remarks}

%\begin{acknowledgements}
%If you'd like to thank anyone, place your comments here
%and remove the percent signs.
%\end{acknowledgements}

% BibTeX users please use one of
\bibliographystyle{spbasic}      % basic style, author-year citations
%\bibliographystyle{spmpsci}      % mathematics and physical sciences
%\bibliographystyle{spphys}       % APS-like style for physics
%\bibliography{}   % name your BibTeX data base
\begin{thebibliography}{9} 
	\bibitem{example}
	Author1 and Author2, paper paper paper paper, Journal Title 68 (2011), 1207--1221.
	\bibitem{web:SIPLIB1}
	S. Ahmed, R. Garcia, N. Kong, L. Ntaimo, G. Parija, F. Qiu, S. Sen. SIPLIB: A Stochastic Integer Programming Test Problem Library. http://www.isye.gatech.edu/~sahmed/siplib, 2015.
	\bibitem{web:SPS}
	SPS: Stochastic Programming Society (https://stoprog.org/what-stochastic-programming).
	\bibitem{book:BL2011}
	Introduction to Stochastic Programming, J. R. Birge, F. Louveaus.
	\bibitem{journal:AG2003}
	S. Ahmed and R. Garcia. "Dynamic Capacity Acquisition and Assignment under Uncertainty," Annals of Operations Research, vol.124, pp. 267-283, 2003.
	\bibitem{journal:KZ2015}
	Kibaek Kim and Victor M. Zavala. "Algorithmic Innovations and Software for the Dual Decomposition Method applied to Stochastic Mixed-Integer Programs" Mathematical Programming Computation, 2015.
	\bibitem{journal:WWH2012}
	J.-P. Watson, D. L. Woodruff, and W. E. Hart, PySP: modeling and solving stochastic programs in Python, Mathematical Programming Computation, 2012.
	\bibitem{web:SMI}
	SMI - Stochastic Modeling Interface. https://github.com/coin-or/Smi
	\bibitem{journal:BEKS2017}
	Julia: A Fresh Approach to Numerical Computing. Jeff Bezanson, Alan Edelman, Stefan Karpinski and Viral B. Shah (2017) SIAM Review, 59: 65–98.
	\bibitem{web:JuMP}
	JuMP - Julia for Mathematical Optimization, https://jump.readthedocs.io/en/latest/index.html
	\bibitem{web:StructJuMP}
	StructJuMP - Parallel algebraic modeling framework for block structured optimization models in Julia, https://github.com/StructJuMP/StructJuMP.jl
	\bibitem{journal:MPT2014}
	F. Maggioni, G. Perboli, and R. Tadei, The multi-path traveling salesman problem with stochastic travel socts: Building realistic instances for city ligistics applications, Transportation Research Procedia, 2014
	\bibitem{journal:PGM2017}
	G. Perboli, L. Gobbato, and F. Maggioni, A progressive hedging method for the multi-path travelling salesman problem with stochastic travel times, IMA Journal of Management Mathematics, 2017
	\bibitem{journal:TPP2017}
	R. Tadei, G. Perboli, and F. Perfetti, The multi-path traveling salesman problem with stochastic travel costs, EURO Journal on Transportation and Logistics, 2017
\end{thebibliography}
\end{document}
% end of file template.tex

