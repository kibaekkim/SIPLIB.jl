%%%%%%%%%%%%%%%%%%%%%%% file template.tex %%%%%%%%%%%%%%%%%%%%%%%%%
%
% This is a general template file for the LaTeX package SVJour3
% for Springer journals.          Springer Heidelberg 2010/09/16
%
% Copy it to a new file with a new name and use it as the basis
% for your article. Delete % signs as needed.
%
% This template includes a few options for different layouts and
% content for various journals. Please consult a previous issue of
% your journal as needed.
%
%%%%%%%%%%%%%%%%%%%%%%%%%%%%%%%%%%%%%%%%%%%%%%%%%%%%%%%%%%%%%%%%%%%
%
% First comes an example EPS file -- just ignore it and
% proceed on the \documentclass line
% your LaTeX will extract the file if required
\begin{filecontents*}{example.eps}
%!PS-Adobe-3.0 EPSF-3.0
%%BoundingBox: 19 19 221 221
%%CreationDate: Mon Sep 29 1997
%%Creator: programmed by hand (JK)
%%EndComments
gsave
newpath
  20 20 moveto
  20 220 lineto
  220 220 lineto
  220 20 lineto
closepath
2 setlinewidth
gsave
  .4 setgray fill
grestore
stroke
grestore
\end{filecontents*}
%
\RequirePackage{fix-cm}
%
%\documentclass{svjour3}                     % onecolumn (standard format)
%\documentclass[smallcondensed]{svjour3}     % onecolumn (ditto)
\documentclass[smallextended]{svjour3}       % onecolumn (second format)
%\documentclass[twocolumn]{svjour3}          % twocolumn
%
\smartqed  % flush right qed marks, e.g. at end of proof
%
\usepackage{graphicx}
\usepackage{hyperref}
\usepackage{float}
\usepackage{mathrsfs,amsmath,amssymb,amscd,mathtools}

\makeatletter
\let\scshape\relax % to avoid a warning
\DeclareRobustCommand\scshape{%
	\not@math@alphabet\scshape\relax
	\ifnum\pdf@strcmp{\f@family}{\familydefault}=\z@
	\fontfamily{qbk}%
	\fi
	\fontshape\scdefault\selectfont}
\makeatother

% \usepackage{mathptmx}      % use Times fonts if available on your TeX system
%
% insert here the call for the packages your document requires
%\usepackage{latexsym}
% etc.
%
% please place your own definitions here and don't use \def but
% \newcommand{}{}
%
% Insert the name of "your journal" with
% \journalname{myjournal}
%
\begin{document}

\title{SIPLIB 2.0
%\thanks{Grants or other notes
%about the article that should go on the front page should be
%placed here. General acknowledgments should be placed at the end of the article.}
}
\subtitle{Stochastic Integer Programming Library version 2.0}

\titlerunning{SIPLIB 2.0}        % if too long for running head

\author{Yongkyu Cho \and 
		Kibaek Kim \and
        Cong Han Lim \and
        James Luedtke \and
        Jeffrey Linderoth
}

%\authorrunning{Short form of author list} % if too long for running head

\institute{Yongkyu Cho, Kibaek Kim \at
           Mathematics and Computer Science Division, Argonne National Laboratory, Lemont, IL 60439, USA \\
           \email{choy@anl.gov; kimk@anl.gov}
           \and
           Cong Han Lim \and James Luedtke \and Jeffrey Linderoth \at
           Department of Industrial and Systems Engineering, University of Wisconsin-Madison Madison, WI 53706, USA \\
%          \and
%           Cong Han Lim \at
           \email{clim9@wisc.edu; jim.luedtke@wisc.edu; linderoth@wisc.edu}
%           \and 
%           James Luedtke \at
%           \email{jim.luedtke@wisc.edu}
%           \and 
%           Jeffrey Linderoth \at
%           \email{linderoth@wisc.edu}
}

\date{Received: date / Accepted: date}
% The correct dates will be entered by the editor


\maketitle

\begin{abstract}
We present a collection of stochastic integer programming problem instances.
\keywords{Stochastic Integer Programming \and Problem Instances}
% \PACS{PACS code1 \and PACS code2 \and more}
% \subclass{MSC code1 \and MSC code2 \and more}
\end{abstract}

\section{Introduction}
(What is SIPLIB?) The \texttt{SIPLIB} \cite{web:SIPLIB1} is an abbreviated term of the Stochastic Integer Programming (SIP) LIBrary firstly contructed in 2002 by Shabbir Ahmed and his colleagues. The library has been providing a collection of test instances to facilitate computational and algorithmic research in SIP. Some new test problems with instances have been added to \texttt{SIPLIB} gradually and now it contains nine different problems in total.  The instances are basically given in the standard \texttt{SMPS} format accompanied with additional information including parameter data, size of the instance in the number of rows, columns, and integers, benchmark data such as best known objective value or bounds, optimality gap, and solution time.

(Motivation) At the time \texttt{SIPLIB} appeared, it provided enoughly large-sized instances that is reasonable to argue that the performance of algorithm is remarkable if it solves the instances. The state-of-art in SIP combined with the speedup in computing machinery, however, makes many instances of \texttt{SIPLIB} trivial so that we have no more basis to use \texttt{SIPLIB} for showing the excellence of new solution methods. At this point, we are motivated to develop the second version of \texttt{SIPLIB}, say, \texttt{SIPLIB 2.0} that provides larger-sized test instances with higher degree of tailorability, e.g., users can expand the size of instance sas much as they want in terms of the included number of scenarios.

(What SIP is and our restriction) Stochastic programming (SP) is a framework for modeling optimization problems that involve uncertaity. Whereas optimization problems are typically formulated with known parameters, the problems in real world contain some unknown parameters in many cases. For details on SP, see, e.g., \cite{web:SPS,book:BL2011}. SIP is a branch of SP that indicates any type of SP including at least one integer decision variable. The integers can be placed anywhere in general SIP. However, we restrict our focus on two-stage SIP that contains integer variables (including binary) in its second stage throughout this paper and \texttt{SIPLIB 2.0}. The main reason is that the class of SIP is most widely used to model real world problems. Moreover, two-stage SIP itself has enough difficulties that have not been conquered yet without any other details like chance-constraints and multi-stages. The main difficulty in solving two-stage SIP is that the second-stage value function is not necessarily convex, but only lower semicontinuous. Thus, the standard decomposition approaches that work nicely for stochastic \textit{linear} programs, break down when the second stage integer variables are present \cite{journal:AG2003}. Hereinafter, we use the term SIP to indicate the two-stage SIP that contains integer variables in its second stage.

(SMPS, Julia)
We provide the problem sets in two formats: \texttt{SMPS} files (*.cor, *.tim, *.stoch) and \texttt{Julia} files (*.jl). \texttt{SMPS} is widely used to describe stochastic linear and quadratic programs. Once having \texttt{SMPS} files of a problem instance, we can directly solve it using various mixed integer linear program (MILP) solvers such as \texttt{CPLEX}, \texttt{GUROBI}, and \texttt{CBC}. One can also use the existing open-source SIP solvers like \texttt{DSP} \cite{journal:KZ2015}, \texttt{PySP} \cite{journal:WWH2012}, and \texttt{SMI} \cite{web:SMI} given that \texttt{SMPS} files. One drawback of \texttt{SMPS} format is its low readability by human, which we decided to provide \texttt{Julia} files to let users be able to easily read problems and tailor the instances.

(Convenience of Julia) \texttt{Julia} is an open source high-level, high-performance dynamic programming language for numerical computing. It is also known as its good performance, approaching that of statically-compiled languages like \texttt{C} \cite{journal:BEKS2017}. The syntax of \texttt{Julia} is simple and should feel familiar to anyone who has experienced in other high-level languages like \texttt{MATLAB} or \texttt{Python}. A \texttt{Julia} package called \texttt{JuMP} (Julia for Mathematical Programming \cite{web:JuMP}) provides a domain-specific modeling language for mathematical optimization embedded in \texttt{Julia}. \texttt{JuMP} enables us to easily translate mathematical model to \texttt{JuMP.Model}-type object. Some structured mathematical models like SIP can also be translated to the \texttt{JuMP.Model}-type object combined with the package \texttt{StructJuMP} \cite{web:StructJuMP}. Once we have \texttt{JuMP.Model}-type object of mathematical model, it is easy to modify and convert to \texttt{SMPS} files. For each problem in \texttt{SIPLIB 2.0}, we provide a \texttt{Julia} script for constructing \texttt{JuMP.Model}-type object. We also provide a \texttt{Julia} script (SmpsWriter.jl) for converting any \texttt{JuMP.Model}-type objects to \texttt{SMPS} files for users' convenience.

(Contribution of SIPLIP2.0) By \texttt{SIPLIB 2.0}, we provide 1) richer collection of test instances for computational and algorithmic research in SIP with benchmarking computational results, 2) not only \texttt{SMPS} files but also \texttt{Julia} files that are easily readable/tailorable. Hence, the users can obtain as large-sized instances as they need by generating new scenarios and including them into instances. For those who want to utilize the instances in the legacy \texttt{SIPLIB} with strong tailorability provided by \texttt{SIPLIB 2.0}, we include the original \texttt{SIPLIB} instances as well.
\section{Stochastic Integer Programming}

\subsection{Formulation}


\subsection{Algorithms}
\subsubsection{Stage-wise Decomposition Algorithm}
\subsubsection{Scenario-wise Decompostion Algorithm}

\subsection{Software Libraries}
\subsubsection{Modeling Languages}
\subsubsection{Solvers}

\section{Test Sets Description}

In this section, we explain each problem in \texttt{SIPLIB 2.0}. To aviod confusion, we define the terms used: problem, instance, . The term \textit{problem} indicates  For some problems, we provide more than one formulation based on the reference papers. 
\subsection{Mutli-Path Traveling Salesman Problem with Stochastic Travel Times (MPTSPS)}
\subsubsection{Problem Class}
\begin{table}[H]
	\centering
	%\caption{My caption}
	\label{mptsps-class}
	\begin{tabular}{|c|c|c|}
		\hline
		& 1$^{st}$ stage & 2$^{nd}$ stage \\ \hline
		Variables   & Bin & Bin               \\ \hline
		Constraints &                &                \\ \hline
	\end{tabular}
\end{table}
\subsubsection{Notation}
\begin{table}[H]
	\caption{Notations corresponding to problem}
	\label{notation}
	\resizebox{\textwidth}{!}
	{
		\begin{tabular}{ll}
			\hline 
			\textbf{Index sets} &  \\ 
			$T$ & \textrm{index set of time slots $(t=1,\ldots,|T|)$} \\ 
			$A$ & \textrm{index set of applications $(i=1,\ldots,|A|)$} \\ 
			$S$ & \textrm{index set of servers $(j=1,\ldots,|S|)$}\\
			$F_j$ & \textrm{index set of frequency options of server $j\in S$ $(f=0,\ldots,|F_j|)$} \\ 
			\textbf{Parameters} &   \\ 
			$\lambda_{it}$ & \textrm{average workload of application $i\in A$ that arrives in time interval $t\in T$} \\ 
			$U_j$  & \textrm{maximum number of applications installable to server $j\in S$} \\ 
			$C_{jf}$ & \textrm{processing capacity of server $j\in S$ under frequency $f\in F_j$} \\ 
			$\beta_{jft}$ & \textrm{cost incurred when server $j\in S$ runs at frequency $f\in F_j$ in time interval $t\in T$} \\ 
			$\rho$ & \textrm{target load for all servers (surrogate for quality of service)}\\ 
			\textbf{Decision variables} &  \\ 
			$a_{ij}$ (virtualization) & \textrm{1 if application $i\in A$ is installed to server $j\in S$, 0 otherwise} \\ 
			$x_{jft}$ (server provisioning) & \textrm{1 if server $j\in S$ runs at frequency $f\in F_j$ during time interval $t\in T$, 0 otherwise} \\ 
			$r_{ijt}$ (routing) & \textrm{fraction of workloads of application $i\in A$ assigned to server $j\in S$ in time  interval $t\in T$}\\
			\hline
		\end{tabular}
	}
\end{table} 

\subsubsection{Formulations}
\begin{align}
(\textrm{MIP}_O): \textrm{minimize    } &\sum_{j\in S}\sum_{f\in F_j}\sum_{t\in T}\beta_{jft}x_{jft},\\
\textrm{subject to} \nonumber\\ 
&\sum_{i\in A}a_{ij} \le U_j, \quad \forall j\in S,\\
&\sum_{j\in S}r_{ijt}=1, \quad \forall i\in A, \forall t\in T,\\
&r_{ijt} \le a_{ij}, \quad \forall i\in A, \forall j\in S, \forall t\in T, \\
&\sum_{i\in A}\lambda_{it}r_{ijt} \le \rho \sum_{f\in F_j}C_{jf}x_{jft}, \quad \forall j\in S, \forall t\in T,\\
&\sum_{f\in F_j}x_{jft}=1, \quad \forall j\in S, \forall t\in T, \\
&\sum_{i\in S}r_{ijt} \le U_j(1-x_{j0t}), \quad \forall j\in S, \forall t\in T, \\
&a_{ij}, x_{jft}\in \{0,1\}, \quad \forall j\in S, \forall f\in F_j, \forall t\in T  \\
&0 \le r_{ijt} \le 1, \quad \forall j\in S, \forall t\in T
\end{align}

\begin{align}
(\textrm{subtour elimination}): \textrm{minimize    } &\sum_{j\in S}\sum_{f\in F_j}\sum_{t\in T}\beta_{jft}x_{jft},\\
\textrm{subject to} \nonumber\\ 
&\sum_{i\in A}a_{ij} \le U_j, \quad \forall j\in S,\\
&\sum_{j\in S}r_{ijt}=1, \quad \forall i\in A, \forall t\in T,\\
&r_{ijt} \le a_{ij}, \quad \forall i\in A, \forall j\in S, \forall t\in T, \\
&\sum_{i\in A}\lambda_{it}r_{ijt} \le \rho \sum_{f\in F_j}C_{jf}x_{jft}, \quad \forall j\in S, \forall t\in T,\\
&\sum_{f\in F_j}x_{jft}=1, \quad \forall j\in S, \forall t\in T, \\
&\sum_{i\in S}r_{ijt} \le U_j(1-x_{j0t}), \quad \forall j\in S, \forall t\in T, \\
&a_{ij}, x_{jft}\in \{0,1\}, \quad \forall j\in S, \forall f\in F_j, \forall t\in T  \\
&0 \le r_{ijt} \le 1, \quad \forall j\in S, \forall t\in T
\end{align}

\section{Implementation of SMPS Writer}

We describe our Julia implementation, how to model SIP and generate SMPS files..

\section{Solution Report}

\section{Concluding Remarks}

%\begin{acknowledgements}
%If you'd like to thank anyone, place your comments here
%and remove the percent signs.
%\end{acknowledgements}

% BibTeX users please use one of
\bibliographystyle{spbasic}      % basic style, author-year citations
%\bibliographystyle{spmpsci}      % mathematics and physical sciences
%\bibliographystyle{spphys}       % APS-like style for physics
%\bibliography{}   % name your BibTeX data base
\begin{thebibliography}{9} 
	\bibitem{example}
	Author1 and Author2, paper paper paper paper, Journal Title 68 (2011), 1207--1221.
	\bibitem{web:SIPLIB1}
	S. Ahmed, R. Garcia, N. Kong, L. Ntaimo, G. Parija, F. Qiu, S. Sen. SIPLIB: A Stochastic Integer Programming Test Problem Library. http://www.isye.gatech.edu/~sahmed/siplib, 2015.
	\bibitem{web:SPS}
	SPS: Stochastic Programming Society (https://stoprog.org/what-stochastic-programming).
	\bibitem{book:BL2011}
	Introduction to Stochastic Programming, J. R. Birge, F. Louveaus.
	\bibitem{journal:AG2003}
	S. Ahmed and R. Garcia. "Dynamic Capacity Acquisition and Assignment under Uncertainty," Annals of Operations Research, vol.124, pp. 267-283, 2003.
	\bibitem{journal:KZ2015}
	Kibaek Kim and Victor M. Zavala. "Algorithmic Innovations and Software for the Dual Decomposition Method applied to Stochastic Mixed-Integer Programs" Mathematical Programming Computation, 2015.
	\bibitem{journal:WWH2012}
	J.-P. Watson, D. L. Woodruff, and W. E. Hart, PySP: modeling and solving stochastic programs in Python, Mathematical Programming Computation, 2012.
	\bibitem{web:SMI}
	SMI - Stochastic Modeling Interface. https://github.com/coin-or/Smi
	\bibitem{journal:BEKS2017}
	Julia: A Fresh Approach to Numerical Computing. Jeff Bezanson, Alan Edelman, Stefan Karpinski and Viral B. Shah (2017) SIAM Review, 59: 65–98.
	\bibitem{web:JuMP}
	JuMP - Julia for Mathematical Optimization, https://jump.readthedocs.io/en/latest/index.html
	\bibitem{web:StructJuMP}
	StructJuMP - Parallel algebraic modeling framework for block structured optimization models in Julia, https://github.com/StructJuMP/StructJuMP.jl
\end{thebibliography}
\end{document}
% end of file template.tex

