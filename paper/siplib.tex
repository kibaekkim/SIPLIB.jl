%%%%%%%%%%%%%%%%%%%%%%% file template.tex %%%%%%%%%%%%%%%%%%%%%%%%%
%
% This is a general template file for the LaTeX package SVJour3
% for Springer journals.          Springer Heidelberg 2010/09/16
%
% Copy it to a new file with a new name and use it as the basis
% for your article. Delete % signs as needed.
%
% This template includes a few options for different layouts and
% content for various journals. Please consult a previous issue of
% your journal as needed.
%
%%%%%%%%%%%%%%%%%%%%%%%%%%%%%%%%%%%%%%%%%%%%%%%%%%%%%%%%%%%%%%%%%%%
%
% First comes an example EPS file -- just ignore it and
% proceed on the \documentclass line
% your LaTeX will extract the file if required
\begin{filecontents*}{example.eps}
%!PS-Adobe-3.0 EPSF-3.0
%%BoundingBox: 19 19 221 221
%%CreationDate: Mon Sep 29 1997
%%Creator: programmed by hand (JK)
%%EndComments
gsave
newpath
  20 20 moveto
  20 220 lineto
  220 220 lineto
  220 20 lineto
closepath
2 setlinewidth
gsave
  .4 setgray fill
grestore
stroke
grestore
\end{filecontents*}
%
\RequirePackage{fix-cm}
%
%\documentclass{svjour3}                     % onecolumn (standard format)
%\documentclass[smallcondensed]{svjour3}     % onecolumn (ditto)
\documentclass[smallextended]{svjour3}       % onecolumn (second format)
%\documentclass[twocolumn]{svjour3}          % twocolumn
%
\smartqed  % flush right qed marks, e.g. at end of proof
%
\usepackage{graphicx}
%
% \usepackage{mathptmx}      % use Times fonts if available on your TeX system
%
% insert here the call for the packages your document requires
%\usepackage{latexsym}
% etc.
%
% please place your own definitions here and don't use \def but
% \newcommand{}{}
%
% Insert the name of "your journal" with
% \journalname{myjournal}
%
\begin{document}

\title{SIPLIB 2.0
%\thanks{Grants or other notes
%about the article that should go on the front page should be
%placed here. General acknowledgments should be placed at the end of the article.}
}
\subtitle{Stochastic Integer Programming Library version 2.0}

\titlerunning{SIPLIB 2.0}        % if too long for running head

\author{Kibaek Kim \and
        Cong Han Lim \and
        James Luedtke \and
        Jeffrey Linderoth
}

%\authorrunning{Short form of author list} % if too long for running head

\institute{Kibaek Kim \at
           Mathematics and Computer Science Division, Argonne National Laboratory, Lemont, IL 60439, USA \\
           \email{kimk@anl.gov}
           \and
           Cong Han Lim \and James Luedtke \and Jeffrey Linderoth \at
           Department of Industrial and Systems Engineering, University of Wisconsin-Madison Madison, WI 53706, USA \\
           \and
           Cong Han Lim \at
           \email{clim9@wisc.edu}
           \and 
           James Luedtke \at
           \email{jim.luedtke@wisc.edu}
           \and 
           Jeffrey Linderoth \at
           \email{linderoth@wisc.edu}
}

\date{Received: date / Accepted: date}
% The correct dates will be entered by the editor


\maketitle

\begin{abstract}
We present a collection of stochastic integer programming problem instances.
\keywords{Stochastic Integer Programming \and Problem Instances}
% \PACS{PACS code1 \and PACS code2 \and more}
% \subclass{MSC code1 \and MSC code2 \and more}
\end{abstract}

\section{Introduction}

Stochastic integer programming is ... Small test instances are available in Shabbir's website. We need more..

\section{Stochastic Integer Programming}

\subsection{Formulation}

\subsection{Algorithms}

Benders, dual, ...

\subsection{Software Libraries}

modeling, solving, ...

\section{Problem Instances}

We introduce the set of problem instances. The instances are available in SMPS file format.

characteristics, categorization

\section{Implementation of SMPS Writer}

We describe our Julia implementation, how to model SIP and generate SMPS files..

\section{Solution Report}

\section{Concluding Remarks}

%\begin{acknowledgements}
%If you'd like to thank anyone, place your comments here
%and remove the percent signs.
%\end{acknowledgements}

% BibTeX users please use one of
%\bibliographystyle{spbasic}      % basic style, author-year citations
%\bibliographystyle{spmpsci}      % mathematics and physical sciences
%\bibliographystyle{spphys}       % APS-like style for physics
%\bibliography{}   % name your BibTeX data base

\end{document}
% end of file template.tex

