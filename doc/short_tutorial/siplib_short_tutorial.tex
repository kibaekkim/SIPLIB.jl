\documentclass[11pt]{article}
\usepackage[left=2.5cm,right=2.5cm,top=3cm,bottom=3cm,a4paper]{geometry}
\usepackage{graphicx}
\usepackage{dhucs}
\usepackage{mathrsfs,amsmath,amssymb,amscd,mathtools,amsthm}
\usepackage{array}
\usepackage{algorithm2e}
\usepackage{algpseudocode}
\usepackage{hyperref}
\usepackage{indentfirst}
\usepackage{subfig}
\usepackage{adjustbox}
\usepackage{multirow}
\usepackage{booktabs}
\usepackage{color}
\usepackage{setspace}
\usepackage{rotating}
\usepackage{filecontents}
\usepackage{authblk}
\usepackage{soul}
\usepackage{float}
\usepackage{caption}
\usepackage{xspace}
\usepackage{listings}
\usepackage[usenames,dvipsnames]{xcolor}
\usepackage[flushleft]{threeparttable}
\onehalfspacing

%\graphicspath{{./figures/}}

% set shortcuts
%% Math notations
\DeclareMathOperator*{\PP}{\mathbb{P}}
\DeclareMathOperator*{\EE}{\mathbb{E}}
\DeclarePairedDelimiter\ceil{\lceil}{\rceil}
\DeclarePairedDelimiter\floor{\lfloor}{\rfloor}

%% Package, Library, abbreviate terms
\newcommand{\siplibtwo}{\textsf{SIPLIB 2.0}}
\newcommand{\siplib}{\textsf{SIPLIB}}
\newcommand{\miplib}{\textsf{MIPLIB 2010}}
\newcommand{\smps}{\textsf{SMPS}}
\newcommand{\mps}{\textsf{MPS}}
\newcommand{\mpsx}{\textsf{MPSX}}
\newcommand{\jump}{\textsf{JuMP}}
\newcommand{\structjump}{\textsf{StructJuMP}}

% Problems
\newcommand{\dcap}{\textsf{DCAP}}
\newcommand{\dclp}{\textsf{DCLP}}
\newcommand{\mptsps}{\textsf{MPTSPs}}
\newcommand{\sizes}{\textsf{SIZES}}
\newcommand{\smkp}{\textsf{SMKP}}
\newcommand{\sslp}{\textsf{SSLP}}
\newcommand{\suc}{\textsf{SUC}}

% Solvers
\newcommand{\dsp}{\textsf{DSP}}
\newcommand{\pysp}{\textsf{PySP}}
\newcommand{\pyomo}{\textsf{Pyomo}}
\newcommand{\cplex}{\textsf{CPLEX}}

%% Programming languages
\newcommand{\julia}{\texttt{Julia}}
\newcommand{\python}{\texttt{Python}}
\newcommand{\clang}{\texttt{C}}
\newcommand{\cpp}{\texttt{C++}}
\newcommand{\matlab}{\texttt{MATLAB}}

%% Siplib.jl related terms
\newcommand{\jumpmodel}{\texttt{JuMP.Model}}
\newcommand{\siplibjl}{\texttt{Siplib.jl}}

% for Julia script
\lstdefinelanguage{julia}
{
	sensitive=true,	
	basicstyle=\ttfamily\scriptsize,
	columns=fullflexible, % make sure to use fixed-width font, CM typewriter is NOT fixed width
	numbers=left, 
	numberstyle=\small\ttfamily\color{Gray},
	stepnumber=0,              
	numbersep=10pt, 
	numberfirstline=true, 
	numberblanklines=true, 
	tabsize=4,
	lineskip=-1.5pt,
	extendedchars=true,
	breaklines=true,        
	keywordstyle=\color{Blue}\bfseries,
	identifierstyle=, % using emph or index keywords
	commentstyle=\sffamily\color{OliveGreen},
	stringstyle=\color{Maroon},
	showstringspaces=false,
	showtabs=false,
	upquote=false,
	keywordsprefix=\@,
	keywords={exit,whos,edit,load,is,isa,isequal,typeof,tuple,ntuple,uid,hash,finalizer,convert,promote,
		subtype,typemin,typemax,realmin,realmax,sizeof,eps,promote_type,method_exists,applicable,
		invoke,dlopen,dlsym,system,error,throw,assert,new,Inf,Nan,pi,im,begin,while,for,in,return,
		break,continue,macro,quote,let,if,elseif,else,try,catch,end,bitstype,ccall,do,using,module,
		import,export,importall,baremodule,immutable,local,global,const,Bool,Int,Int8,Int16,Int32,
		Int64,Uint,Uint8,Uint16,Uint32,Uint64,Float32,Float64,Complex64,Complex128,String,Symbol,Any,Nothing,None,
		function,type,typealias,abstract,struct, mutable},
	comment=[l]{\#},
	%morecomment=[s]{#=}{=#},
	morestring=[d]\',
	morestring=[b]\",
}

\begin{document}
\nocite{*}
\title{Using \siplibjl}
\author{Yongkyu Cho}
\affil{POSTECH\\ E-mail: jyg1124@postech.ac.kr}
\maketitle

\section{Prerequisites}
We assume that you are in Linux environment. To use \siplibjl, you need to perform the following steps:
\begin{enumerate}
	\item Download $\julia\ge 0.6.2$ and set up.
	\item Install \julia\ packages: \texttt{Distributions.jl}, \texttt{StructJuMP.jl}, \texttt{PyPlot.jl} by executing
	\begin{itemize}
		\item \texttt{Pkg.add("Distributions")}
		\item \texttt{Pkg.add("StructJuMP")}
		\item \texttt{Pkg.add("PyPlot")}
	\end{itemize}
	Then, execute \texttt{Pkg.update()} to make them up-to-date.
	\item Download and place the \siplibjl\ package to any directory (say \texttt{dir}) in your computer
	\item Open a terminal and change working directory to \texttt{dir/Siplib/src/}:
	\begin{itemize}
		\item \texttt{cd dir/Siplib/src}	
	\end{itemize}
	\item Run \julia\ in that directory
	\item excute \texttt{include("Siplib.jl")}
	\item excute \texttt{using Siplib}
\end{enumerate}
Then, you are all set to use \siplibjl. To make it sure, execute the following line to generate DCAP\_2\_2\_2\_10 instance:
\begin{lstlisting}[frame=single,language=julia]
julia> generateSMPS(:DCAP, [2,2,2,10])
\end{lstlisting}
If it works well, you will see the three files in \texttt{dir/Siplib/instance}:
\begin{itemize}
	\item DCAP\_2\_2\_2\_10.cor
	\item DCAP\_2\_2\_2\_10.tim
	\item DCAP\_2\_2\_2\_10.sto
\end{itemize}

\section{Basic usage}
\subsection{Generating \smps\ instance}
Use \texttt{generateSMPS(problem, params\_arr)} with proper values in Table \ref{table:numparameter} to generate \smps\ instances, for example:
\begin{lstlisting}[frame=single,language=julia]
julia> generateSMPS(:DCAP, [2,2,2,10])
\end{lstlisting}
The default directory in which the files are stored is \texttt{dir/Siplib/instance}. You can change the directory by specifying explicit path:
\begin{lstlisting}[frame=single,language=julia]
julia> generateSMPS(:DCAP, [2,2,2,10], "another/directory")
\end{lstlisting}
\texttt{generateSMPS(problem, params\_arr)} has four more optional keyword arguments: \texttt{seed},  \texttt{splice}, \texttt{genericnames}, \texttt{lprelax}. For details, please see the manual.

\begin{table}[H]
	\centering
	\resizebox{\textwidth}{!}{%
	\begin{threeparttable}
		\caption{Acceptable values for \texttt{problem} and \texttt{params\_arr} arguments pairs}
		\label{table:numparameter}
		\begin{tabular}{@{}ccl@{}}
			\toprule
			\texttt{problem}  & \texttt{params\_arr}     & Remark                                                                                                                                     \\ \midrule
			\texttt{:DCAP}               & \texttt{[R, T, N, $\mathcal{S}$]} & All parameters are integer.                                                                                                                \\
			\texttt{:MPTSPs}             & \texttt{[D, N, $\mathcal{S}$]}    & String $\texttt{D}\in \{\mathrm{``D0"}, \mathrm{``D1"}, \mathrm{``D2"}, \mathrm{``D3"}\}$. All other parameters are integer.                                                              \\
			\texttt{:SIZES}              & \texttt{[$\mathcal{S}$]}          & Integer $\mathcal{S}$.                                                                                   \\
			\texttt{:SMKP}               & \texttt{[I, $\mathcal{S}$]}       & All parameters are integer.                                                                                                                \\
			\texttt{:SSLP}               & \texttt{[I, J, $\mathcal{S}$]}    & All parameters are integer.                                                                                                                \\
			\texttt{:SUC}                & \texttt{[D, $\mathcal{S}$]}       & \multicolumn{1}{l}{String $\texttt{D}\in \{\mathrm{``FallWD"}, \mathrm{``FallWE"}, \mathrm{``WinterWD"}, \mathrm{``WinterWE"}, $}                                             \\
			\multicolumn{1}{l}{}         & \multicolumn{1}{l}{}            & \multicolumn{1}{r}{$\mathrm{``SpringWD"}, \mathrm{``SpringWE"}, \mathrm{``SummerWD"}, \mathrm{``SummerWE"} \}.$} Integer $\mathcal{S}\le 1000$. \\ \bottomrule
		\end{tabular}
		\begin{tablenotes}
			\item Note: $\mathcal{S}$ is always the number of scenarios.
		\end{tablenotes}
	\end{threeparttable}
	}
\end{table}


\subsection{Plotting sparsity}
Use \texttt{generateSparsityPlots(problem, params\_arr)} to generate the sparsity plots in constraint matrix. Excute the following lines:
\begin{lstlisting}[frame=single,language=julia]
julia> generateSparsityPlots(:DCAP, [2,2,2,10])
\end{lstlisting}
The default directory in which the plots are stored is \texttt{dir/Siplib/plot}. You can change the directory by specifying explicit path:
\begin{lstlisting}[frame=single,language=julia]
julia> generateSparsityPlots(:DCAP, [2,2,2,10], "another/directory")
\end{lstlisting}

\subsection{Reporting size and sparsity}
Use \texttt{getSize(problem, params\_arr)} and \texttt{getSparsity(problem, params\_arr)} to get the size and sparsity information. Excute the following lines to construct objects that contain the information:
\begin{lstlisting}[frame=single,language=julia]
julia> size = getSize(:DCAP, [2,2,2,10])
julia> sparsity = getSparsity(:DCAP, [2,2,2,10])
\end{lstlisting}
\end{document}
